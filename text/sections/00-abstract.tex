US public universities have systematically substituted away from tenure-track/tenured professors towards contingent lecturers for the last thirty years, while their states have systematically cut higher education funding despite rising enrolment.
This paper asks whether the changes the faculty composition at US public universities are caused by the fall in state funding, using a shift-share approach to instrument for state funding.
I find that universities employ $4.4$\% more lecturers per student in response to a fall of $10$\% per student funding, and correspondingly decrease their reliance on tenure-track faculty by employing $1.4$\% fewer assistant professors and $1.2$\% fewer full professors per student.
%Dynamic estimates show that these effects linger multiple years after the initial funding cut.
In a secondary analysis, using data on all faculty at Illinois public universities, I find that incumbent professors' salaries, promotion rate, and quit rate are all not affected by funding cuts, implying that faculty composition change arose by limiting the hiring of new tenure-track/tenured professors at public universities.
These results show the institutional effects of stagnating state funding for higher education, and exhibit one possible mechanism for deteriorating student outcomes at US public universities over this same time period.

\vfill
\noindent
\textbf{Key-words:}
State and Local Budget and Expenditures,
Higher Education,
Public Sector Labour Markets

\vspace{0.05cm}
\noindent
\textbf{JEL Codes:} H72, I23, J45
