\noindent
State support for higher education per-student has stagnated while universities have substituted towards contingent lecturers and away from tenure-track/tenured professors.
This analysis relates these changes in the faculty composition at public universities to the fall in state funding per student for higher education, using a shift-share approach to identify changes in a state funding among public universities.
A decrease in a university's funding of 10\%, via a shock to state higher education funding, decreases the number of assistant professors per student at a public university by 1.4\% and full professors by 1.2\%, yet increases the number of lecturers per student by 4.4\%.
Local projection estimates show that these effects linger for up to three years after the initial funding shock.
Analysis of all the professors at Illinois public universities 2011-2021 shows that incumbent professors are not affected by the changes in state funding, and suggestive evidence shows that these changes in faculty composition arose by impacting the hiring of new professors at public universities impacted by the funding shocks.
These results show the long-term effects of stagnating state support for higher education, and raise questions about the direction that public education heads as these financial headwinds show no sign of dissipating.

\vfill
\noindent
\textbf{Key-words:}
State and Local Budget and Expenditures,
Higher Education,
Public Sector Labour Markets

\vspace{0.1cm}
\noindent
\textbf{JEL Codes:} H72, I23, J45
