US public universities have systematically substituted away from tenure-track/tenured professors towards lecturers over the last thirty years, while states have systematically cut public funding for higher education.
This paper shows that falling state funding caused changes in faculty composition, using a shift-share approach to instrument for state funding cuts.
Universities employ $4.4$\% more lecturers per student following a $10$\% funding cut, and correspondingly rely less on tenure-track faculty by employing $1.4$\% fewer assistant professors and $1.2$\% fewer full professors per student.
In a secondary analysis, using data on all faculty at Illinois public universities, incumbent professors' salaries, promotion rate, and quit rate are not affected by funding cuts, implying that composition change arose by limiting hiring of new tenure-track/tenured professors.
These results show institutional effects of stagnating public funding for education, exhibiting one possible mechanism for deteriorating student outcomes over this same time period.

\vfill
\noindent
\textbf{Key-words:}
State and Local Budget and Expenditures,
Higher Education,
Public Sector Labour Markets

\vspace{0.05cm}
\noindent
\textbf{JEL Codes:} H72, I22, J45
