US universities have systematically substituted away from tenure-track/tenured professors towards contingent lecturers for the last thirty years, while at the same time states have limited how much they fund higher education.
This paper connects the changes the faculty composition at US public universities to the fall in the states' funding for their public universities, using a shift-share approach to instrument for state funding and identify the causal effects of stagnating funding for higher education.
A fall of $10$\% in funding for a public university, via a state funding shock, decreases the number of assistant professors per student by $1.4$\% and full professors by $1.2$\%, yet increases the number of lecturers per student by $4.4$\%; long-run estimates show that the faculty composition is affected both in the same year as the funding shock, and two to three years following.
Analysis of all the professors at Illinois public universities shows that incumbent professors are not affected by substantial funding shocks over 2011-2021, with suggestive evidence showing that the faculty changes arose by limiting the hiring of new professors to replace leaving tenure-track/tenured professors.
These results show the institutional effects of stagnating state funding for higher education, and raise questions about the direction that public education heads as these financial headwinds show no sign of dissipating.

\vfill
\noindent
\textbf{Key-words:}
State and Local Budget and Expenditures,
Higher Education,
Public Sector Labour Markets

\vspace{0.1cm}
\noindent
\textbf{JEL Codes:} H72, I23, J45
