US public universities universities have systematically substituted away from tenure-track/tenured professors towards contingent lecturers for the last thirty years, while their states have systematically reduced the funding to these universities relative to enrolment.
This paper connects the changes the faculty composition at US public universities to the fall in the states' funding for their public universities, using a shift-share approach to instrument for state funding and identify the causal effects of stagnating funding for higher education.
A fall in funding of \$1,000 per student leads to univeristies employing 6 more lecturers on average; relative to enrolment, a fall of $10$\% in funding per student decreases the number of assistant professors per student by $1.4$\% and full professors by $1.2$\%, yet increases the number of lecturers per student by $4.4$\%.
Dynamic estimates show that these effects linger multiple years after the initial funding shock.
Analysis of all faculty at Illinois public universities shows that incumbent professors' salaries, promotion rate, and quit rate are all are not affected by funding cuts over 2011-2021, implying that the composition change arose by limiting the hiring of new tenure-track/tenured professors.
These results show the institutional effects of stagnating state funding for higher education, and exhibit one possible mechanism for deteriorating student outcomes at US public universities over this same time period.

\vfill
\noindent
\textbf{Key-words:}
State and Local Budget and Expenditures,
Higher Education,
Public Sector Labour Markets

\vspace{0.05cm}
\noindent
\textbf{JEL Codes:} H72, I23, J45
