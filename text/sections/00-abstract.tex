US public universities employ more lecturers and fewer professors than at any other point in the last thirty years, relative to student enrolment.
At the same time, state funding for higher education has stagnated.
This paper shows that the decline in state funding led to a substitution away from professors toward lecturers at US public universities.
Using a shift-share approach to instrument for state funding, I find that universities employ 4.4\% more lecturers per student following a 10\% funding cut.
This shift is accompanied by a reduction in assistant professors and full professors per student by 1.4\% and 1.2\%, respectively.
Incumbent professors' salaries, promotion rates, and quit rates at Illinois universities remain unaffected by funding cuts, indicating that the substitution arose from limiting the hiring of new tenure-track/tenured professors.
Stagnating state funding impacts public universities and faculty, likely contributing to the deterioration of student outcomes at public universities since the 1990s.

\vfill
\noindent
\textbf{Key-words:}
Faculty,
Higher Education,
State Funding

\vspace{0.05cm}
\noindent
\textbf{JEL Codes:} H72, I22, J45
