Public universities employ more lecturers and fewer professors than at any other point in the last thirty years, relative to student enrolment.
At the same time, state funding for higher education has stagnated.
This paper shows falling state funding caused a substitution away from professors towards lecturers at US public universities, using a shift-share approach to instrument for state funding cuts relative to enrolment.
Universities employ $4.4$\% more lecturers per student following a $10$\% funding cut, relying less on professors by employing $1.4$\% fewer assistant professors and $1.2$\% fewer full professors per student.
Incumbent professors' salaries, promotion rate, and quit rate at Illinois universities are not affected by funding cuts, so that the substitution arose by limiting hiring of new tenure-track/tenured professors.
Stagnating public funding affects faculty and public universities as institutions, likely contributing to worsening student outcomes at public universities since the 1990s.

\vfill
\noindent
\textbf{Key-words:}
State and Local Budget and Expenditures,
Higher Education,
Public Sector Labour Markets

\vspace{0.05cm}
\noindent
\textbf{JEL Codes:} H72, I22, J45
