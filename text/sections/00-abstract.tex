While states have limited their funding for higher education for the last forty years, US universities have systematically substituted away from tenure-track/tenured professors towards contingent lecturers.
This analysis relates the changes in the faculty composition at US public universities to the fall in state funding for higher education, using a shift-share approach to isolate the causal effects of changes in state funding.
A $-10$\% fall in a university's funding, via a state funding shock, decreases the number of assistant professors per student by $-1.4$\% and full professors by $-1.2$\%, yet increases the number of lecturers per student by 4.4\%; long-run estimates show the effects linger for multiple years after the initial funding shock.
Analysis of all the professors at Illinois public universities 2011-2021 shows that incumbent professors are not affected by the changes in state funding, and suggestive evidence shows that these changes arose by limiting the hiring of new professors at public universities to replace leaving tenure-track/tenured professors.
These results show the institutional effects of stagnating state funding for higher education, and raise questions about the direction that public education heads as these financial headwinds show no sign of dissipating.

\vfill
\noindent
\textbf{Key-words:}
State and Local Budget and Expenditures,
Higher Education,
Public Sector Labour Markets

\vspace{0.1cm}
\noindent
\textbf{JEL Codes:} H72, I23, J45
