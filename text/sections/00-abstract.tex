\noindent
State support for higher education per-student has stagnated at the same time as public universities have reduced their employment of tenure-track or tenured professors, and increased their employment of temporary contingent lecturers.
This analysis relates these changes in the faculty composition at public universities to the fall in per-student revenues from state funding for higher education.
A shift-share instrumental variables approach addresses endogeneity in decisions for state higher education funding by exploiting (1) yearly change in each state's funding for their public universities and (2) each institution's reliance on that support.
A decrease in state funding of 10\%, via a shock to state funding, decreases the number of assistant professors per student at a public university by 1.4\% and full professors by 1.2\%, yet increases the number of lecturers per student by 4.4\%.
Local projection estimates show that these effects linger for up to three years after the initial funding shock.
Analysis of all the professors at Illinois public universities 2011-2021 shows that incumbent professors are not affected by the changes in state support, while tenured-track hiring counts fall over this time period, showing that these changes in faculty composition arose by reduced hiring of tenure-track and tenured professors at public universities.
These results show the long-term effects of stagnating state support for higher education, and raises questions about the direction that public education heads as these financial headwinds show no sign of dissipating.

\noindent
\textbf{Key-words:}
State and Local Budget and Expenditures,
Higher Education,
Public Sector Labour Markets

\noindent
\textbf{JEL Codes:} H72, I23, J45
