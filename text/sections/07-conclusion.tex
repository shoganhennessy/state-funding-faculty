%%%%%%%%%%%%%%%%%%%%%%%%%%%%%%%%%%%%%%%%%
%% Conclusion section
\section{Summary and Concluding Remarks}
\label{sec:conclusion}
Public universities have systematically substituted away from tenure-track and tenured professors, toward lecturers, in the face of persistent declines in state funding for higher education.
The effects on incumbent professors in Illinois are non-distinguishable from zero, and suggestive evidence shows that funding cuts limited hiring of new professors, implying that substitution towards lecturers is driven by reduced hiring of new professors.
Public universities are using more contingent lecturers to teach their students, while private universities continue to employ more tenure-track and tenured professors than their public counterparts, and each year the gap widens.

This work contributes to the literature along two primary dimensions.
First, this work provides an explanation for the increased reliance on lecturers and away from tenure-track and tenured professors at US public universities, by isolating the effects funding cuts from state-wide funding shifts in both the short-- and long--run.
Second, this work investigates how individual faculty are affected by changes in state funding for their university using a dataset new in the economic literature (IBHED).
These data allow for detailed analysis of thousands of faculty salaries, and employment outcomes, in the Illinois university system, and show that incumbent faculty were relatively unaffected by state funding cuts to their employers.

While state funding for higher education has stagnated, costs of education has been rising at the same time, so that funding cuts have even more impact today than they did three decades ago.
I show that these financial headwinds led to systematic change at public universities, leading to a substitution away from stable employment contracts (tenure, professorship) and towards less stable arrangements (adjunct contracts, lecture positions).
These results show large changes in faculty composition, and that stagnation in state support explains at least a third of the observed shift away from tenured professors and towards lecturers.
This likely had knock-on effects on instruction, contribiting to worsening student outcomes at US public universities since the early 1990s --- while private universities were not exposed to financial headwinds of the same magnitude or persistence over the same time period.
While public universities educate the majority of US higher education students, we should worry about the effects of restricting their funding has on faculty, and the wider impact on higher education as a whole.
