%%%%%%%%%%%%%%%%%%%%%%%%%%%%%%%%%%%%%%%%%
%% Conclusion section
\section{Summary and Concluding Remarks}
\label{sec:conclusion}

This analysis investigates how the recent stagnation in state support for higher education has affected public universities, and their faculty composition.
This work contributes to the literature along two primary dimensions.
Firstly, by isolating changes in state funding on public universities via state funding shocks, the analysis provides a plausible mechanism for the observed negative effects on students outcomes that this stagnation has brought \citep{NBERw23736,NBERw27885}.
Secondly, this analysis considers individual professors as the unit of analysis using a dataset (IBHED) new in the economic literature.
These data allow for detailed analysis of thousand's of professors' salaries, and are a starting point for further analysis of the determinants of professors wages in the Illinois public university system.

Findings show that public universities have substituted away from tenure-track and tenured professors, towards non-tenured lecturers, in the face of persistent declines in state funding for higher education.
The effects on incumbent professors in the state of Illinois are non-distinguishable from zero, which implies that the effect is driven by changes in university hiring of new professors.
Each year, public universities are using more contingent lecturers to teach their education students, while private universities continue to employ more tenure-track and tenured professors than their public counterparts.
Private universities have always had more tenured professor per student than public institutions, but each year the gap widens.

While costs education have been rising in the US, thanks to multiple overlapping trends \citep{ehrenberg2012}, public universities have also dealt with declining state support.
It is natural to expect that such headwinds will lead to systematic change at public universities, changes which are not neutral to the composition of faculty, or to their goals of research and education.
These results show large changes in faculty composition, and that stagnation in state support explains at least a third of the observed shift away from tenured professors and towards contingent lecturers.
At the same time, private universities were not exposed to financial headwinds of the same magnitude or persistence.
If public universities' hiring of faculty is disrupted by a fall in state funding, then we can expect that private universities will benefit (in a competitive sense) by being able to hire better professors from relatively resource constrained public universities.
While public universities continue to educate the majority of higher education students in the US, we should worry about the effects that restricting their funding has on the composition of faculty, and the impact on higher education as a whole.
