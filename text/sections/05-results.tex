%%%%%%%%%%%%%%%%%%%%%%%%%%%%%%%%%%%%%%%%%
%% Results section
\section{Results}
\label{sec:results}

\subsection{First Stage Estimates}
The shift-share instrument has a strong first stage for universities' yearly state funding.
\autoref{tab:firststage-reg} presents results of the first-stage regression among IPEDS data, separately with and without a control for tuition revenue per student, plus institution and year fixed effects.
I use the same empirical strategy in the analysis of Illinois faculty, so  \autoref{tab:firststage-illinois} presents first-stage estimates among the IBHED data (which are very similar to those calculated in IPEDS data).
\autoref{tab:firststage-reg} Panel A shows estimates of a funding shock of \$1 per student in the state on state funding per student at the university, and Panel B the \% increase effects of a 1\% increase in the shock per student on state funding per student.
Column (1) shows that a funding shock of \$10 (per student in the entire state) is associated with \$11.76 less state funding per student at the university, with the corresponding -10\% funding shock leads to -9.77\%
less funding (column 1, panel B), with similar estimates with and without including fixed effects.
Column (2) shows estimates of the first-stage without including fixed effects, and gives less precise estimates for the funding shock, likely thanks to systematic differences in universities unaccounted for without fixed effects.
Columns (3) and (4) include the tuition revenue control (explained in \autoref{foot:control}) to exhibit estimates with the inclusion of this possible collider (or bad control).
Column (3) shows similar estimates to column (1) in both Panels A and B thanks to inclusion of fixed effects, so that the uncertainty in including tuition revenue as a possible bad control for the level of state funding does not matter thanks to the inclusion of fixed effects.
Column (1) represents the estimates for \autoref{eqn:firststage} with fixed effects, omitting the tuition revenue control, and is the preferred form that I proceed with.

The above shows that the first stage is strong among both IPEDS and IBHED data, and the argument in \autoref{sec:empirics} for exogenity comes from independence of the state funding mechanism to any indivdiaul public university campus.
\autoref{tab:firststage-balance} provides a balance test, showing that the shift-share instrument is correlated with the professor outcomes, but not with other sources of university funding.
These results, together with the case for exogeneity, show the funding shock instrument
strongly predicts university revenues in the first-stage estimation.

\subsection{National Level, IPEDS}
Changes in state funding have clear effects on the composition of faculty public universities employ.
\autoref{tab:facultycount-shock-reg} presents OLS and IV estimates for the effect of a change in state funding on the number of faculty at the university, individually by position (and for the total faculty), in both count of professors and in percent terms for faculty per student ratio.
An extra \$1,000 in state funding per student leads to the average university employing 6 fewer lecturers, while effects for the other positions are not discernible from zero in absolute measure (\autoref{tab:facultycount-shock-reg}, Panel A).
However, Panel A \autoref{tab:facultycount-shock-reg} measures professor count in percentage terms for professors per student.
A 10\% increase in state funding lead to a 4.4\% decrease in the number of lecturers per student, an increase of 14\% for both assistant and full professors, leading to a 0.6\% increase in the total number of faculty per student.
% In the state of Illinois, using the count of professors represented in the IBHED databse, state funding is correlated with the number of faculty, but lacks precision to identify the exact effects given the 144 university-year observations in the panel for 2010--2021.

Faculty composition is not only impacted in the same year as a state funding shock, but also for a number years following.
\autoref{fig:count-lp} shows local projection estimates, as a series of impulse responses, where the estimate is the elasticity for professor count per student in year $t+k$ with respect to state funding in year $t$ for follow-on years $k = 0, \hdots, 10$.
A 10\% decrease in state funding increases count of lecturers per student by 4\% for the first two years after the initial shock.
While count of assistant professors decreases 10 to 18.5\% three years later;
count of tenured professors decreases 9 to 6\% three years later.
Together, the total count of professors per student decreases around 1\% for three years after the initial funding shock, and the effect is not distinguishable from zero after this point.

\subsection{Illinois Faculty, IBHED}
This section refers to results using data on every professor at a public university in the state of Illinois.
Incumbent professors are relatively unaffected by changes in the state funding for their university in the years following their hire.
\autoref{tab:faculty-shock-illinois-rolling} show estimates of a 1\% change in state funding per student on faculty salaries, rate of exiting the Illinois public university system, and rate of promotion.
The estimates are not discernible from zero, so that faculty who are already at the university (incumbent faculty) seem to have no effects of state funding cuts passed on to them.
Additionally, local projection estimates show that professors' salaries are also not affected on average in yeard following a funding shock  (\autoref{fig:salaries-illinois-lp-rolling}).
Lecturers on the other hand, see a 0.75\% increase in salary in the second and third year after a 10\% increase in state funding (\autoref{fig:salaries-lecturer-illinois-lp-rolling}), reflecting how lecturers are again more affected by changes in public university funding.
Similarly, promotion rate and exit rate among incumbent faculty  (from the Illinois public university system) are unaffected in the years after an initial funding shock (\autoref{fig:promoted-illinois-lp-rolling}, \ref{fig:exit-illinois-lp-rolling}).

These findings show that public universities increase (decrease) their count of tenure-track and tenured professors per student in years when revenues are more (less) plentiful.
In the same vein, when funding increased, count of tenure-track and tenured professors per student increased --- and the opposite in years of funding cuts.
These findings are in line with the observation that universities froze hiring for a couple of years in response to 2008 negative budget shocks, particularly for tenure-track positions \citep{turner2014impact}, and similarly from shocks to university endowments \citep{brown2014endowment}.
For lecturers, we see a negative effect which lines up with two trends noted in \autoref{sec:trends}: funding for public universities (per student) decreased while relative usage of lecturers increased substantially.

\subsection{Reduced Hiring of New Professors}
Public universities substituted professors for more lecturers thanks to stagnating state funding, but there are little discernible effects on incumbent professors or lecturers.
This implies that the substitution arose by impacting the hiring of new professors to replace those retiring or leaving academia.
% \textbf{Expand this paragraph}.
I take recent data on the total faculty hired at US public universities over 2011--2021 (provided by \citealt{wapman2022quantifying,wapman2022zenodo}), to show that universities with lower state funding (per student) also hired fewer professors in the same time period --- shown in \autoref{fig:hiring-correlation}.
The shift-share IV model shows that a $10$\% cut in per student state funding leads to $13$\% fewer faculty hires per student across the decade 2011--2021.\footnote{
    This limited sample of data are a cross-section for count of professor hires 2011--2021, so that local projection estimates are not possible here.
    See \autoref{sec:appendix-hiring} for further details.
}
These results provide suggestive evidence that funding cuts caused faculty substitution by disrupting hiring of new professors at public universities.
This provides a coherent mechanism for how incumbent faculty (those who are already hired at the university) were not meaningfully impacted by funding cuts, yet their wider composition nonetheless changed when their universities experienced funding cuts. 

%\subsection{Robustness checks}
% Describe robustness checks here.
