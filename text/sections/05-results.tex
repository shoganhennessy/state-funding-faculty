%%%%%%%%%%%%%%%%%%%%%%%%%%%%%%%%%%%%%%%%%
%% Results section
\section{Results}
\label{sec:results}

\subsection{First Stage Estimates}
The shift-share instrument has a strong first stage for universities' yearly state funding.
\autoref{tab:firststage-reg} presents results of the first-stage regression among IPEDS data, separately with and without a control for tuition revenue per student, plus institution and year fixed effects.
\autoref{tab:firststage-reg} Panel A shows estimates of a state funding shift-share of \$1 per student on state funding per student at the university, and Panel B the effects of a 1\% increase in the state funding shift-share on state funding per student.
Column (1) shows that a state funding shift-share of \$10 per student is associated with \$11.76 less state funding per student at the university; -10\% state funding shift-share leads to -9.77\% state funding per student at the average university.
Estimates are presented for specifications with and without fixed effects.
Column (2) shows estimates of the first-stage without including fixed effects, and gives less precise estimates for the funding shift-share --- this may be due to systematic differences in universities, which are not accounted for without fixed effects.
Columns (3) and (4) include the tuition revenue as a control (explained in \autoref{foot:control}) to exhibit estimates with or without this control variable.
Column (3) shows similar estimates to column (1) in both Panels A and B thanks to inclusion of fixed effects, so that 
whether tuition revenue is a bad control does not affect empirical estimates.
Column (1) represents the estimates for \autoref{eqn:firststage} with fixed effects (not including tuition revenue as a control variable), and is the preferred form that I proceed with.
I use the same empirical strategy in the analysis of Illinois faculty, and \autoref{tab:firststage-illinois} presents first-stage estimates among the IBHED data; results are similar to those calculated in IPEDS data.
These results, show the funding shift-share instrument strongly predicts university revenues in the first-stage estimation.

The first stage is strong in both IPEDS and IBHED data, and the argument for exogeneity comes from independence of the state funding mechanism to any individual public university campus (as in \autoref{sec:empirics}).
The IV approach also assumes that shifts in state-wide funding for higher education (the instrument) affect outcomes only via affecting state funding for the university (exclusion restriction).
This assumption is fundamentally untestable, but is bolstered by the lack of relationship with university characteristics.
\autoref{tab:summary-quantiles} shows the mean of various university variables, across the instrument distribution.
The university--years in the lowest 20\% of the instrument distribution have more research spending than those in the next 20\%; this is to be expected, as university spending is heavily related to state--wide changes in funding.
On the other hand, the measures of selectivity (acceptance rate, 6--year graduation rate) are not substantially different across the distribution of the instrument.
This allays concerns that state-wide shifts in state funding are targetted more towards less selective universities (a possible confounder), and supports use of the shift-share IV empirical design.

Do state funding cuts meaningfully affect universities?
While the shift-share performs well in predicting state funding (i.e., a strong instrument), it could be the case that universities are not affected.
For expample, universities can respond to funding cuts by charging higher tuition fees; it could be the case that this entirely off-sets funding cuts, leading to no effects on the university.
I estimate the IV model on IPEDS data for areas of university spending  (\autoref{tab:expenditures-shock-reg}).
This exercise gauges the effects of state funding on areas of university spending to guage whether the first stage is economically meaningful, and to see which areas of higher education operations are affected by state funding cuts.
Following a state funding cut, universities spend less on instruction, student services, operations, and grant aid.
Additionally, they charge higher tuition fees, though this rise is at most a quarter of the state funding cut, so does not anywhere near off-set state funding cuts.
These results are in line with previous work showing state funding cuts (or limits) reduce student-focused spending \citep{NBERw23736}, financial aid \citep{miller2022making}, and lead to increased tuition fees \citep{bound2019public}.

\subsection{University Level, IPEDS}
\label{sec:results-ipeds}
Funding cuts lead to measurable changes in the number of faculty at US public universities.
\autoref{tab:facultycount-shock-reg} presents OLS and IV estimates for the effect of a change in state funding on the number of faculty at the university, for each faculty position (lecturer, assistant and full professors, total).
The results are presented in two different units: Panel A shows how a \$1 increase in state funding changed the number of faculty per students, while Panel B shows how a 1\% increase in state funding led to a percent change in the number of faculty per students.

An extra \$1,000 in state funding per student leads to the average university employing 6 fewer lecturers, while effects for the other positions are not discernible from zero (\autoref{tab:facultycount-shock-reg}, Panel A).
A 10\% state funding cut lead to a 4.4\% increase in the number of lecturers per student, and a decrease of 1.35\% and 1.37\% for assistant and full professors, respectively.
A state funding cut of 10\% leads to a 0.65\% decrease in the total number of faculty per student (\autoref{tab:facultycount-shock-reg}, Panel B).
These are large substitution effects, and explain roughly 40\% of the trend in subsitutional towards lecturers over this same time period.
The average public university has roughly 36\% less state funding per student in 2021 than they did in 1990, while employing over double the number of lecturers and 20\% fewer professors (relative to enrolment, \autoref{fig:mean-funding-fte}, \ref{fig:fte-perprof}).

Once the funding cut has passed, universities do not re-adjust to hiring more professors to make up for last year's substitution towards lecturers.
\autoref{fig:count-lp} shows local projection estimates, where the estimate is the elasticity for professor count per student in year $t+k$ with respect to state funding in year $t$.
These elasticity estimates are shown in \autoref{tab:facultycount-shock-reg} Panel B, where the outcome is faculty count per students in later years.
State funding cuts affect the number of lecturers and professors at universities up to 9 years after the initial funding cut.
A 10\% state funding cut leads to 4.2\% (2.9--5.5 in a 95 percent confidence interval) more lecturers per student the following year, and around 3.1\% (0.6--5.5 confidence interval) more lecturers nine years later.
A 10\% state funding cut leads to 1.5\% (0.6--2.2 in a 95 percent confidence interval) fewer assistant professors per student the following year, and around 1.1\% (0.2--2.0 confidence interval) fewer nine years later;
the same cut leads to 1.2\% (0.6--1.8 in a 95 percent confidence interval) fewer full professors per student the following year, and around 0.9\% (0.1--1.7 confidence interval) fewer nine years later.

These effects are a substitution of professors for lecturers, as the total number of faculty per students are relatively unaffected in the follow-on years, while numbers of lecturers increase and professors decreases.
This agrees with a model for universities departments having fixed teaching requirements making human-resources decisions under constrained resources (e.g., \citealt{abe2015implications}), lending credibility to this type of model for understanding recent human resources decisions of public universities.
Furthermore, if a university must teach more students, but cannot afford to hire a full-time professor who teaches 2--3 classes a year, then it makes sense that they hire 2--3 additional lecturers on part-time contracts as a replacement.
This provides simple intuition for why the lecturers' elasticity is 2--3 times larger than that for assistant/full professors in \autoref{tab:facultycount-shock-reg}.
While no previous research causally connects funding cuts to substituting professors for lecturers, the phenomenon has been alluded to in popular media \citep{wiu2016} and university rankings (where more lecturers hurts a university's ranking, \citealt{usnews2023}); these results provides real evidence for these claims.

\subsection{Illinois Faculty, IBHED}
Incumbent professors at Illinois universities are relatively unaffected by state funding cuts for their university.
\autoref{tab:faculty-shock-illinois-rolling} show estimates of a 10\% change in state funding per student on faculty salaries, rate of exiting the Illinois public university system, and rate of promotion.
The estimates are not discernible from zero in any specification, including the shift-share IV estimates, so that faculty seem to have little or no state funding cuts passed on to them for these outcomes.

It could be the case that effects on faculty do not materialise until multiple years after a funding cut, though the data do not support this possibility.
Local projection estimates show that professors' salaries, including salaries, are not significantly affected in years following funding cuts (\autoref{fig:salaries-illinois-lp-rolling}) --- they do not receive lower rates of pay raises, or delayed salary cuts, in response to state funding cuts.
Similarly, promotion rate and rate of exit from the Illinois public university system are also unaffected in later years (\autoref{fig:promoted-illinois-lp-rolling}, \ref{fig:exit-illinois-lp-rolling}).
Lecturers receive a 0.75\% increase in salary in the second and third year after a 10\% increase in state funding (\autoref{fig:salaries-lecturer-illinois-lp-rolling}), though they are they only faculty who saw any effect of state funding cuts in terms of salaries.

These results show that faculty --- both lecturers and professors --- are relatively unaffected by state funding cuts, in the Illinois public university system.
Illinois is a state with a large public university system, relative to student enrolment, with trends in state funding for education similar to national trends, and with both selective and non-selective universities (\autoref{sec:trends-illinois}).
If faculty are unaffected in Illinois, then it is unlikely that faculty in the rest of the nation are affected on average.
Note, however, that this analysis studies faculty who are already faculty.
The IBHED data do not measure the entire pool of academics applying for faculty jobs at Illinois universities, but only those who applied to, were accepted, and then started a job as a lecturer or professor.
These are incumbent faculty, and it is clear they are unaffected by state funding cuts.
This raises a question: how can funding cuts lead to a large substitution of professors for lecturers, when incumbent faculty are not affected?

\subsection{Impacts on Faculty Hiring}
Turn-over is the primary channel that changes in faculty composition happen, where, where faculty exit their jobs and new faculty are hired to replace those leaving --- or no more are hired to replace those who left.
\autoref{fig:exit-illinois-lp-rolling} shows that Illinois faculty are no more likely to exit the Illinois system (either retire or take a new job elsewhere) following state funding cuts, so there is one remaining channel for the substitution of professors for lecturers: increased hiring of lecturers and reduced hiring of new professors.

To investigate reduced hiring of new professors, I first investigate faculty hiring rates among Illinois universities from the IBHED data.
To do so, I count the number of faculty new in the panel of all Illinois public university faculty for years 2011--2021, where the panel first year is 2010.
I then use this as the outcome in OLS and shift-share IV regressions, where each observation is an Illinois university--year and data for state funding come from IPEDS.\footnote{
    This is the same specification as the IPEDS analysis restricted to Illinois universities.
    The IPEDS data have yearly faculty counts, but not discern between faculty promotions, quits, and hires; the Illinois data can do so thanks to the panel structure of the data.
}
To study whether these effects hold true nation-wide, I use public on the number of professors at all US public universities for the entire decade 2011--2021, provided by \citep{wapman2022quantifying},
and perform the same analysis to see if state funding affected the number of professors hired.
See \autoref{sec:appendix-hiring} for further details on this data source.

\autoref{tab:facultyhires-illinois-reg} shows these estimates for Illinois public universities.
In general, faculty counts are relatively unrelated to state funding among Illinois universities; though the point estimates line up with analysis in IPEDS, the smaller number of observations mean that the precision of estimates are small and indistinguishable from zero, in both raw counts and in log units.
On the other hand, universities with more state funding hire more professors across 2011--2021, as seen in \autoref{fig:hiring-correlation}.
An extra 10\% in state funding is associated with 8.5\% more professors hired per students.
I use these data as part of a shift-share analysis, showing that a 10\% state funding cut per student leads to 13\% fewer professor hires (relative to enrolment), and that these hires are roughly equal for gender of professor hired.

State funding cuts lead to lower professor hiring, and the persistent funding cuts to public universities led to lower professor hiring across the last three decades.
The channels for faculty turn-over were unaffected by state funding cuts in the analysis of Illinois faculty, and rate professor hiring is related to professor hiring in both correlational terms and in a shift-share analysis.
This means that effects on faculty hiring were likely the main mechanism for how state funding cuts led to a substitution of professors for lecturers at public universities.

\subsection{Robustness Checks}
\label{sec:results-robustness}
It is possible that the identification strategy is not robust to some confounding issues or trends.
For example, selective institutions may have received funding cuts for other reasons, or states with lower performing institutions or declining populations strategically cut funding for only less selective universities.
I address concerns for this with three different analyses.

First, I re-estimate the primary results from \autoref{sec:results-ipeds} with additional controls for university selectivity.
Specifically, I estimate \autoref{eqn:secondstage} with controls added for the base share of state funding, average acceptance rate and 6-year completion rate  in 2004--2006, tuition revenue, total enrolment, size of state enrolment, size of university enrolment relative to the rest of the state.
Measures of selectivity are only available in certain years, so I use these values across the entire panel.
This means that institution fixed effects would over-identify the system, so I estimate with state $+$ year fixed effects (not institution $+$ year fixed effects).
In this way, this model also works as an alternative specification test.
Additionally, this analysis uses the enrolment measure of full-time equivalent (FTE), and not reported enrolment as in the original analysis.
The first-stage is strong of the same order as the original analysis (\autoref{tab:firststage-robustness-checks}).
The relationship between state funding, in both raw count terms and log units, are roughly the same as for the main analysis (\autoref{tab:facultycount-robustness-checks}, \ref{tab:facultycount-rawcount-robustness-checks}), despite the multiple specification changes, additional controls, and different measures for enrolment.

Secondly, I return to the primary specification in \autoref{eqn:secondstage}, and estimate effect of state funding cuts on faculty counts separately among universities of different selectivity rates.
This investigates whether results are consistent across different levels of selectivity, and whether less selective universities have unreasonably high estimates for substitution --- as would be the case if selectivity was an unaddressed confounder.
\autoref{tab:facultycount-heterogeneity} collects these results together, separated by the \cite{barrons2009} selectivity rankings.
Barrons rank 1-2 are the most selective (including e.g., UC Berkeley and SUNY Binghamton), 3-4 selective (including e.g., Clemson University and University of Dallas), and rank 5 for the rest of universities.
The first-stage is strong in every group of universities, even among the small sample size of the 12 universities in the most selective group, and the 28 universities in the next selective group.
Faculty substitution rates are not very strong among the most selective universities, but are very similar between selective universities and unranked universities, showing that even selective public universities exhibit the substitution towards lecturers.

The final robustness check I employ is a falsification test, testing whether the effects for public universities would be picked up among a sample of private universities.
Private universities do not rely on state funding in the same that public universities do,\footnote{
    Though private universities do receive some state funding, as some states fund for project at private universities and other smaller scale operating costs.
}
so we would expect state funding to not affect their faculty counts.
If the state funding and faculty counts were related among private universities, this would be evidence that the IV model picks up otherwise concurrent trends in higher education, unrelated to stagnating state funding for public universities.
To impute a base share of reliance on state funding, I assign private universities the average reliance on state funding in the base period of universities in the same state with the same Barrons selectivity ranking, and then compute the shift-share instrument from this value.
\autoref{tab:facultycount-shock-private-robustness} shows that the first-stage is not strong among private universities; despite this, the IV results are still not significant for any faculty count, showing that state funding is not related to private universities' faculty counts.
The OLS columns show that imputed state funding is highly correlated with lecturer and professor counts at private universities.
The IV estimates are not distinguishable from zero, showing that OLS correlations are not causal relationships, and validating the IV approach for inference on universities' faculty counts.
