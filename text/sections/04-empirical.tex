%%%%%%%%%%%%%%%%%%%%%%%%%%%%%%%%%%%%%%%%%
%% Empirical section
\section{Empirical Framework}
\label{sec:empirics}

This paper identifies the effects of changes in state funding on faculty.
However, a university's state funding is not exogenous to state decisions for support of higher education, or exogenous to internal institutional decisions.
Instead, the state government and university administration undertake a complex process of allotting resources across multiple different priorities, including instruction, research, or between departments.
For example, it may be the case that a state government restricts funding for only its lowest performing universities in response to a budget shock, introducing a treatment selection-on-outcome threat to identification. 
Importantly, funding from a university's state government provides opportunity to address such endogeneity.

\subsection{State Funding Shocks}
\label{sec:approp-shocks}

I use a shift-share instrument to address endogeneity concerns for the amount of funding a state allots to each of its public universities.
\cite{NBERw23736,NBERw27885} develop the instrument for public university finances by exploiting a shift-share instrument for changes in state-level funding interacted with university reliance on state funding in a base period.
\begin{align}
    \label{eqn:public-instrument}
    Z_{i,t} &\coloneqq - \left[
    \left( \frac{\text{Total State Funding}_{s(i),t}}{\text{Student Population}_{s(i),t}} \right)
    \sum_{\tau = 0}^{3} \frac 14
    \left( \frac{\text{State Funding}_{i,1990 + \tau}}{\text{Total Revenues}_{i,1990 + \tau}} \right) \right]
\end{align}
The system exploits the fact that institutions who rely on state funding more will be affected by state funding shocks.
$Z_{i,t}$ is the instrument for state funding for institution $i$ in year $t$, interacting the average funding for universities in state $s(i)$ with reliance on state funding relative to total revenues, averaged across the base years 1990--1993.\footnote{
    1990--1993 are defined as data for public university finance data are most comparable (i.e. without many missing values) beginning in 1990.
    \cite{NBERw23736} use the single year 1990 as the base year, though I use the four years to ameliorate missing values in the single year of 1990.
    Results are similar in either specification.
}
$Z_{i,t}$ is constructed as negative to reflect the fact that the long term trend in, and most of the short-run shocks to, state funding for higher education have been negative.\footnote{
    When used in log terms, the instrument is the negative of the logged shock --- i.e., $- \log \left( -Z_{i,t} \right)$ as opposed to log of the negative shock $\log Z_{i,t}$ directly.
}
State funding has been falling, so that the instrument describes shocks to university revenues, mostly in a negative direction.\footnote{
    \label{foot:control}
    \cite{NBERw27885} note the tendency for public universities to respond to state funding cuts by increasing reliance on tuition, where \cite{NBERw23736} specifically instruments for tuition revenues with collected information on legislative tuition price controls.
    It may be argued that tuition revenues are confounder between the causal effect of changes in state funding on a university's total revenues, so that this analysis focuses on state support for higher education (not total revenue), as do \cite{NBERw27885}.
    On the other hand, rises in tuition revenues (per student) may arise as result of tuition hikes thanks declining state support, which would mean controlling for tuition would constitute a bad control.
    Nonetheless, estimates including tuition revenues (per student) as a control in the second stage of the IV estimates produces results of very similar magnitude and direction, and so are omitted.
}
The instrument approach relies on the conditional independence assumption, that the instrument is independently assigned to the universities.
While this assumption is fundamentally untestable, we see that universities are by-in-large smaller (in terms of enrolment, total revenues, and professor count) at the top of the state funding shock distribution (see~\autoref{tab:summary-quantiles}).
Though on a per student basis, there is little difference across the distribution (except in the outcomes under consideration).
The state funding shock instrument is positively associated with the total enrolment and total amount of state funding for each university, in both \$ and $\log$/percentage change terms, while other the other sources of finances fopr the university are not associated with the funding shock.
So that the other sources of finances are not clear confounders for the instrumental variables strategy, as they exhibit balance with respect to the funding shock instrument \citep{pei2019poorly}.
Yearly and individual fixed effects are included in regressions throughout to implicitly condition on mean differences between universities and years, leading to the assumption of conditional independence of the instrument with respect to state funding for each public university.

The state funding shock is an instrument the level of state funding for each university in each year.  
The first-stage is then as follows, including institution and year fixed effects, where $X_{i,t}$ represents the amount of state funding divided by the number of full-time students attending the university.
\begin{equation}
    \label{eqn:firststage}
    X_{i,t} = \eta_i + \zeta_t + \delta Z_{i,t} + \epsilon_{i,t}
\end{equation}
We note the conditions for exogeneity in the instrument (following the discussion presented by \citealt{NBERw27885}).
The instrument is exogenous if state policy decisions for funding of public universities are uncorrelated with unobserved institutional changes of any specific college or university in the state \citep{borusyak2022quasi}.
This assumption is plausible given that the majority of states have multiple (i.e., more than five) public universities, without any single university campus receiving the majority of state funding within any single state.
Secondly, the shift-share identification strategy requires exogeneity in either the base-line share or shift component of the instrument.
In this case, we satisfy the second: universities' institutional-level decisions are not correlated with contemporaneous or upcoming shocks to state funding.\footnote{
    It would be plausible to consider the case that universities make institutional-level decisions in a consistently different manner to those 
    with differing reliance on state funding in 1990, so that exogeneity by the base-line share is not plausible here.
}
Lastly, the shift-share approach assumes that state funding shocks affect faculty outcomes only via affecting university finances.

The shift-share instrument has a strong first stage for universities' yearly state funding.
\autoref{tab:firststage-reg} presents results of the first-stage regression, separately with and without a control for tuition revenue per student, plus institution and year fixed effects.
\autoref{tab:firststage-reg} Panel A shows estimates of a funding shock of \$1 per student in the state on state funding per student at the university, and Panel B the \% increase effects of a 1\% increase in the shock per student on state funding per student.
Column (1) shows that a funding shock of \$10 (per student in the entire state) is associated with \$11.76 less state funding per student at the university, with the corresponding -10\% funding shock leads to -9.77\%
less funding (column 1, panel B), with similar estimates with and without including fixed effects.
Column (2) shows estimates of the first-stage without including fixed effects, and gives less precise estimates for the funding shock, likely thanks to systematic differences in universities unaccounted for without fixed effects.
Columns (3) and (4) include the tuition revenue control (explained in \autoref{foot:control}) to exhibit estimates with the inclusion of this possible collider (or bad control).
Column (3) shows similar estimates to column (1) in both Panels A and B thanks to inclusion of fixed effects, so that the uncertainty in including tuition revenue as a possible bad control for the level of state funding does not matter thanks to the inclusion of fixed effects.
Column (1) represents the estimates for \autoref{eqn:firststage} with fixed effects, omitting the tuition revenue control, and is the preferred form that I proceed with.

These results, together with the case for exogeneity, show the funding shock instrument strongly predicts university revenues in the first-stage estimation.


\subsection{Instrumental Variables Model}
\label{sec:iv-model-uni}

I use the instrument defined in \autoref{eqn:public-instrument} to overcome the endogeneity concerns for state funding to each public university, so the primary empirical model is an instrumental variables model.
\autoref{eqn:firststage} is the first stage for exogenous variation in the state funding for university $i$ in year $t$, and \autoref{eqn:secondstage} the second stage for the effect of state funding on faculty outcomes.\footnote{
    It is important to note the treatment effect isolated here; the instrumental variables approach identifies the local average treatment effect, one weighted to level of exposure when treatment is continuous as is the case here.
    So we interpret this treatment effect as a weighted average of effects on faculty at public universities to state funding changes, specific to state funding shocks, among the complier group --- i.e., among universities who respond to funding shocks and would not have made faculty-outcome changes absent the funding shock.
    Also, we assume that no universities state funding increased in response to the negative state funding shock (monotonicity).
}
\begin{equation}
    \label{eqn:secondstage}
    Y_{i,t} = \alpha_i + \gamma_t + \beta \widehat X_{i,t} + \varepsilon_{i,t}
\end{equation}
I estimate the system by two stage least squares, including institution and year fixed effects.\footnote{
    Note that dividing by student count also implicitly controls for the size of the university, so that this model implicitly accounts for yearly variation in professor count and university revenues arising from a university's size of growth/decline.
}
$Y_{i,t}$ represents faculty outcomes for university $i$ in year $t$, $\alpha_i, \gamma_t$ university and year fixed effects, and $\widehat X_{i,t}$ state funding for university $i$ estimated in first stage \eqref{eqn:firststage}.

Additionally, I investigate the effect of changes in state funding among incumbent professors.
Incumbent professors are faculty who are already employed at the university; state funding changes may affect faculty who are already at the university (beyond affecting whether they get hired), so I base the instrument in the year that the professor was hired, and include fixed effects for the hiring year.\footnote{
    This formulation follows that presented by \cite{NBERw27885}, where individual student outcomes are analysed via variation in state funding after their freshman-year.
    This contrasts with \autoref{sec:iv-model-uni} and \cite{NBERw23736}, where the unit of analysis is the university-year, where base year 1990 is more appropriate.
    See \autoref{sec:iv-model-indiv} for the instrument and second-stage specification.
}
The instrument exogeneity and exclusion follows the same argument as above, where the Illinois legislature did not target any single campus for funding cuts and no single Illinois campus takes the majority of state funding.

It is not a priori clear which units are appropriate for this analysis;
does it make sense to consider state funding in purely dollar amounts per student, or in percent change rate?
The funding shock instrument is a strong predictor for the average university's level of state funding in either unit; a funding shock of \$1,000 per student in the entire state leads to around \$1,176 per student at the university, while a funding shock of $-10$\% leads to around 9.77\% less state funding per year.
Yet, the level of state funding (and the outcome variables) vary greatly between states for the unit of analysis.
For example, the average Illinois public university receives \$10,709 in state funding per student in 1990 and  \$6,713 in 2021 (a fall of over 30\%), while California went from \$19,224 per student to \$12,915 in the same time span (a fall of 37\%).
While most states are not exactly the same to California and Illinois, this is example is telling for the phenomenon of stagnating state funding:
states vary in how much the funding higher education in 1990, but most have experienced a decline on order of 30\%, so that the stagnation in state funding has been a percent change trend over this time period.
As such, I include regression specifications where the explanatory and outcome variables are $\log$ transformed, and refer to these when stating results in percent change terms.

\subsection{Effects in Years After the Funding Shock}
\label{sec:empirical-substituion}

The effects of a change in the universities funding may not be immediate, and faculty in may be effected multiple years after a funding shock.
Yet, the funding shock instrument is significantly auto-correlated year-on-year; a large state-level funding shock in year $t$ is also likely to experience a large funding shock in year $t-1$.
Similarly, funding shocks to higher education in year are highly correlated with state funding per student in the 5 years before and after the original shock in IPEDS data (see \autoref{fig:lag-firststage}).
Thus, a linear model correlated state funding in year $t$ with outcomes in year $t$ will suffer from time-series confounding.

I employ a local projections approach to model how faculty outcomes in years $t+k$ are affected by state funding in year $t$, for future years $k > 0$ \citep{jorda2005}.
The local projections method is an empirical model used to estimate dynamic treatment effects when the treatment is not binary, so that time-series confounding (i.e., treatment in time $t$ is correlated with treatment in time $t-1$) is present, even for instrumental variable models \citep{montiel2021local}.
This method estimates the effect of treatment $X_{i,t}$ on outcome $Y_{i, t+ k}$ in follow-on years $k > 0$, while accounting for the auto-correlation between the other years.\footnote{
    The funding shock has a persistent effect on state funding, multiple years after the initial funding shock, so that the instrument is similarly strong for the local projections method (see \autoref{fig:firststage-lp}).
}
Additionally, the approach accomodates use of an instrument for the regressor \citep{olea2021inference}, so that I use the shift-share instument as part of the local projections estimation.
I present estimates in graphical form, where the x-axis represents the year relative to the funding shock (i.e., $x = 0$ represents the year of the state funding shock) and the y-axis represents the estimated effect of the funding shock on that year's outcome.\footnote{
    See \autoref{fig:count-lp} for the graphical format, and the corresponding note for further explanation.
}
