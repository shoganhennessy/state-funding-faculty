Public universities educate the majority of higher education students in the US, yet have experienced a stagnation in state funding over the last three decades .
At the same time, public university employment of lecturers dramatically increased at public universities, while enrolment rose dramatically.
Previous research has examined the effects of the decline in state funding on student outcomes \citep{NBERw23736,NBERw27885}, but there is little evidence of the effects of this funding decline on faculty.
I show that four-year, degree-granting public universities have experienced a systematic decline in state funding per student, and use a shift-share instrument to show that this fall led to a substitution away from tenure-track/tenured professors towards contingent lecturers.
Analysis of all public university faculty in the state of Illinois shows that incumbent faculty were relatively unaffected, in terms of wages, promotion and quit rates, implying that changes in faculty composition arose by disrupting public universities' ability to hire or replace their tenure-track/tenured faculty. 
Private universities were not exposed to similar financial constraints during this same time period, and do not exhibit the substitution away from tenure-track/tenured professors, so that the stagnating state funding for public universities has implications for the wider structure of US higher education and research.

There are various margins through which funding cuts can impact a university's faculty.
A university's level of resources affect the ability that departments can hire new professors, retain their current faculty, or provide retirement incentives, so it is natural to expect that funding cuts may affect the composition of professors at a university.
Similarly, the wages a university pays their faculty is affected by funding for the university, including the starting salary it offers to new professors and the amount of yearly raises (if any) for incumbent faculty.
So that it is natural to expect that systematic funding cuts will lead to changes in faculty composition, and possibly affect incumbent professors, too.
Yet, it is a priori unclear whether/how much each of these margins adjusts to funding cuts, as these decision depend on how the university administrations view the productivity of different inputs, plus the degree of wage and employment stickiness over time.

US universities are widely considered the highest performing in the world, yet there are consequential differences between its universities that operate in the private sector and those established by state governments.
Public universities are subject to numerous state-level administrative laws, and rely on their state governments for funding: an average public university received around \$11,600 of state funding per student in 1990, and only \$8,300 in 2021.
This fall is driven by a stagnation in the absolute level of state funding for higher education in most states, combined with a large increase of 46\% in student enrolment at public universities over the same time period.
Similarly, the number of lecturers per student over doubled, while the number of tenure-track/tenured professors per students fell by $-23$\% from 1990 to 2021.

I use data from the \cite{ipeds} to study the impact of changes in state funding for US public universities 1990--2017.
I combine these data with panel data for every faculty member at Illinois public universities for 2010--2021 \citep{ibhed}, to measure how incumbent faculty (i.e., those already employed at the university) are affected by changes in state funding.
These data allow me to answer the following research question: if a US public university receives an extra \$1,000 for each of their students (or a 10\% increase), do they change the composition of faculty they employ?
If so, which positions do universities substitute away from, or towards, and do these funding cuts affect the faculty members themselves?

I employ a shift-share instrument to identify changes in state funding for higher education, which exploits how universities differ in their financial reliance on state funding, combined with yearly shocks to the amount an entire state funds higher education --- following \cite{NBERw23736,NBERw27885}.
State governments decide how to fund higher education by a complicated process which can be influenced by local economic conditions, changing state priorities, or even lobbying from the state universities themselves.
As such, state funding is not randomly assigned to universities and so it is necessary to employ an identification strategy, such as the shift-share instrument, to account for this.
The shift-share instrument interacts how much a university relies on state funding as a percent of revenues in a base period with the entire state's total funding per higher education student for each year, exploiting changes in funding and how much each university historically relied on state funding for higher education.
Additionally, I estimate effect of state funding on faculty multiple years after the initial funding cuts, using the local projections method thanks to the presence of time-series confounding between the funding shock and later years' level of state funding.\footnote{
    To the best of my knowledge, this is also the first paper to use the local projections method for a shift-share instrument.
    %Previous work ignored the potential time-series confoduning, and estimated linear instrumental variables.
}

Falls in state funding affect faculty composition at public universities, which substitute away from tenured and tenure-track professors towards contingent lecturers.
A funding cut of \$1,000 per student leads to an average university employing 6 more lecturers;
in percentage terms, a funding cut of 10\% per student leads to a fall of 1.4\% in the number of assistant professors per student at a university, a fall of 1.2\% for tenured professors, and an increase in 4.4\% the number of lecturers per student.
Local projection estimates show that these effects linger for three to four years after the initial funding shock, showing that the effect is not isolated to the year of the funding shock. 
Over the same time period, state funding per student fell by around 35\%, count of (tenure-tack/tenured) professors per student fell by 9\%, and lecturers per student increased by 99\%, so that these results show that falls in state funding explain around 53\% of the fall in professors per student
% i.e. -(35 * 0.137) / -9
and 15.5\% of the rise in lecturers per student.
% ie., (35 * 0.437) / 99
Additionally, I use individual level data on all Illinois public university professors in 2010-2021 to investigate whether the shocks to state funding affected individual professors.
Incumbent professors were not meaningfully affected, in terms of total salary, promotion rate, or rate of leaving the Illinois public university system.
Yet the hiring rate for new professors at public universities was negatively impacted by the falls in state funding (and by the funding shocks).
This implies that faculty composition change arose by limiting the hiring of new professors; public universities increasingly hired contingent lecturers, and increasingly did not replace their retiring (or leaving) professors.

Mine is the first paper to provide causal evidence on the impact of state funding on faculty composition, and wages.
The closest related research on higher education funding has examined how university spending affected graduation rates and levels of student debt \citep{NBERw23736,NBERw27885}, and university finances \citep{miller2022making,bound2019public,brown2014endowment}.
For faculty outcomes, multiple theoretically model university decision-making and faculty hiring (see e.g., \citealt{abe2015implications,johnson2009jep,NBERc13879}), while others measured how faculty at universities were affected by endowment shocks and the 2008 recession \citep{brown2014endowment,turner2014impact}.
These papers measure effects of funding and changes in university revenues, but do not measure effects on faculty, or other outcomes related to university instruction.

My results provide suggestive evidence on mechanisms for the previous research connecting state funding cuts for higher education and worsening student outcomes.
Substitution towards lecturers is likely a cost-cutting measure for funding constrained universities, though relying on these faculty --- who are often over-worked and not granted long-term employment protections --- may lead to worse student outcomes \citep{ehrenberg2005tenured,zhu2021limited,jaeger2011examining}.
Lastly, my results provide evidence that the long term trends in higher education funding are causally related, contributing to the literature in the long run trends in faculty outcomes \citep{ehrenberg2003studying} and trends in US higher education and funding \citep{hoxby2009changing}.

This paper proceeds as follows.
\autoref{sec:data} describes the data for university finances and faculty in Illinois, and trends in public university funding for the last three decades.
\autoref{sec:conceptual} gives the conceptual framework for how state funding may affect faculty.
\autoref{sec:empirics} draws the empirical framework for isolating the causal effects of state funding on faculty composition and individual faculty, and \autoref{sec:results} presents the empirical results.
\autoref{sec:discussion} discusses the context and implications for the findings.
\autoref{sec:conclusion} concludes.
