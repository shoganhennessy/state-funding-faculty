%%%%%%%%%%%%%%%%%%%%%%%%%%%%%%%%%%%%%%%%%
%% Introduction section
\section{Introduction}
\label{sec:intro}
States fund their public universities with roughly the same amount today as they did in the 1990s, despite enrolment rising by over 50\%.
At the same time, public universities have systematically substituted away from professors towards lecturers, who often work on contingent and lower-salaried contracts.
Previous research shows state funding cuts led to worsening student outcomes \citep{NBERw23736,NBERw27885}, but there is little evidence on whether these funding cuts impacted faculty.
In this paper I show that falling state funding caused a change in the composition of faculty at US public universities, substituting professors for more lecturers.

Per-student state funding for higher education has stagnated the last thirty years.
The average public university received around \$11,600 of state funding per-student in 1990 and \$8,300 in 2021, while the number of lecturers per-student increased by 109\% and professors per-student decreased by $23$\% over 1990--2021.
I use a shift-share instrumental variables (IV) approach to estimate how faculty are affected by these funding cuts.
Universities employ $4.4$\% more lecturers per-student following a $10$\% funding cut, and correspondingly rely less on tenure-track faculty by employing $1.4$\% fewer tenure-track professors and $1.2$\% fewer tenured professors per-student.
In a secondary analysis of all public university faculty in the state of Illinois, I show that incumbent faculty were unaffected, in terms of wages, promotion and quit rates.
This implies that the faculty substitution arose by disrupting hiring of new tenure-track/tenured professors, which is supported supplementary data on professor hiring at US public universities over 2010--2021.
This paper shows the institutional effects of stagnating state funding for higher education, and offers one possible mechanism for how funding cuts affected student outcomes.

I use two different analyses to explore the causal effects of state funding on faculty composition and salaries.
My first analysis uses data from the \citet[IPEDS]{ipeds} to study the impact of changes in state funding for all US public universities 1990--2021.
My second analysis uses panel data for every faculty member at Illinois public universities for 2010--2021 \citet[IBHED]{ibhed}, to measure how incumbent faculty (i.e., those already employed at the university) are affected by changes in state funding.
These data allow me to answer the following research question: if a US public university has its funding cut by \$1,000 per-student (or a 10\% cut), do they change how many faculty members they employ?
If so, which positions do universities substitute towards, and do these funding cuts affect the faculty themselves?

It is a priori unclear how faculty may be affected, as there are various margins that universities may to respond to funding cuts.
On one hand, faculty may be negatively affected by universities cutting salaries or laying-off incumbent faculty.
This happened in 2010 when the University of Illinois System furloughed university staff following budget cuts after the Great Recession \citep{furlough2010}.
Similarly, faculty hiring committees are often cancelled or disrupted in response to funding cuts, meaning there are fewer faculty entering a department to replace those leaving.
On the other hand, lecturers could be hired to fill in teaching duties, on shorter and cheaper contracts with fewer employment protections.
But it is not certain that universities would hire more lecturers in the face of budget cuts.
Public university administrators, motivated by increased competition \citep{hoxby2009changing,urquiola2020markets}, may notice that their rankings are negatively affected by falling professor-lecturer ratios, and so limit departments from hiring more lecturers to replace leaving/retiring professors.
Administrators could even start courting private funding or increase tuition prices to fill funding gaps \citep{bound2019public}, possibly leaving faculty unaffected by state funding cuts.
These are some ways that the long-run stagnation in state funding would not guarantee a substitution towards lecturers at US public universities.

State funding for higher education is not decided randomly;
universities that rely more on state funding may also be institutions that find it difficult to attract new faculty thanks to location, or other factors.
This means that correlations between state funding and faculty outcomes do not necessarily represent causal relationships.
As such, it would be na\"ive to think of state higher education funding as independent of public university faculty, so I employ a shift-share IV approach to identify causal effects of state funding cuts.
The shift-share instrument identifies exogenous changes in state funding by exploiting differences in universities' reliance on state funding, interacted with state-wide shifts in higher education funding --- following the methods of \cite{NBERw23736,NBERw27885}.
Yearly state government budgets are decided without targetting individual universities for funding cuts, so more reliant universities will be more affected by state-wide funding cuts, so the shift-share approach estimates causal effects among more reliant universities.
Additionally, I estimate effect of state funding on faculty multiple years after the initial funding cuts, using the local projections method.\footnote{
    To the best of my knowledge, this is also the first paper to use the local projections method for a shift-share instrument, and to acknowledge the time-series confounding for shift-share IV models when the outcome is in future time periods.
}
This approach allows me to test whether the effects of funding cuts are persistent or universities recover quickly e.g., by quickly resuming faculty hiring in years after funding cuts.

In my university-level analysis (IPEDS), I find that state funding cuts cause public universities to substitute away from tenured and tenure-track professors towards lecturers.
A funding cut of 10\% per-student leads to a fall of 1.4\% in the number of assistant professors per-student at a university, a fall of 1.2\% for tenured professors, and an increase in 4.4\% the number of lecturers per-student.
Local projection estimates show that these effects linger for up to ten years after the initial funding cut, showing that effects are persistent and not transitory.
Over the same time period, state funding per-student fell by around 35\%, professors per-student fell by 9\%, and lecturers per-student increased by 99\%.
These results show that falls in state funding explain around 53\% of the fall in professors per-student,
% i.e. -(35 * 0.137) / -9
and 15.5\% of the rise in lecturers per-student.
% ie., (35 * 0.437) / 99
In my analysis of all Illinois faculty (IBHED), I find that incumbent professors were not meaningfully affected, in terms of total salary, promotion rate, or rate of leaving the Illinois public university system.
This implies that faculty substitution arose by limiting the hiring of new professors, and is supported by suggestive evidence on professor hiring counts.

I provide the first causal evidence on the impact of state funding on faculty composition and wages.
In a contemporaneous working paper, \cite{hinrichs2022state} use the same national-level variation in university funding to analyse effects on faculty and wages, coming to different conclusions.\footnote{    
    \cite{hinrichs2022state} considers a shorter time-period than this paper (2010--2021), uses only national variation in IPEDS data, relies on IPEDS data for faculty salaries, and has no individual-level analysis of faculty or their salaries.
    See \autoref{sec:data-desc}, \ref{sec:recession-analysis} for an elaboration on these issues.
}
The closest related research on higher education funding examined how university spending affected graduation rates and levels of student debt \citep{NBERw23736,NBERw27885}, and university finances \citep{miller2022making,bound2019public,webber2017state}.
For faculty outcomes, descriptive work measures how US faculty salaries are correlated with other factors \citep{hilmer2020labor}, and applied theory work models university decision-making and faculty hiring (see e.g., \citealt{abe2015implications,johnson2009jep,NBERc13879}).
Other papers measure university decision--making after endowment shocks \citep{brown2014endowment}, and trends in the academic job market following the 2008 recession \citep{turner2014impact}.
These papers measure changes in state funding and university revenues, but do not measure effects of state funding cuts on faculty composition or salaries, or other outcomes related to university instruction.
My results provide evidence that the long term trends in higher education funding and faculty outcomes are causally related, contributing to the literature on faculty outcomes \citep{ehrenberg2003studying} and trends in US higher education and funding \citep{hoxby2009changing,ehrenberg2012american}.

Another contribution of my paper is to provide evidence on possible mechanisms connecting state funding cuts for higher education and worsening student outcomes.
In 2021, around 65\% of lecturers at US public universities are employed part-time (IPEDS); this arrangement is often arranged as an adjunct contract, where faculty are paid based on how many courses they teach, without employment benefits \citep{aau2014Characteristics}.
While funding cuts force universities to substitute towards lecturers, there is consistent evidence that students are worse off being taught by part-time lecturers than full-time faculty.
Universities that rely on more lecturers have lower student retention and graduation rates \citep{jaeger2011examining,ehrenberg2005tenured}.
Similarly, students at a large public university have consistently worse education outcomes if taught by adjunct lecturers than those taught by full-time faculty, largely thanks to the worse working conditions adjunct lecturers face \citep{zhu2021limited}.
Alternatively, \cite{bettinger2010does,figlio2015tenure} study lecturers at a highly selective universities, finding that adjunct lecturers improve subject retention for undergraduates compared to professors.
This paper studies faculty at public universities, which are by-majority not selective, and where lecturers are more often employed part-time with less institutional support.
Combining the evidence that adjunct lecturers lead to worse student outcomes with the trends in education finances and performance \citep{NBERw23736,NBERw27885}, implies that substituting towards lecturers contributed to deteriorating student outcomes at public universities.

This paper proceeds as follows.
\autoref{sec:data} describes the data for university funding, data on faculty in Illinois, and recent trends in higher education funding.
\autoref{sec:conceptual} gives the conceptual framework for how state funding may affect faculty.
\autoref{sec:empirics} draws the empirical framework for isolating the causal effects of state funding on faculty, with \autoref{sec:results} presenting the empirical results.
\autoref{sec:conclusion} concludes.
