Public universities educate the majority of higher education students in the US, yet have experienced a secular decline in state funding (per student) over the last three decades.
At the same time, both public university student enrolment and employment of contingent lecturers dramatically increased at public universities.
I show that four-year, degree-granting public universities have experienced a systematic fall in state funding per student, and this fall led to a substitution away from tenure-track/tenured professors towards contingent lecturers.
Falls in state funding for public universities, instrumented by state-level shocks to state-wide higher education, led to a fall in the number of tenure-track/tenured professors per student within a university, and an increase in the number of lecturers.
Private universities were not exposed to similar financial constraints during this same time period, and do not exhibit the substitution away from tenure-track/tenured professors, so that the stagnating state support for public universities has implications for the wider structure of US higher education and research.

US universities are widely considered the highest performing in the world, yet there are consequential differences between its universities that operate in the private sector and those established by state governments.
Public universities are subject to numerous state-level administrative laws, and rely on their state governments for funding: an average public university received around \$11,600 of state funding per student in 1990, and only \$8,300 in 2021.
This fall is driven by a stagnation in the absolute level of state funding for higher education in most states, combined with a large increase of 46\% in student enrolment at public universities over the same time period.
Similarly, the number of lecturers per student over doubled, while the number of tenure-track/tenured professors per students fell by $-23$\% from 1990 to 2021.

I employ a shift-share instrument to identify changes in state funding for higher education, which exploits how universities differ in their financial reliance on state funding, combined with yearly shocks to the amount an entire state funds higher education --- following \cite{NBERw23736,NBERw27885}.
State governments decide how to fund higher education, and everything else, by a complicated process which can be influenced by local economic conditions, changing state priorities, or even lobbying from the state universities themselves.
As such, it is necessary to employ an identification strategy, such as the shift-share instrument, to account for endogeneity in universities' state funding.
The shift-share instrument interacts reliance on state funding in a base period with the entire state's total funding per higher education student for each year, to exploit changes in funding and how much each university historically relied on state funding for higher education.

\textbf{TODO: Write section on the data sources.  See Lovenheimn (2020) on how to do it.}

Falls in state funding affects faculty composition at public universities, which substitute away from tenured and tenure-track professors towards contingent lecturers.
A fall of $-10$\% in state funding leads to a fall of $-1.4$\% in the number of assistant professors per student at a university, a fall of $-1.2$\% for tenured professors, and an increase in 4.4\% the number of lecturers per student.
Local projection estimates show that these effects linger for three to four years after the initial funding shock, showing that the effect is not isolated to the immediate year of the first funding shock. 
Over the same time period, state funding fell by around $-$35\%, while the count of professors per student fell by $-$9\%, so that these results show that falls in state funding explain around a third of the substitution away from tenure-track and tenure professors towards contingent lecturers.
Additionally, I use individual level data on all Illinois public university professors in 2010-2021 to investigate whether the shocks to state funding affected individual professors.
Incumbent professors were not meaningfully affected, in terms of total salary, promotion rate, or rate of leaving the Illinois public university system.
Yet the hiring rate for new professors at public universities was negatively impacted by the falls in state funding (and by the funding shocks).
This implies that faculty composition change arose by limiting the hiring of new professors; public universities increasingly hired contingent lecturers, and increasingly did not replace their retiring (or leaving) professors.

A number of studies have shown that stagnating state funding for higher education negatively impacts student outcomes.
\cite{NBERw23736} show that increases in public university spending (via funding shocks) increases enrolment and degree completion among students,\footnote{
    \cite{miller2022making} further analyse the effects of falls in university revenues, by finding that reductions in public universities tuition prices (suggestively motivated by falling state funding) leads to reductions in provided financial aid.
}
that high school students exposed to similar public school funding shocks are less likely to attend college \citep{jackson2021school}, and that these shocks induce public universities to rely more on tuition as a source of revenue \citep{bound2019public}.
\textbf{Faculty outcomes surveyed by \cite{ehrenberg2003studying}.}
\cite{NBERw27885} use the same methodology to show that increases in state funding lead to degree lower completion time and later-life debt for students.\footnote{
    \cite{bound2007cohort} use similar variation in state funding per student to show that lower funding leads to lower completion rates at public universities.
}

Over the last half century, there have been drastic changes to the heirarchy of US higher education, where the top universities have become more selective for student applicants, and the average university less selective \citep{hoxby2009changing}.
Similarly, research faculty at US universities are increasingly concentrated among the graduates of highly selective PhD programmes \citep{wapman2022quantifying}.
Faculty composition is an interesting outcome on its own, but its relationship with higher education finances has so far not been studied.
The closest previous examples study  how university endowments react to negative endowment shocks \citep{brown2014endowment} and the 2008 recession \citep{turner2014impact}, and multiple theoretically model university decision-making and faculty hiring (see e.g., \citealt{abe2015implications,johnson2009jep,NBERc13879}).
%Lastly, instruction costs within individual universities vary substantially across subject areas, primarily thanks to department differences in class size and faculty salaries \citep{hemelt2021math}.
The number of professors per student may be particularly important for quality of instruction and thus educational outcomes;
small class sizes in secondary schools lead to increases in test scores \citep{angrist1999using}, and UK university students perform worse academically in particularly large classes \citep{bandiera2010heterogeneous}.
Though, the economics literature is not decided on whether contingent lecturers lead to better or worse outcomes in higher education.
\textbf{Beef up this part to focus more on negative effects on students.}
Lecturers are more effective teachers than research professors at a highly selective university \citep{bettinger2010does,figlio2015tenure}, but students taught by part-time lecturers have worse student outcomes relative to full-time lecturers or faculty \citep{zhu2021limited,ehrenberg2005tenured,jaeger2011examining}.

\textbf{Tenure research \citep{mcpherson1999tenure,mcpherson1983economics}}

While it is not immediately clear whether a systematic substitution towards lecturers away from professors reduces teaching effectiveness, it is clear that stagnating state funding has negatively impacted student outcomes \citep{NBERw23736,NBERw27885}.
\textbf{REWRITE:} where faculty composition is a possible mechanism for how the stagnating state funding for higher education transmits to students, and has led to worsening education outcomes at public universities for the last thirty years.


\textbf{TO-DO:
    This paper precedes as follows.
}
