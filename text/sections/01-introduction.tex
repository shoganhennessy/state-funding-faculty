\noindent
Public universities educate the majority of higher education students in the US, yet have experienced a secular decline in state funding (per student) the last three decades.
This decline has been shown to lead to worse education and later-life outcomes for students, yet it is not clear what mechanisms drive these effects, or how public universities are affected as institutions.
I show that four-year, degree-granting public universities have experienced falling state funding per student, and this affected their composition of faculty.
Falls in public university revenues from the state, instrumented by state-level funding shocks, lead to a fall in the number of tenure-track or tenured professors per student within a university, and an increase in the number of lecturers.
US private universities were not exposed to such financial constraints during this same time period, and do not exhibit the substitution away from tenure-track and tenured professors, so that the stagnating state support for public universities has implications for the wider structure of higher education instruction and research in the US.

US universities are widely considered the highest performing in the world, yet there are consequential differences between its universities that operate in the private sector and those established by state governments.
Public universities are subject to numerous state-level administrative laws, and rely on their state governments for funding: an average public university received around \$11,600 per enrolled student in 1990, yet only \$8,300 per enrolled student in 2021.
This fall is driven by stagnating funding provided by state governments, while enrolment among public universities rose by 46\% over the same time period.
At the same time, the number of professors per student at private universities stayed relatively stable at 0.045, yet fell from 0.04 to around 0.35 professors per student at public universities.
This change was driven by a fall of over 20\% in the count of associate and full professors per student at public universities.

To identify changes in state funding for higher education, I employ a shift-share instrument for state funding of public universities, which exploits reliance on state funding and yearly shocks to the amount an entire state funds higher education \citep{NBERw23736,NBERw27885}.
State appropriations for public universities are plausibly endogenous to many outcomes: state governments decide on yearly budgets for their higher education sector, and this process can be influenced by local financial conditions, or changing state priorities.
The shift-share instrument interacts reliance on state funding in a base period with the yearly total appropriations per student in the entire state, to exploit both changes in support for higher education and how much each university historically has relied on that support.

Negative changes in state funding affects faculty composition at public universities, away from tenured and tenure-track professors towards non-tenure track lecturer positions.
A reduction of 10\% in state funding, via a shock to state appropriations, leads to a fall in 1.4\% in the number of assistant professors per student within a university, 1.2\% for full (tenured) professors, and an increase in 4.4\% the number of lecturers per student.
Local projection estimates show that these effects linger for three to four years after the initial funding shock.
Over the same time period, state funding fell by around 35\%, while the count of professors per student fell by 9\%, so that these results show that falls in state funding explain around a third of the observed shift away from tenure-track and tenure professors towards contingent faculty.
Additionally, I use individual level data on professors at Illinois public universities over 2010-2021 to investigate whether the shocks to state funding affected individual professors.
Incumbent professors were not meaningfully affected, in terms of total salary, promotion rate, or rate of leaving the Illinois public university system.
Yet, the hiring rate of tenure-track professors declined over this time period for all public universities.
This shows that changes in faculty composition came about by hiring reductions of new professors at public universities.

Absent this analysis, it is not immediately clear why stagnating state support and changes in public university faculty composition would be causally linked.
Over the 21st Century there have been drastic changes to college selectivity, where the top universities have become more selective while the average university less selective \citep{hoxby2009changing}.
While the average public university is becoming less selective, it is possible that their most productive or research-focused faculty more often move to more selective and prestigious (and often private) institutions, so that over time public universities (on average) substitute towards contingent lecturers. 
This is one way in which the trends may be concurrent, but not causal.
This analysis shows that stagnating in state funding for public universities cause the changes in faculty composition, explaining about 40\% of the observed substitution towards contingent lecturers, so that it is possible that other systematic changes in higher education (such as changing selectivity) may explain the rest of the variation.

A number of studies have shown that stagnating state funding for higher education negatively impacts student outcomes.
\cite{NBERw23736} show that increases in public university spending (via funding shocks) increases enrolment and degree completion among students,\footnote{
    \cite{miller2022making} further analyse the effects of falls in university revenues, by finding that reductions in public universities tuition prices (suggestively motivated by falling state funding) leads to reductions in provided financial aid.
}
that high school students exposed to similar public school funding shocks are less likely to attend college \citep{jackson2021school}, and that these shocks induce public universities to rely more on tuition as a source of revenue \citep{bound2019public}.
\cite{NBERw27885} use the same methodology to show that increases in state appropriations lead to degree lower completion time\footnote{
    \cite{bound2007cohort} use similar variation in state funding per student to show that lower funding leads to lower higher education completion rate at public universities, 
} and later-life debt for students.

Faculty are a core component of the US higher education and research-innovation university system, yet their composition within universities, and its relationship with higher education finances has so far not been studied.
The closest previous examples study  how university endowments react to negative endowment shocks \citep{brown2014endowment} and the 2008 recession \citep{turner2014impact}, and multiple theoretically model university decision-making and faculty hiring (see e.g., \citealt{abe2015implications,johnson2009jep,NBERc13879}).
Lastly, instruction costs within individual universities vary substantially across subject areas, primarily thanks to department differences in class size and faculty salaries \citep{hemelt2021math}.

The number of professors, both in absolute terms and relative to number of students, may be particularly important for quality of instruction and thus educational outcomes.
Small class sizes in secondary schools lead to increases in test scores \citep{angrist1999using}, and UK university students perform worse academically in particularly large classes \citep{bandiera2010heterogeneous}.
Regarding faculty composition, the literature is not decided on whether contingent lecturers lead to better or worse education outcomes relative to tenured faculty.
Adjunct professors are shown to better undergraduate teachers at highly selective universities \citep{bettinger2010does,figlio2015tenure}, yet universities who use a greater share of lecturers have lower graduation rates \citep{ehrenberg2005tenured}.
Notably, students taught by part-time adjuncts lecturers have worse student outcomes than full-time professors at a less-selective university \citep{zhu2021limited}, but this difference disappears when individual lecturers gain a full-time teaching position at the university. 
My approach focuses on count of professors per student at universities, and considers faculty count and composition change as possible mechanisms for how stagnating state support for higher education negatively impacts student outcomes \citep{NBERw23736,NBERw27885}.
