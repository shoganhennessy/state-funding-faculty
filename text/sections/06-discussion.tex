%%%%%%%%%%%%%%%%%%%%%%%%%%%%%%%%%%%%%%%%%
%% Discussion section
\section{Discussion}
\label{sec:discussion}

%\textbf{implication: public unis are increasing sheer count of lecturers, in response to enrolment, and not their tenured faculty.
%SO prof per student is where the movemenet is most stark.}
Universities received less state funding, relative to increasing enrolment, and reacted by increasing their reliance on lecturers relative to assistant and full professors in both the short and medium-run; these results imply that lecturers can be considered substitutes for professors in the tenure-system.\footnote{
    Additional results in Appendix \autoref{sec:appendix-substitution} formalises the marginal rate of substitution between lecturers and professors on tenure-track, showing how assistant and full professors were substituted for lecturers while there is no substitution between assistant and full professors.
}
The average public university's state funding fell by 36\% per student, whilethe number of lecturers per student increased by 113\%, and assistant and full professors  fell by 17\% and 23\%, respectively.
Stagnating state funding explains about 40\% of the observed substitution towards lecturers away from tenure-track and tenured professors.

These effects are persistent: public higher education in the US was not exposed to one large funding cut, but systematic funding short-falls since the 1990s.
The results in this paper show that state funding cuts had persistent effects on universities and faculty for multiple years after the initial funding cut, which were compounded by many years of further funding cuts.
These long-run changes resulted in large changes in the composition of faculty, with more lecturers and fewer professors at public universities.

At the same time, incumbent faculty in the state of Illinois are unaffected by the state funding cuts in terms of salary, promotion rate, and rate of leaving their faculty position.
These results imply that composition change arose by lower hiring of tenure-track faculty in the public university system.
The private university system did not grow in any corresponding amount, implying that there are fewer tenure-track openings at US universities.\footnote{
    At the same time as these changes occured, the number of PhD graduates in the US has continued to rise \citep{aau2021survey,wapman2022quantifying}.
}
Securing academic employment on the tenure-track is more selective than ever, so that more PhD graduates will end up teaching as lecturers or leaving academia.

The results here are in line with theories explaining the increased competition and inequality between US universities.
\cite{urquiola2020markets} interprets the success of research in US higher education as a result of free market policies, where increased competition in a free market in the early 1900s led to improvements in university research among US private universities.
The rate of selectivity among US universities has become increasingly polarised, where top (and mostly private) universities with the most resources have become more selective at the undergraduate level \citep{hoxby2009changing}, and increasingly dominate academia at the graduate level \citep{wapman2022quantifying}.
% These changes have not just affected research, they have affected teaching, too.
Public universities are by-in-large not selective, so are caught in a relative decline when their selective and private competitors are more selective, and have research and student outcomes.

While market forces have been effective in the successes of US higher education research, we should worry about the effects on education.
Enrolment in higher education has increased drastically since 1990, and public universities educate more than twice as many undergraduates in 2021 as their private counterparts.
%At the same time returns to higher education have increased drastically \citep{berman1998implications}.
Increasingly these students are being taught by lecturers, and not professors.
Lecturers are often employed on short-term or part-time contracts (adjunct), with limited job stability.
While we should worry about the effect on faculty working conditions, there is credible evidence that relying on adjunct lecturers leads to worse education and student outcomes, relative to full-time lecturers and professors \citep{zhu2021limited}.
As tenure has increasingly becomes a private sector phenomenon, and most US undergraduate attend public universities, we should worry about the long-term effects of stagnating funding for higher education.
