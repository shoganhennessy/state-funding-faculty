%%%%%%%%%%%%%%%%%%%%%%%%%%%%%%%%%%%%%%%%%
%% Section on possible outcomes
\section{Conceptual Framework}
\label{sec:conceptual}

Universities can respond multiple ways to state funding cuts.
\cite{NBERw23736} established that public universities experiencing state funding cuts did not fully off-set by raising tuition, and instead cut spending.
It is not clear how these funding or spending cuts went on to affect faculty --- or whether they affected faculty at all.

Wider changes in the American higher education sector could concurrently explain the trends in state funding and faculty substitution.
College selectivity changed drastically over the 21st century, where the top universities have become more selective while the average university less selective \citep{hoxby2009changing}.
While the average public university is becoming less selective, it is possible that their most productive or research-focused faculty more often move to more selective and prestigious (and often private) institutions, leaving public universities substituting toward lecturers over time.
This is one way in which the trends may be concurrent, but not causally related.
I address this issue by using shift-share IV approach to isolate funding cuts that affected more funding-intensive universities, measuring the rate of substitution between lecturers and professors in response to yearly changes in state-wide funding.

Universities could respond to funding cuts by cutting faculty salaries, leading to more faculty leaving for jobs at other universities.
Multiple universities passed a university-wide pay-cut for their faculty in response to state budget cuts around the 2008 recession, and Cornell University implemented a professor salary cut in 2020 at the onset of the Covid-19 pandemic.\footnote{
    The Cornell University administration expected large fiscal squeezes in mid-2020, so imposed a faculty salary cut, and then returned the amount cut later in 2020 when the financial squeeze did not materialise.}
Funding cuts could also limit the salaries public universities can offer to new faculty hires, leading to fewer or less accomplished faculty accepting offers at public universities.\footnote{
    Faculty often consider their outside options in job decisions, considering offers for employment and/or promotion from multiple universities at the same time --- see \cite{blackaby2005} for an overview of faculty outside options.
    If public universities can only make low salary offers thanks to funding cuts, then they are less likely to land professors with multiple offers.
}
This effect may not be the same across faculty seniority; most universities hire more junior faculty each year, so that yearly hiring disruptions may disrupt the looser market for established full professors more than tighter market for assistant professors or lecturers.
While these factors make it unclear who will be affected most by funding cuts, there is one empirical fact worth noting: lecturers have substantially lower annual salaries than professors.
The average Illinois assistant professor's annual salary is $\$76,897$, tenured professors' is $\$109,283$, while lecturers' is $\$31,449$ (annual salary for 2010--2021 in 2021 USD, see \autoref{tab:illinois-summary}).
If a public university's primary obligation is to teach, and they must fulfil this objective with less funding, then substituting away from tenure-track and tenured professors towards lecturers is rationalisable purely on this basis.

If faculty are affected by state funding cuts, then it is still not clear whether effects are persistent or merely transitory.
Universities employ professors on multiple-year contracts (e.g., the tenure contract), and professors' decisions to retire, move university, or leave academia are multiple-year commitments, so that it may take multiple years for effects to trickle down to faculty.
On the other hand, university departments hire in annual cycles, so that a funding cut in one year may immediately impact faculty hiring in the same year, and thus faculty counts in the next academic year.
% Yet, it could be the case that faculty hiring resumes the very next year, or even hires more to compensate for last year's cancelled hiring cycle.
%This means that state funding cuts could effect faculty in only transitory changes in faculty counts.
As such, this paper investigates the dynamics of state funding cuts and their effects on faculty, by employing local projections methods to measure the persistence of effects for multiple years following a state funding cut.
