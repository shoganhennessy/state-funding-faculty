%%%%%%%%%%%%%%%%%%%%%%%%%%%%%%%%%%%%%%%%%
%% Section on possible outcomes
\section{Conceptual Framework}
\label{sec:conceptual}

While state support for public higher education has stagnated at the same time as education costs rose, there are multiple possible ways that a university can respond.
\cite{NBERw23736} established that corresponding rises in tuition did not offset the falling state support, so that university spending (per student) fell in response to these persistent, negative state funding shocks.
There are multiple ways that these changes in finances may affect the faculty composition at public universities.

Most universities hire a number of new professors every year, either to expand their departments or to replace leaving professors.
The number of professors hired in each year is usually highest among non-tenured adjunct faculty, as these instructors are mostly hired on short-term contingent contracts, so that they are less costly to hire (or to not reinstate) in response to yearly changes in teaching needs.
Secondly, tenure-track assistant professors are hired to replace leaving assistant professors, and are most often hired with a four to six year contract and agreement for tenure consideration.
Lastly, tenured professors have undergone review and have successfully secured a full-time appointment at their university with no expiration date; tenured professors may be hired from outside a university, though this position has the lowest yearly hiring rate among most universities.
It is common for universities to restrict their faculty hiring in the face of financial shocks.
Since the yearly hiring rate is highest among lecturers and assistant professors, financial shocks may lower the number of lecturer and assistant professors at public universities the most, if faculty hiring is affected by falls in funding.

Additionally, faculty salaries may be affected by changes in public university finances.
When the university has a lower budget for its yearly hiring, it may respond by lowering the salary they offer to new hires.
This possible effect may not be the same across each position of professor; tenured faculty are often hired away from another university, so that new tenured professor hires may be less likely to accept a lower salary offer from a public university.\footnote{
    Faculty often consider their outside options, as in offers for employment and/or promotion from other universities.
    See \cite{blackaby2005} for an overview of faculty outside options.
}
Yet salary for all the professors, not just new hires, may also be affected: multiple universities passed a university-wide pay-cut for their faculty in response to state budget cuts around the 2008 recession, for example.\footnote{
    Indeed, Cornell University implemented hiring freezes and a nominal salary reduction for professors, in anticipation of a financial shock in early 2021.
    The salary cut was not permanent, as the oncoming financial shock turned out to not be as serious as projected, so that the cut was returned to professors in the year 2021.
}
Faculty are not paid the same by position, so that there are financial consequences for a university promoting their own faculty.
In response to tightening fiscal restraints, universities may be less likely to grant tenure to assistant professors (as tenured professors generally receive a salary raise and long term contract), or to promote tenured professors from associate to full professor.

PARAGRAPH ON CONCURRENT TRENDS, WHICH MAY NOT BE CAUSAL.
CINCURRENT TRENDS IN SELECTIVITY (HOXBY 2009)
\begin{enumerate}
    \item CONSIDER TAKING THE PARAGRAPH FROM THE INTRO, AND PUT HERE.
\end{enumerate}

It is not immediately clear which effects will dominate, or whether tenured professors vs lecturers will be affected more by changes in their university's funding.
Yet, there is one empirical fact worth noting: a mean assistant or tenured professor earns more than double that of an adjunct lecturer per year (see \autoref{tab:illinois-summary}), and the tenure contract means universities must pay a higher salary for a guaranteed number years, more so than for contingent lecturers.
If a public university's primary obligation is to teach, and they must fulfil this objective with fewer and fewer resources, then they may substitute away from tenure-track and tenured professor towards contingent lecturers to achieve this obligation.\footnote{
    That is, substituted away from tenure-track and tenured professors according to the salary ratio, and the relative productivity between these two positions.
}
