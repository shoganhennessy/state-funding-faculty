%%%%%%%%%%%%%%%%%%%%%%%%%%%%%%%%%%%%%%%%%
%% Section on possible outcomes
\section{Conceptual Framework}
\label{sec:conceptual}

While state support for public higher education has stagnated at the same time as education costs rose, there are multiple possible ways that a university can respond.
\cite{NBERw23736} established that corresponding rises in tuition did not offset the falling state support, so that university spending (per student) fell in response to these persistent, negative state funding shocks.
There are multiple ways that these changes in finances may affect the faculty composition at public universities.

It is a priori unclear that stagnating state support and changes in faculty composition at public universities would be causally linked.
Over the 21st Century there have been drastic changes to college selectivity, where the top universities have become more selective while the average university less selective \citep{hoxby2009changing}.
While the average public university is becoming less selective, it is possible that their most productive or research-focused faculty more often move to more selective and prestigious (and often private) institutions, so that over time public universities (on average) substitute towards contingent lecturers. 
This is one way in which the trends may be concurrent, but not causal.

Employment composition changes likely arise via differences in hiring rate across levels of professor seniority.
Universities hire new professors most years, either to expand their departments or to replace leaving professors, but not hire at the same rate for each level of seniority.
The number of professors hired in each year is usually highest among non-tenured adjunct faculty, as these instructors are mostly on short-term or contingent contracts, so  are less costly to hire (or to let go) in response to yearly changes.
Tenure-track assistant professors are mostly hired with a four to six year contract, and formal agreement for tenure consideration at the end of this term.
Lastly, tenured professors have successfully secured a full-time appointment at their university with no expiration date; this position has the lowest yearly hiring rate among most universities.
Similarly, there is highest turn-over (including leaving the university) among lecturers, and lowest among full professors.
The differences in yearly hiring and turnover rate mean that if hiring is affected by state funding, then there will be faculty composition change in the years following budget cuts.

Faculty salaries may be affected by changes in public university finances.
When the university has a lower budget for its yearly hiring, it may respond by lowering the salary they offer to new hires.
This possible effect may not be the same across each position of professor; tenured faculty are often hired away from another university, so that new tenured professor hires may be less likely to accept the lower salary offers from public universities.\footnote{
    Faculty often consider their outside options, as in offers for employment and/or promotion from other universities.
    See \cite{blackaby2005} for an overview of faculty outside options.
}
Yet salary for all the professors, not just new hires, may also be affected: multiple universities passed a university-wide pay-cut for their faculty in response to state budget cuts around the 2008 recession, for example.\footnote{
    Indeed, Cornell University implemented hiring freezes and a nominal salary reduction for professors, in anticipation of a financial shock in early 2021.
    The salary cut was not permanent, as the oncoming financial shock turned out to not be as serious as projected, so that the cut was returned to professors in the year 2021.
}
Additionally, it is not immediately clear which faculty (among those already hired) will be most affected by the changes in their university's funding.
Yet, there is one empirical fact worth noting: the average assistant or tenured professor earns more than double that of a lecturer (see \autoref{tab:illinois-summary}).
If a public university's primary obligation is to teach, and they must fulfil this objective with less resources, then they may substitute away from tenure-track and tenured professors towards lecturers.

Similarly, its is not clear when the effects of funding short-falls will be realised.
University departments often employ faculty on multiple-year contracts (e.g., the tenure contract), and individual professors' decisions to retire, move university, or leave academia are multiple-year commitments, so that it may take multiple years for university funding to change the faculty composition, or trickle down to faculty.
For example, university departments coordinate hiring in year-long cycles, so that a budget short-fall in year $t$ may have no effect on the in-progress hiring committee and resulting hiring decisions, yet the hiring cycle in year $t+1$ may be postponed or cancelled in response.
As such, this paper investigates the dynamics of state funding shocks' effects on faculty, for the years following a university funding shock.
