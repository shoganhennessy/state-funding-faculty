%%%%%%%%%%%%%%%%%%%%%%%%%%%%%%%%%%%%%%%%%
%% Data section
\section{Data and Institutional Context}
\label{sec:data}

\subsection{Data Description}
The data used in this paper come from two primary sources: \citet[IPEDS]{ipeds} for data on university funding, finances, and enrolment, and \citet[IBHED]{ibhed} for data on every faculty member in the Illinois public university system.

IPEDS is a survey of higher educational institutions in the US, and legally requires institutions to participate in order to receive Federal Title IV student aid.\footnote{
    IPEDS does not necessarily cover the universe of US higher education institutions, yet in practice every public university and not for-profit four-year institution is represented.
}
Data are consistent for 1990--2021,\footnote{
    The years 1987-1989 are represented in these data in an incompletion fashion, so I focus on the years 1990 onwards.
    Year refers to the calendar year of the spring term --- i.e., 1990 refers to the academic year that ran August 1989 to July 1990.
}
and provide information on university funding, enrolment, and numerous other characteristics.\footnote{
    I combine the Urban Institute's 2018 compilation of IPEDS data for the years 1990--2017, and manually combine raw National Center for Education Statistics (NCES) data on 2018--2021 for all relevant variables.
    Figures for enrolment and faculty counts come from the raw NCES version of IPEDS for all years, addressing inconsistencies in the Urban Institute's data formulation for these variables.
    I also integrate Barron's selectivity index, from the 2009 rankings, to analyse effects for different levels of selectivity \citep{barrons2009}.
}
I restrict analysis to public, four-year, degree-granting institutions, as these institutions have the largest majority of faculty employed and students enrolled.
For-profit institutions employ and enrol a negligible share of professors and students respectively, while students at two-year institutions by majority intend to eventually enrol at a four-year institution \citep{mountjoy2022}, so that these institutions are not considered.
IPEDS reports the count of professors employed by position.\footnote{
    IPEDS gives a mean salary measure for faculty by rank, but these figures have many missing values and disagree with calculated values from other reputable sources.
    Over 40\% of university--year observations are missing in the IPEDS panel data-set, representing an average of 55\% undergraduate enrolment in each year.
    Yearly averages of faculty salaries by the non-missing values do not agree with trends in average professor salary over the sample time period compared to summary statistics provided in \cite{aau2021survey}.
    IPEDS values consistently report that public university professors are paid on average 20\% more than private sector faculty, which disagrees with other sources on the salary gap direction, and IPEDS figures are consistently higher than averages calculated from individual-level IBHED data.
    For these reasons I do not further analyse IPEDS salary data, and use salary data from IBHED to analyse faculty salaries.
    Notably, IPEDS data on faculty counts do not suffer from these problems.
}
This gives a resulting panel data-set, where each row represents a university-year, and includes columns for university funding and tuition revenue, plus total count for each faculty position (lecturer, assistant professor, tenured professor, and total faculty), and other university characteristics.
\autoref{tab:ipeds-summary} presents summary statistics for these variables in IPEDS data.

IPEDS provides information at the university level, but only provides aggregated information within a university.
To investigate outcomes for individual faculty members, I integrate individual-level data for all faculty members at all Illinois public universities 2010--2021.
These data allow me to investigate how individual faculty members are affected by funding cuts to their university, investigating individual-level professor outcomes unavailable in IPEDS data.
IBHED freely provides this information, as the state of Illinois is required to publicly report base salary and benefits\footnote{
    Real salary is computed by scaling nominal salary and benefits to 2021 dollars by the CPI-U.
} for all administrators, faculty members, and instructors employed by each public college or university.\footnote{
    Public Act 96-0266, effective 1 January 2010, is the relevant Illinois law that requires publicly publishing salary data for all public university faculty salary and benefits \citep{illinois-public-act}.
    A pdf copy of Public Act 096-0266 is included in the supplmentary data of this paper.
    This law applies to all nine Illinois public universities: Chicago State University, Eastern Illinois University, Governors State University, Illinois State University, Northeastern Illinois University, Northern Illinois University, Southern Illinois University (all five campuses), University of Illinois (all four campuses), Western Illinois University.
}
These data provide the basis to build a panel of Illinois public university faculty across years 2010--2021; I define a faculty member as an individual by their first plus last name and university pairing,\footnote{
    Some faculty are listed multiple times in these panel data.
    For example, Professor Alberto Agustin Lopez-Scala of the University of Illinois Chicago has two appointments in each year 2013--2021, one as an adjunct faculty and one as a department director.
    For this observation I take the highest paid position (department director) as the primary appointment and drop the secondary appointment (adjunct faculty).
    In analysis of faculty exit rate, I collapse the appointments into one and only consider faculty exits as having a faculty appointment versus no appointment.
}
and link this database to IPEDS for data on their university employer.
The Illinois sample represents 16,932 professors in the year 2010 and 15,352 in the year 2021, with summary statistics presented in \autoref{tab:illinois-summary}.
Analysis of professors' in their first year on the job focuses on the subset of professors with observed year of hiring 2011--2021 (i.e., after the panel's first year), representing 1,778 professors in 2011, and 9,099 in 2021.

\subsection{Trends in Funding, Enrolment, and Faculty Counts}
\label{sec:trends}

States vary largely in how much they fund their public university systems, thanks in part to the state-wide budgeting process \citep{NBERw23736}.
Planning for an annual budget begins two years ahead of the fiscal year, and the legislature votes to approve or reject the governor office's budget request a number of months before the fiscal year begins.\footnote{
    \cite{NBERw23736} present a full discussion of the decision-making process for state funding, drawing on administrative records originally analysed by \cite{parmley2009state}.
}
US state governments, by majority, are legally obliged to run a balanced-budget, so that yearly variation in tax revenues (e.g., caused by changing economic conditions) necessarily led to state government spending cuts.
Public universities have lower lobbying power than other state institutions, so often bear the brunt of these funding cuts \citep{delaney2011state}.\footnote{
    \cite{delaney2011state} fully describe the financial environment of state expenditures, and what makes spending on higher education an attractive area for state governments to expand funding during years of higher tax revenues, and retract funding in leaner years.
    An analysis of state expenditures for the years 1980-2004 (overlapping with the sample for this analysis) provides solid evidence for these trends, and \autoref{fig:funding} observes these same trends.
}
Additionally, the number of higher education students in each state varies thanks to the size of each birth cohort.
For example, the birth cohort of 1971 was larger than that of 1970 or 1972, leading to more student demand for limited public university places for students turning 18 in 1989 \citep{bound2007cohort}.
These features lead to yearly variation in state funding not seen in other revenues sources, such as federal funding.

State funding for public universities stagnated over the last thirty years.
\autoref{fig:funding} show the trends in revenues for the mean public university for the years 1990--2021.
We see a rise in total revenues received by public universities (from all sources), and a notable increase in mean tuition revenues from \$48 million per year-university to \$150 million per year-university.
At the same time, total state funding stagnated at around \$100 million per year-university for 1990--2008, falling around 2008 and have not recovered ever since.
While public universities experienced a stagnation in state support, private universities were not exposed to the same constraints, receiving \$37,000 per student in 1990 and \$49,000 in 2021, experiencing no corresponding decline in any specific component.

At the same time, student enrolment at public universities rose precipitously.
6.2 million students were enrolled in public universities in 1990, and this number rose by 47\% to 9.1 million in 2021, with most of the increase occurring after the year 2000.
\autoref{fig:enrolment} shows that total enrolment at private universities has also risen over the same time period, but not as drastic in either relative or absolute terms; the mean private university grew from 9,800 students in 1990 to 11,800 in 2021.
This means that funding per student has stagnated for all sources (seen in \autoref{fig:mean-funding-fte}), falling from \$11,000 per student on average in 1990 to less than \$8,000 per student in 2021.

There are large differences in the average number of professors per student between the private and public sector.
\autoref{fig:fte-perprof} shows that private universities start with an average of 40 students for every one full professor, and little change thereafter.
Public universities start with 40 students per professor, and by 2021 there are 52.4 students professors for every full professor --- with the largest rise coming in the 2008--2011 time period.
The general trend is similar for the number of assistant professors at both public and private universities.
Private and public universities have similar numbers of associate and full professors before the year 2000.
After 2000, the number of professors at public univeristies fell drastically, with the average public university has 6 fewer full professors per hundred students than the average private university by 2021.
Over the same time period, we see the rise in use of lecturers; lecturers were employed at similar rates in private and public sectors in 1990.
Since 1990, both sectors began employing more lecturers on a per-student basis.
In sum, this means public universities have begun relying on lecturers, because they decreased their professor counts, compared to private unviersities who started employing more lecturers and professors.

\subsection{Trends in Illinois}
\label{sec:trends-illinois}
Illinois funding for higher education has stagnated, experiencing serious declines in the decade 2010--2021 --- similar in magnitude to the nation-wide decline over 1990--2021.
State funding Illinois public universities fell by over 50\% over 2010--2021, in both absolute and per-student terms (see \autoref{fig:illinois-funding}).

There was not only a stagnation in state funding in this time period, but also large annual rises and falls, particularly around 2016.
In the calendar year 2015, partisan disagreements between the democratic legislature and republican governor led to the 2016 fiscal year starting with no state budget.
State agencies, and higher education institutions, employed accounting techniques to continue operating without any resources provided by the state government.\footnote{
    Fiscal year 2016 refers to June 2015 to June 2016, so is the same as the academic year definition.
}
While most public universities were able to stay open, there were drastic revenue and spending cuts in response to the budget impasse, as it continued through fiscal year 2017, and ended with a new budget restoring funding to state institutions for 2018.
This means that Illinois public universities exhibit sizable changes in their state funding over 2010--2021, of similar order to those for the rest of the country over 1990--2021.
Additionally, the 2016 episode stemmed entirely from political disagreements, and not from state decisions regarding higher education and its finances, exhibiting how state-level changes in funding affect public universities thanks to unrelated issues \citep{young2020squandered}.
