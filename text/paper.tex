% Page settings
\documentclass[notitlepage,12pt]{article}
\usepackage{fancyvrb, verbatim, listings}                                  % Various for inserting/hosting code
\usepackage{setspace}                                                      % Space between lines (customisable)
\usepackage[margin=2.54cm]{geometry}                                       % Set margins (standard is 2.54cm)
\usepackage[UKenglish,cleanlook]{isodate}                                  % Set default date and date display
\usepackage{fancyhdr} \pagestyle{fancy}                                    % Line the top of the page
\usepackage[activate={true,nocompatibility},final,tracking=true,kerning=true,spacing=true,factor=1100,stretch=10,shrink=10]{microtype}
\microtypecontext{spacing=nonfrench}                                       % Change font and minor spacings to look nicer
\setlength{\parskip}{0.5\baselineskip}                                     % Set default paragraph behaviour: skip a line
%\setlength{\parindent}{0pt}                                                % Set default paragraph behaviour: no indent
%% Packages to use:
% Tables + figures
\usepackage{graphicx}                                                      % control over the import of graphics
\usepackage[table,dvipsnames]{xcolor}                                      % Define tables (with colour control)
\usepackage{float}                                                         % Control over graphics + tables float
\usepackage[justify]{ragged2e}                                             % control over text alignment
\usepackage{booktabs}                                                      % caption graphics + tables (with name in bold)
\usepackage[labelfont=bf]{caption} 
\usepackage{subcaption}
% caption graphics + tables (with name in bold)
\usepackage{tikz}                                                          % Draw graphs inline, guide: https://sites.google.com/site/kochiuyu/Tikz
\usetikzlibrary{shapes,decorations,arrows,calc,arrows.meta,fit,positioning}
\tikzset{
    -Latex,auto,node distance =1 cm and 1 cm,semithick,
    state/.style ={ellipse, draw, minimum width = 0.7 cm},
    point/.style = {circle, draw, inner sep=0.04cm,fill,node contents={}},
    bidirected/.style={Latex-Latex,dashed},
    el/.style = {inner sep=2pt, align=left, sloped}
}
\usepackage{pgfplots} \pgfplotsset{compat=1.16}                            % Automatic graphing, read https://www.overleaf.com/learn/latex/pgfplots_package for examples
% Maths + numbers
\usepackage{mathtools}                                                     % Various maths functions
\usepackage{amssymb}                                                       % Various maths functions
\usepackage{amsmath}                                                       % Various maths functions
\usepackage{dsfont}                                                        % Various maths functions
\usepackage{centernot}                                                     % center \not usage
\usepackage{siunitx} \sisetup{round-mode=places, round-precision=3}        % Formalise use of units and numbers among text
\DeclareMathOperator{\eps}{\varepsilon}                                    % epsilon short hand shortcut
\DeclareMathOperator{\st}{\text{ s.t. }}                                   % "such that" short hand shortcut
\DeclareMathOperator{\then}{\text{ then }}                                 % "then" in equation shortcut
\DeclareMathOperator{\ifeq}{\text{ if }}                                   % "if" in equation shortcut
\DeclareMathOperator{\oreq}{\text{ or }}                                   % "or" in equation shortcut
\DeclareMathOperator{\andeq}{\text{ and }}                                 % "and" in equation shortcut
\DeclareMathOperator{\all}{,\; \forall}                                    % all (with spacing) in equation shortcut
\DeclareMathOperator{\N}{\mathbb{N}}                                       % N, natural number, shortcut
\DeclareMathOperator{\R}{\mathbb{R}}                                       % R, real number, shortcut
\DeclareMathOperator{\Ll}{\mathcal{L}}                                     % L, Lagrangian, shortcut
\renewcommand{\vec}[1]{\boldsymbol{\mathit{#1}}}                           % vector notation shortcut
\newcommand{\mat}[1]{\boldsymbol{\mathit{#1}}}                             % matrix notation shortcut
\DeclarePairedDelimiter\abs{\lvert}{\rvert}                                % absolute value notation shortcut
\DeclarePairedDelimiter\norm{\lVert}{\rVert}                               % norm notation shortcut
\newcommand{\Prob}[1]{\Pr\left( #1 \right)}                         % SHortcut for probability notation
\newcommand{\Probgiven}[2]{\Pr\left( #1 \, \middle\vert \, #2 \right)} % SHortcut for probability notation, given
\newcommand{\E}[2][]{\mathbb{E}_{#1} \left[ #2 \right]}                    % Expectation (with optional subscript) shortcut
\newcommand{\Egiven}[3][]{\mathbb{E}_{#1} \left[ #2 \, \middle\vert \, #3 \right]} % Expectation given (with optional subscript) shortcut
\newcommand{\Var}[2][]{\text{Var}_{#1} \left( #2 \right)}                  % Variation (with optional subscript) shortcut
\newcommand{\Cov}[1]{\text{Cov} \left( #1 \right)}                         % Covariance (with optional subscript) shortcut
\newcommand{\median}[1]{\text{median} \left( #1 \right)}                   % Median (with optional subscript) shortcut
\newcommand{\indicator}[1]{\mathds{1}\left\{ #1 \right\}}                  % SHortcut for indicator function
\newcommand{\diff}[2][]{\frac{d#1}{d#2}}                                   % SHortcut for differential fraction as a function
\newcommand{\partialdiff}[2][]{\frac{\partial#1}{\partial#2}}              % SHortcut for partial differential fraction as a function
\newcommand{\converge}[1]{\xrightarrow{ #1 \to\infty}}                     % SHortcut for convergence arrow
\renewcommand{\hat}[1]{\widehat{#1}}                                       % Default estimator notation is widehat
\renewcommand{\bar}[1]{\overline{#1}}                                      % Make over bar look nicer
\renewcommand{\tilde}[1]{\widetilde{#1}}                                   % Make over tilde look better
% Citations
\usepackage[longnamesfirst]{natbib}                                        % Citation package, see https://en.wikibooks.org/wiki/LaTeX/Bibliography_Management#Natbib
\usepackage[backref=page]{hyperref}                                        % Allow for links across the text, with colour options
\hypersetup{colorlinks=true, linkcolor=blue, citecolor=blue, filecolor=magenta, urlcolor=blue}
\def\sectionautorefname~#1\null{Section~#1\null}                       % Fix autoref for sections
\def\equationautorefname~#1\null{Equation~(#1)\null}                       % Fix autoref for equations
% IMPORTANT: follow style guide here https://github.com/Wookai/paper-tips-and-tricks


%%%%%%%%%%%%%%%%%%%%%%%%%%%%%%%%%%%%%%%%
%% Title page
% Author
\author{Senan Hogan-Hennessy\thanks{
        For helpful comments I thank
        Levon Barseghyan,
        Francine Blau,
        Ryan Dycus,
        Ronald Ehrenberg,
        Michael Lovenheim,
        Jake Meyer,
        Douglas Miller,
        and Evan Riehl,
        as well as seminar participants at Cornell University (Autumn 2022) for helpful discussion.} \\
    \vspace{0.1cm}
    Economics Department, Cornell University
}
% Title
\title{Stagnating State Funding for Higher Education and its Effect on Faculty at US Universities
    }
\date{This version: \today\thanks{
    This project's Github repository, with materials and anonymised data for replication, is available at 
    \url{https://github.com/shoganhennessy/state-funding-faculty}.
    Any comments or suggestions may be sent to me at \href{mailto:seh325@cornell.edu}{\nolinkurl{seh325@cornell.edu}}, or raised as an issue on the Github project.
    } \\ \vspace{0.1cm}
    \href{https://github.com/shoganhennessy/state-funding-faculty/blob/main/state-funding-faculty-2023.pdf}{Newest version available here.}
}
\rhead{\today}
\lhead{Stagnating State Funding for Higher Education and its Effect on Faculty.}
% Begin
\begin{document}
\maketitle
\thispagestyle{empty}
% Abstract
\begin{abstract}
    \noindent
    Public universities employ more lecturers and fewer professors than at any other point in the last thirty years, relative to student enrolment.
At the same time, state funding for higher education has stagnated.
This paper shows that the decline in state funding led to a substitution away from professors toward lecturers at US public universities.
Using a shift-share approach to instrument for state funding, I find that universities employ 4.4\% more lecturers per student following a 10\% funding cut.
This shift is accompanied by a reduction in assistant professors and full professors per student by 1.4\% and 1.2\%, respectively.
Incumbent professors' salaries, promotion rates, and quit rates at Illinois universities remain unaffected by funding cuts, indicating that the substitution arose from limiting the hiring of new tenure-track/tenured professors.
Stagnating state funding impacts public universities and faculty, likely contributing to the deterioration of student outcomes at public universities since the 1990s.

\vfill
\noindent
\textbf{Key-words:}
Faculty,
Higher Education,
State Funding

\vspace{0.05cm}
\noindent
\textbf{JEL Codes:} H72, I22, J45

\end{abstract}

%%%%%%%%%%%%%%%%%%%%%%%%%%%%%%%%%%%%%%%%%
%% Starting the paper
\newpage
\setcounter{page}{1}
\doublespacing
% Introduction section
\noindent
Public universities educate the majority of higher education students in the US, yet have experienced a stagnation in state funding over the last three decades .
At the same time, public university employment of lecturers dramatically increased at public universities, while enrolment rose dramatically.
Previous research has examined the effects of the decline in state funding on student outcomes \citep{NBERw23736,NBERw27885}, but there is little evidence of the effects of this funding decline on faculty.
I show that four-year, degree-granting public universities have experienced a systematic decline in state funding per student, and use a shift-share instrument to show that this fall led to a substitution away from tenure-track/tenured professors towards contingent lecturers.
Analysis of all public university faculty in the state of Illinois shows that incumbent faculty were relatively unaffected, in terms of wages, promotion and quit rates, implying that changes in faculty composition arose by disrupting public universities' ability to hire or replace their tenure-track/tenured faculty. 
Private universities were not exposed to similar financial constraints during this same time period, and do not exhibit the substitution away from tenure-track/tenured professors, so that the stagnating state funding for public universities has implications for the wider structure of US higher education and research.

There are various margins through which funding cuts can impact a university's faculty.
A university's level of resources affect the ability that departments can hire new professors, retain their current faculty, or provide retirement incentives, so it is natural to expect that funding cuts may affect the composition of professors at a university.
Similarly, the wages a university pays their faculty is affected by funding for the university, including the starting salary it offers to new professors and the amount of yearly raises (if any) for incumbent faculty.
So that it is natural to expect that systematic funding cuts will lead to changes in faculty composition, and possibly affect incumbent professors, too.
Yet, it is a priori unclear whether/how much each of these margins adjusts to funding cuts, as these decision depend on how the university administrations view the productivity of different inputs, plus the degree of wage and employment stickiness over time.

US universities are widely considered the highest performing in the world, yet there are consequential differences between its universities that operate in the private sector and those established by state governments.
Public universities are subject to numerous state-level administrative laws, and rely on their state governments for funding: an average public university received around \$11,600 of state funding per student in 1990, and only \$8,300 in 2021.
This fall is driven by a stagnation in the absolute level of state funding for higher education in most states, combined with a large increase of 46\% in student enrolment at public universities over the same time period.
Similarly, the number of lecturers per student over doubled, while the number of tenure-track/tenured professors per students fell by $-23$\% from 1990 to 2021.

I use data from the \cite{ipeds} to study the impact of changes in state funding for US public universities 1990--2017.
I combine these data with panel data for every faculty member at Illinois public universities for 2010--2021 \citep{ibhed}, to measure how incumbent faculty (i.e., those already employed at the university) are affected by changes in state funding.
These data allow me to answer the following research question: if a US public university receives an extra \$1,000 for each of their students (or a 10\% increase), do they change the composition of faculty they employ?
If so, which positions do universities substitute away from, or towards, and do these funding cuts affect the faculty members themselves?

I employ a shift-share instrument to identify changes in state funding for higher education, which exploits how universities differ in their financial reliance on state funding, combined with yearly shocks to the amount an entire state funds higher education --- following \cite{NBERw23736,NBERw27885}.
State governments decide how to fund higher education by a complicated process which can be influenced by local economic conditions, changing state priorities, or even lobbying from the state universities themselves.
As such, state funding is not randomly assigned to universities and so it is necessary to employ an identification strategy, such as the shift-share instrument, to account for this.
The shift-share instrument interacts how much a university relies on state funding as a percent of revenues in a base period with the entire state's total funding per higher education student for each year, exploiting changes in funding and how much each university historically relied on state funding for higher education.
Additionally, I estimate effect of state funding on faculty multiple years after the initial funding cuts, using the local projections method thanks to the presence of time-series confounding between the funding shock and later years' level of state funding.\footnote{
    To the best of my knowledge, this is also the first paper to use the local projections method for a shift-share instrument.
    %Previous work ignored the potential time-series confoduning, and estimated linear instrumental variables.
}

Falls in state funding affect faculty composition at public universities, which substitute away from tenured and tenure-track professors towards contingent lecturers.
A funding cut of \$1,000 per student leads to an average university employing 6 more lecturers;
in percentage terms, a funding cut of 10\% per student leads to a fall of 1.4\% in the number of assistant professors per student at a university, a fall of 1.2\% for tenured professors, and an increase in 4.4\% the number of lecturers per student.
Local projection estimates show that these effects linger for three to four years after the initial funding shock, showing that the effect is not isolated to the year of the funding shock. 
Over the same time period, state funding per student fell by around 35\%, count of (tenure-tack/tenured) professors per student fell by 9\%, and lecturers per student increased by 99\%, so that these results show that falls in state funding explain around 53\% of the fall in professors per student
% i.e. -(35 * 0.137) / -9
and 15.5\% of the rise in lecturers per student.
% ie., (35 * 0.437) / 99
Additionally, I use individual level data on all Illinois public university professors in 2010-2021 to investigate whether the shocks to state funding affected individual professors.
Incumbent professors were not meaningfully affected, in terms of total salary, promotion rate, or rate of leaving the Illinois public university system.
Yet the hiring rate for new professors at public universities was negatively impacted by the falls in state funding (and by the funding shocks).
This implies that faculty composition change arose by limiting the hiring of new professors; public universities increasingly hired contingent lecturers, and increasingly did not replace their retiring (or leaving) professors.

Mine is the first paper to provide causal evidence on the impact of state funding on faculty composition, and wages.
The closest related research on higher education funding has examined how university spending affected graduation rates and levels of student debt \citep{NBERw23736,NBERw27885}, and university finances \citep{miller2022making,bound2019public,brown2014endowment}.
For faculty outcomes, multiple theoretically model university decision-making and faculty hiring (see e.g., \citealt{abe2015implications,johnson2009jep,NBERc13879}), while others measured how faculty at universities were affected by endowment shocks and the 2008 recession \citep{brown2014endowment,turner2014impact}.
These papers measure effects of funding and changes in university revenues, but do not measure effects on faculty, or other outcomes related to university instruction.

My results provide suggestive evidence on mechanisms for the previous research connecting state funding cuts for higher education and worsening student outcomes.
Substitution towards lecturers is likely a cost-cutting measure for funding constrained universities, though relying on these faculty --- who are often over-worked and not granted long-term employment protections --- may lead to worse student outcomes \citep{ehrenberg2005tenured,zhu2021limited,jaeger2011examining}.
Lastly, my results provide evidence that the long term trends in higher education funding are causally related, contributing to the literature in the long run trends in faculty outcomes \citep{ehrenberg2003studying} and trends in US higher education and funding \citep{hoxby2009changing}.

This paper proceeds as follows.
\autoref{sec:data} describes the data for university finances and faculty in Illinois, and trends in public university funding for the last three decades.
\autoref{sec:conceptual} gives the conceptual framework for how state funding may affect faculty.
\autoref{sec:empirics} draws the empirical framework for isolating the causal effects of state funding on faculty composition and individual faculty, and \autoref{sec:results} presents the empirical results.
\autoref{sec:discussion} discusses the context and implications for the findings.
\autoref{sec:conclusion} concludes.

% Data section
%%%%%%%%%%%%%%%%%%%%%%%%%%%%%%%%%%%%%%%%%
%% Data section
\section{Data and Institutional Context}
\label{sec:data}

\subsection{Data Description}
The data used in this paper come from two primary sources: \citet[IPEDS]{ipeds} for data on university funding, finances, and enrolment, and \citet[IBHED]{ibhed} for data on every faculty member in the Illinois public university system.

IPEDS is a survey of higher educational institutions in the US, and legally requires institutions to participate in order to receive Federal Title IV student aid.\footnote{
    IPEDS does not necessarily cover the universe of US higher education institutions, yet in practice every public university and not for-profit four-year institution is represented.
}
Data are consistent for 1990--2021,\footnote{
    The years 1987-1989 are represented in these data in an incompletion fashion, so I focus on the years 1990 onwards.
    Year refers to the calendar year of the spring term --- i.e., 1990 refers to the academic year that ran August 1989 to July 1990.
}
and provide information on university funding, enrolment, and numerous other characteristics.\footnote{
    I combine the Urban Institute's 2018 compilation of IPEDS data for the years 1990--2017, and manually combine raw National Center for Education Statistics (NCES) data on 2018--2021 for all relevant variables.
    Figures for enrolment and faculty counts come from the raw NCES version of IPEDS for all years, addressing inconsistencies in the Urban Institute's data formulation for these variables.
    I also integrate Barron's selectivity index, from the 2009 rankings, to analyse effects for different levels of selectivity \citep{barrons2009}.
}
I restrict analysis to public, four-year, degree-granting institutions, as these institutions have the largest majority of faculty employed and students enrolled.
For-profit institutions employ and enrol a negligible share of professors and students respectively, while students at two-year institutions by majority intend to eventually enrol at a four-year institution \citep{mountjoy2022}, so that these institutions are not considered.
IPEDS reports the count of professors employed by position.\footnote{
    IPEDS gives a mean salary measure for faculty by rank, but these figures have many missing values and disagree with calculated values from other reputable sources.
    Over 40\% of university--year observations are missing in the IPEDS panel data-set, representing an average of 55\% undergraduate enrolment in each year.
    Yearly averages of faculty salaries by the non-missing values do not agree with trends in average professor salary over the sample time period compared to summary statistics provided in \cite{aau2021survey}.
    IPEDS values consistently report that public university professors are paid on average 20\% more than private sector faculty, which disagrees with other sources on the salary gap direction, and IPEDS figures are consistently higher than averages calculated from individual-level IBHED data.
    For these reasons I do not further analyse IPEDS salary data, and use salary data from IBHED to analyse faculty salaries.
    Notably, IPEDS data on faculty counts do not suffer from these problems.
}
This gives a resulting panel data-set, where each row represents a university-year, and includes columns for university funding and tuition revenue, plus total count\footnote{
    Real salary is computed by scaling nominal salary to 2021 dollars by the CPI-U.
} for each faculty position (lecturer, assistant professor, tenured professor, and total faculty), and other university characteristics.
\autoref{tab:ipeds-summary} presents summary statistics for these variables in IPEDS data.

IPEDS provides information at the university level, but only provides aggregated information within a university.
To investigate outcomes for individual faculty members, I integrate individual-level data for all faculty members at all Illinois public universities 2010--2021.
These data allow me to investigate how individual faculty members are affected by funding cuts to their university, investigating outcomes are more granular details than total faculty counts.
IBHED freely provides this information, as the state of Illinois is required to publicly report base salary and benefits for all administrators, faculty members, and instructors employed by each public college or university.\footnote{
    Public Act 96-0266, effective 1 January 2010, is the relevant Illinois law that requires publicly publishing salary data for all public university faculty salary and benefits \citep{illinois-public-act}.
    A pdf copy of Public Act 096-0266 is included in the supplmentary data of this paper.
    This law applies to all nine Illinois public universities: Chicago State University, Eastern Illinois University, Governors State University, Illinois State University, Northeastern Illinois University, Northern Illinois University, Southern Illinois University (all five campuses), University of Illinois (all four campuses), Western Illinois University.
}
These data provide the basis to build a panel of Illinois public university faculty across years 2010--2021; I define a faculty member as an individual by their first plus last name and university pairing,\footnote{
    Some faculty are listed multiple times in these panel data.
    For example, Professor Alberto Agustin Lopez-Scala of the University of Illinois Chicago has two appointments, one as an adjunct faculty and one as a department director.
    For this observation I take the highest paid position (department director) as the primary appointment and drop the secondary appointment (adjunct faculty).
    In analysis of faculty exit rate, I collapse the appointments into one and only consider faculty exits as having a faculty appointment versus no appointment.
}
and link this database to IPEDS for data on their university employer.
The Illinois sample represents 16,932 professors in the year 2010 and 15,352 in the year 2021, with summary statistics presented in \autoref{tab:illinois-summary}.
Analysis of professors' in their first year on the job focuses on the subset of professors with observed year of hiring 2011--2021 (i.e., after the panel's first year), representing 1,778 professors in 2011, and 9,099 in 2021.

\subsection{Trends in Funding, Enrolment, and Faculty Counts}
\label{sec:trends}

States vary largely in how much they fund their public university systems, thanks in part to the state-wide budgeting process \citep{NBERw23736}.
Planning for an annual budget begins two years ahead of the fiscal year, and the legislature votes to approve or reject the governor office's budget request a number of months before the fiscal year begins.\footnote{
    \cite{NBERw23736} present a full discussion of the decision-making process for state funding, drawing on administrative records originally analysed by \cite{parmley2009state}.
}
US state governments, by majority, are legally obligated to run a balanced-budget, so that yearly variation in tax revenues (e.g., caused by changing economic conditions) necessarily affect state expenditures.
Public universities have lower lobbying power than other state institutions, so often bear the brunt of state government funding cuts \citep{delaney2011state}.\footnote{
    \cite{delaney2011state} fully describe the financial environment of state expenditures, and what makes spending on higher education an attractive area for state governments to expand funding during years of higher tax revenues, and retract funding in leaner years.
    An analysis of state expenditures for the years 1980-2004 (overlapping with the sample for this analysis) provides solid evidence for these trends, and \autoref{fig:funding} observes these same trends.
}
Additionally, the number of higher education students in each state varies thanks to the size of each birth cohort.
For example, the birth cohort of 1971 was larger than that of 1970 or 1972, leading to more student demand for limited public university places for students turning 18 in 1989 \citep{bound2007cohort}.
These features lead to yearly variation in state funding not seen in other revenues sources, such as federal funding.

State funding for public universities stagnated over 1990--2021.
\autoref{fig:funding} show the trends in revenues for the mean public university for the years 1990--2021.
We see a rise in total revenues received by public universities (from all sources), and a notable increase in mean tuition revenues from \$48 million per year-university to \$150 million per year-university.
At the same time, total state funding stagnated at around \$100 million per year-university for 1990--2008, falling around 2008 and have not recovered ever since.
While public universities experienced a stagnation in state support, private universities were not exposed to the same constraints, receiving \$37,000 per student in 1990 and \$49,000 in 2021, experiencing no corresponding decline in any specific component.

At the same time, student enrolment at public universities rose precipitously.
6.2 million students were enrolled in public universities in 1990, and this number rose by 47\% to 9.1 million in 2021, with most of the increase occurring after the year 2000.
\autoref{fig:enrolment} shows that total enrolment at private universities has also risen over the same time period, but not as drastic in either relative or absolute terms; the mean private university grew from 9,800 students in 1990 to 11,800 in 2021.
This means that funding per student has stagnated for all sources (seen in \autoref{fig:mean-funding-fte}), falling from \$11,000 per student on average in 1990 to less than \$8,000 per student in 2021.

There are large differences in the average number of professors per student between the private and public sector.
\autoref{fig:fte-perprof} shows that private universities start with an average of 40 students for every one full professor, and little change thereafter.
Public universities start with 40 students per professor, and by 2021 there are 52.4 students professors for every full professor --- with the largest rise coming in the 2008--2011 time period.
The general trend is similar for the number of assistant professors at both public and private universities.
Private and public universities have similar numbers of associate and full professors before the year 2000, yet this number fell by over 20\% afterwards only for public universities: in 2021 the average public university has 6 fewer full professors per hundred students than the average private university.
Over the same time period, we see the rise in use of lecturers; lecturers were were employed at similar rates in private and public sectors in 1990, but have public universities increased their reliance on lecturers more than private universities ever since.

\subsection{Trends in Illinois}
\label{sec:trends-illinois}
Illinois funding for higher education has stagnated, and has experienced serious declines in the decade 2010--2021 --- similar in magnitude to the nation-wide decline over 1990--2021.
State funding Illinois public universities fell by over 50\% over 2010--2021, in both absolute and per-student terms (see \autoref{fig:illinois-funding}).
%Similar to the figures for the rest of the country, there has been a corresponding substitution towards tuition revenue.

% Paragraph describing how IL is similar to rest of country.

There was not only a stagnation in state funding in this time period, but also large annual rises and falls, particularly around 2016.
In the calendar year 2015, partisan disagreements between the democratic legislature and republican governor led to the 2016 fiscal year starting with no state budget.
State agencies, and higher education institutions, employed accounting techniques to continue operating without any resources provided by the state government.\footnote{
    Fiscal year 2016 refers to June 2015 to June 2016, so is the same as the academic year definition.
}
While most public universities were able to stay open, there were drastic revenue and spending cuts in response to the budget impasse, as it continued through fiscal year 2017, and ended with a new budget restoring funding to state institutions for 2018.
This means that Illinois public universities exhibit sizable changes in their state funding over 2010--2021, of similar order to those for the rest of the country over 1990--2021.
Additionally, the 2016 episode stemmed entirely from political disagreements, and not from state decisions regarding higher education and its finances \citep{young2020squandered}, exhibiting how state-level changes in funding affect public universities thanks to unrelated issues. 


% Conceptual Framework
%%%%%%%%%%%%%%%%%%%%%%%%%%%%%%%%%%%%%%%%%
%% Section on possible outcomes
\section{Conceptual Framework}
\label{sec:conceptual}

While state support for public higher education has stagnated at the same time as education costs rose, there are multiple possible ways that a university can respond.
\cite{NBERw23736} established that corresponding rises in tuition did not offset the falling state support, so that university spending (per student) fell in response to these persistent, negative state funding shocks.
There are multiple ways that these changes in finances may affect the faculty composition at public universities.

It is a priori unclear that stagnating state support and changes in faculty composition at public universities would be causally linked.
Over the 21st Century there have been drastic changes to college selectivity, where the top universities have become more selective while the average university less selective \citep{hoxby2009changing}.
While the average public university is becoming less selective, it is possible that their most productive or research-focused faculty more often move to more selective and prestigious (and often private) institutions, so that over time public universities (on average) substitute towards contingent lecturers. 
This is one way in which the trends may be concurrent, but not causal.

Employment composition changes likely arise via differences in hiring rate across levels of professor seniority.
Universities hire new professors most years, either to expand their departments or to replace leaving professors, but not hire at the same rate for each level of seniority.
The number of professors hired in each year is usually highest among non-tenured adjunct faculty, as these instructors are mostly on short-term or contingent contracts, so  are less costly to hire (or to let go) in response to yearly changes.
Tenure-track assistant professors are mostly hired with a four to six year contract, and formal agreement for tenure consideration at the end of this term.
Lastly, tenured professors have successfully secured a full-time appointment at their university with no expiration date; this position has the lowest yearly hiring rate among most universities.
Similarly, there is highest turn-over (including leaving the university) among lecturers, and lowest among full professors.
The differences in yearly hiring and turnover rate mean that if hiring is affected by state funding, then there will be faculty composition change in the years following budget cuts.

Faculty salaries may be affected by changes in public university finances.
When the university has a lower budget for its yearly hiring, it may respond by lowering the salary they offer to new hires.
This possible effect may not be the same across each position of professor; tenured faculty are often hired away from another university, so that new tenured professor hires may be less likely to accept the lower salary offers from public universities.\footnote{
    Faculty often consider their outside options, as in offers for employment and/or promotion from other universities.
    See \cite{blackaby2005} for an overview of faculty outside options.
}
Yet salary for all the professors, not just new hires, may also be affected: multiple universities passed a university-wide pay-cut for their faculty in response to state budget cuts around the 2008 recession, for example.\footnote{
    Indeed, Cornell University implemented hiring freezes and a nominal salary reduction for professors, in anticipation of a financial shock in early 2021.
    The salary cut was not permanent, as the oncoming financial shock turned out to not be as serious as projected, so that the cut was returned to professors in the year 2021.
}
Additionally, it is not immediately clear which faculty (among those already hired) will be most affected by the changes in their university's funding.
Yet, there is one empirical fact worth noting: the average assistant or tenured professor earns more than double that of a lecturer (see \autoref{tab:illinois-summary}).
If a public university's primary obligation is to teach, and they must fulfil this objective with less resources, then they may substitute away from tenure-track and tenured professors towards lecturers.

Similarly, its is not clear when the effects of funding short-falls will be realised.
University departments often employ faculty on multiple-year contracts (e.g., the tenure contract), and individual professors' decisions to retire, move university, or leave academia are multiple-year commitments, so that it may take multiple years for university funding to change the faculty composition, or trickle down to faculty.
For example, university departments coordinate hiring in year-long cycles, so that a budget short-fall in year $t$ may have no effect on the in-progress hiring committee and resulting hiring decisions, yet the hiring cycle in year $t+1$ may be postponed or cancelled in response.
As such, this paper investigates the dynamics of state funding shocks' effects on faculty, for the years following a university funding shock.

% Empirical methods
%%%%%%%%%%%%%%%%%%%%%%%%%%%%%%%%%%%%%%%%%
%% Empirical section
\section{Empirical Framework}
\label{sec:empirics}

This paper identifies the effects of changes in state funding on faculty.
However, a university's state funding is not exogenous to state decisions for support of higher education, or exogenous to internal institutional decisions.
Instead, the state government and university administration undertake a complex process of allotting resources across multiple different priorities, including instruction, research, or between departments.
For example, it may be the case that a state government restricts funding for only its lowest performing universities in response to a budget shock, introducing a treatment selection-on-outcome threat to identification. 
Importantly, funding from a university's state government provides opportunity to address such endogeneity.

\subsection{State Funding Shocks}
\label{sec:approp-shocks}

I use a shift-share instrument to address endogeneity concerns for the amount of funding a state allots to each of its public universities.
\cite{NBERw23736,NBERw27885} develop the instrument for public university finances by exploiting a shift-share instrument for changes in state-level funding interacted with university reliance on state funding in a base period.
\begin{align}
    \label{eqn:public-instrument}
    Z_{i,t} &\coloneqq - \left[
    \left( \frac{\text{Total State Funding}_{s(i),t}}{\text{Student Population}_{s(i),t}} \right)
    \sum_{\tau = 0}^{3} \frac 14
    \left( \frac{\text{State Funding}_{i,1990 + \tau}}{\text{Total Revenues}_{i,1990 + \tau}} \right) \right]
\end{align}
The system exploits the fact that institutions who rely on state funding more will be affected by state funding shocks.
$Z_{i,t}$ is the instrument for state funding for institution $i$ in year $t$, interacting the average funding for universities in state $s(i)$ with reliance on state funding relative to total revenues, averaged across the base years 1990--1993.\footnote{
    1990--1993 are defined as data for public university finance data are most comparable (i.e. without many missing values) beginning in 1990.
    \cite{NBERw23736} use the single year 1990 as the base year, though I use the four years to ameliorate missing values in the single year of 1990.
    Results are similar in either specification.
}
$Z_{i,t}$ is constructed as negative to reflect the fact that the long term trend in, and most of the short-run shocks to, state funding for higher education have been negative.\footnote{
    When used in log terms, the instrument is the negative of the logged shock --- i.e., $- \log \left( -Z_{i,t} \right)$ as opposed to log of the negative shock $\log Z_{i,t}$ directly.
}
State funding has been falling, so that the instrument describes shocks to university revenues, mostly in a negative direction.\footnote{
    \label{foot:control}
    \cite{NBERw27885} note the tendency for public universities to respond to state funding cuts by increasing reliance on tuition, where \cite{NBERw23736} specifically instruments for tuition revenues with collected information on legislative tuition price controls.
    It may be argued that tuition revenues are confounder between the causal effect of changes in state funding on a university's total revenues, so that this analysis focuses on state support for higher education (not total revenue), as do \cite{NBERw27885}.
    On the other hand, rises in tuition revenues (per student) may arise as result of tuition hikes thanks declining state support, which would mean controlling for tuition would constitute a bad control.
    Nonetheless, estimates including tuition revenues (per student) as a control in the second stage of the IV estimates produces results of very similar magnitude and direction, and so are omitted.
}
The instrument approach relies on the conditional independence assumption, that the instrument is independently assigned to the universities.
While this assumption is fundamentally untestable, we see that universities are by-in-large smaller (in terms of enrolment, total revenues, and professor count) at the top of the state funding shock distribution (see~\autoref{tab:summary-quantiles}).
Though on a per student basis, there is little difference across the distribution (except in the outcomes under consideration).
The state funding shock instrument is positively associated with the total enrolment and total amount of state funding for each university, in both \$ and $\log$/percentage change terms, while other the other sources of finances fopr the university are not associated with the funding shock.
So that the other sources of finances are not clear confounders for the instrumental variables strategy, as they exhibit balance with respect to the funding shock instrument \citep{pei2019poorly}.
Yearly and individual fixed effects are included in regressions throughout to implicitly condition on mean differences between universities and years, leading to the assumption of conditional independence of the instrument with respect to state funding for each public university.

The state funding shock is an instrument the level of state funding for each university in each year.  
The first-stage is then as follows, including institution and year fixed effects, where $X_{i,t}$ represents the amount of state funding divided by the number of full-time students attending the university.
\begin{equation}
    \label{eqn:firststage}
    X_{i,t} = \eta_i + \zeta_t + \delta Z_{i,t} + \epsilon_{i,t}
\end{equation}
We note the conditions for exogeneity in the instrument (following the discussion presented by \citealt{NBERw27885}).
The instrument is exogenous if state policy decisions for funding of public universities are uncorrelated with unobserved institutional changes of any specific college or university in the state \citep{borusyak2022quasi}.
This assumption is plausible given that the majority of states have multiple (i.e., more than five) public universities, without any single university campus receiving the majority of state funding within any single state.
Secondly, the shift-share identification strategy requires exogeneity in either the base-line share or shift component of the instrument.
In this case, we satisfy the second: universities' institutional-level decisions are not correlated with contemporaneous or upcoming shocks to state funding.\footnote{
    It would be plausible to consider the case that universities make institutional-level decisions in a consistently different manner to those 
    with differing reliance on state funding in 1990, so that exogeneity by the base-line share is not plausible here.
}
Lastly, the shift-share approach assumes that state funding shocks affect faculty outcomes only via affecting university finances.

The shift-share instrument has a strong first stage for universities' yearly state funding.
\autoref{tab:firststage-reg} presents results of the first-stage regression, separately with and without a control for tuition revenue per student, plus institution and year fixed effects.
\autoref{tab:firststage-reg} Panel A shows estimates of a funding shock of \$1 per student in the state on state funding per student at the university, and Panel B the \% increase effects of a 1\% increase in the shock per student on state funding per student.
Column (1) shows that a funding shock of \$10 (per student in the entire state) is associated with \$11.76 less state funding per student at the university, with the corresponding -10\% funding shock leads to -9.77\%
less funding (column 1, panel B), with similar estimates with and without including fixed effects.
Column (2) shows estimates of the first-stage without including fixed effects, and gives less precise estimates for the funding shock, likely thanks to systematic differences in universities unaccounted for without fixed effects.
Columns (3) and (4) include the tuition revenue control (explained in \autoref{foot:control}) to exhibit estimates with the inclusion of this possible collider (or bad control).
Column (3) shows similar estimates to column (1) in both Panels A and B thanks to inclusion of fixed effects, so that the uncertainty in including tuition revenue as a possible bad control for the level of state funding does not matter thanks to the inclusion of fixed effects.
Column (1) represents the estimates for \autoref{eqn:firststage} with fixed effects, omitting the tuition revenue control, and is the preferred form that I proceed with.

These results, together with the case for exogeneity, show the funding shock instrument strongly predicts university revenues in the first-stage estimation.


\subsection{Instrumental Variables Model}
\label{sec:iv-model-uni}

I use the instrument defined in \autoref{eqn:public-instrument} to overcome the endogeneity concerns for state funding to each public university, so the primary empirical model is an instrumental variables model.
\autoref{eqn:firststage} is the first stage for exogenous variation in the state funding for university $i$ in year $t$, and \autoref{eqn:secondstage} the second stage for the effect of state funding on faculty outcomes.\footnote{
    It is important to note the treatment effect isolated here; the instrumental variables approach identifies the local average treatment effect, one weighted to level of exposure when treatment is continuous as is the case here.
    So we interpret this treatment effect as a weighted average of effects on faculty at public universities to state funding changes, specific to state funding shocks, among the complier group --- i.e., among universities who respond to funding shocks and would not have made faculty-outcome changes absent the funding shock.
    Also, we assume that no universities state funding increased in response to the negative state funding shock (monotonicity).
}
\begin{equation}
    \label{eqn:secondstage}
    Y_{i,t} = \alpha_i + \gamma_t + \beta \widehat X_{i,t} + \varepsilon_{i,t}
\end{equation}
I estimate the system by two stage least squares, including institution and year fixed effects.\footnote{
    Note that dividing by student count also implicitly controls for the size of the university, so that this model implicitly accounts for yearly variation in professor count and university revenues arising from a university's size of growth/decline.
}
$Y_{i,t}$ represents faculty outcomes for university $i$ in year $t$, $\alpha_i, \gamma_t$ university and year fixed effects, and $\widehat X_{i,t}$ state funding for university $i$ estimated in first stage \eqref{eqn:firststage}.

Additionally, I investigate the effect of changes in state funding among incumbent professors.
Incumbent professors are faculty who are already employed at the university; state funding changes may affect faculty who are already at the university (beyond affecting whether they get hired), so I base the instrument in the year that the professor was hired, and include fixed effects for the hiring year.\footnote{
    This formulation follows that presented by \cite{NBERw27885}, where individual student outcomes are analysed via variation in state funding after their freshman-year.
    This contrasts with \autoref{sec:iv-model-uni} and \cite{NBERw23736}, where the unit of analysis is the university-year, where base year 1990 is more appropriate.
    See \autoref{sec:iv-model-indiv} for the instrument and second-stage specification.
}
The instrument exogeneity and exclusion follows the same argument as above, where the Illinois legislature did not target any single campus for funding cuts and no single Illinois campus takes the majority of state funding.

It is not a priori clear which units are appropriate for this analysis;
does it make sense to consider state funding in purely dollar amounts per student, or in percent change rate?
The funding shock instrument is a strong predictor for the average university's level of state funding in either unit; a funding shock of \$1,000 per student in the entire state leads to around \$1,176 per student at the university, while a funding shock of $-10$\% leads to around 9.77\% less state funding per year.
Yet, the level of state funding (and the outcome variables) vary greatly between states for the unit of analysis.
For example, the average Illinois public university receives \$10,709 in state funding per student in 1990 and  \$6,713 in 2021 (a fall of over 30\%), while California went from \$19,224 per student to \$12,915 in the same time span (a fall of 37\%).
While most states are not exactly the same to California and Illinois, this is example is telling for the phenomenon of stagnating state funding:
states vary in how much the funding higher education in 1990, but most have experienced a decline on order of 30\%, so that the stagnation in state funding has been a percent change trend over this time period.
As such, I include regression specifications where the explanatory and outcome variables are $\log$ transformed, and refer to these when stating results in percent change terms.

\subsection{Effects in Years After the Funding Shock}
\label{sec:empirical-substituion}

The effects of a change in the universities funding may not be immediate, and faculty in may be effected multiple years after a funding shock.
Yet, the funding shock instrument is significantly auto-correlated year-on-year; a large state-level funding shock in year $t$ is also likely to experience a large funding shock in year $t-1$.
Similarly, funding shocks to higher education in year are highly correlated with state funding per student in the 5 years before and after the original shock in IPEDS data (see \autoref{fig:lag-firststage}).
Thus, a linear model correlated state funding in year $t$ with outcomes in year $t$ will suffer from time-series confounding.

I employ a local projections approach to model how faculty outcomes in years $t+k$ are affected by state funding in year $t$, for future years $k > 0$ \citep{jorda2005}.
The local projections method is an empirical model used to estimate dynamic treatment effects when the treatment is not binary, so that time-series confounding (i.e., treatment in time $t$ is correlated with treatment in time $t-1$) is present, even for instrumental variable models \citep{montiel2021local}.
This method estimates the effect of treatment $X_{i,t}$ on outcome $Y_{i, t+ k}$ in follow-on years $k > 0$, while accounting for the auto-correlation between the other years.\footnote{
    The funding shock has a persistent effect on state funding, multiple years after the initial funding shock, so that the instrument is similarly strong for the local projections method (see \autoref{fig:firststage-lp}).
}
Additionally, the approach accomodates use of an instrument for the regressor \citep{olea2021inference}, so that I use the shift-share instument as part of the local projections estimation.
I present estimates in graphical form, where the x-axis represents the year relative to the funding shock (i.e., $x = 0$ represents the year of the state funding shock) and the y-axis represents the estimated effect of the funding shock on that year's outcome.\footnote{
    See \autoref{fig:count-lp} for the graphical format, and the corresponding note for further explanation.
}

% Empirical results
%%%%%%%%%%%%%%%%%%%%%%%%%%%%%%%%%%%%%%%%%
%% Results section
\section{Results}
\label{sec:results}

Changes in state funding have clear effects on the composition of faculty public universities employ.
\autoref{tab:facultycount-shock-reg} presents OLS and IV estimates for the effecet of a change in state funding on the number of faculty at the university, individually by position (and for the total faculty), in both count of professors and in percent terms for faculty per student ratio.
An extra \$1,000 in state funding per student leads to the average university employing 6 fewer lecturers, with no discernible effect on the count of total professors (\autoref{tab:facultycount-shock-reg}, Panel A).
In percentage terms, a 10\% increase in state funding lead to a 4.4\% decrease in the number of lecturers per student, an increase of 14\% for both assistant and full professors, leading to a 0.6\% increase in the total number of faculty per student.
In the state of Illinois, using the count of professors represented in the IBHED databse, state funding is correlated with the number of faculty, but lacks precision to identify the exact effects given the 144 university-year observations in the panel for 2010--2021.

Incumbent professors are relatively unaffected by changes in the state funding for their university in the years following their hire.
\autoref{tab:faculty-shock-illinois-rolling} show estimates of a 1\% change in state funding per student on faculty salaries, rate of exiting the Illinois public university system, and rate of promotion; there are little effects discernible from zero, so that faculty who are already at the university have little of the effects of funding cuts passed on to them.

Faculty composition is not only impacted in the same years as a state funding shock, but also for a number years following.
\autoref{fig:all-count-lp} shows local projection estimates, where the elasticity of total professor count per student in year $t+k$ with respect to state funding in year $t$ is around 0.1, for follow-on years $k = 0, \hdots, 3$.
These findings are in line with the observation that universities froze hiring for a couple of years in response to 2008 negative budget shocks, particularly for tenure-track positions \citep{turner2014impact}, and similarly from shocks to university endowments \citep{brown2014endowment}.
For lecturers, we see a negative effect which lines up with two trends noted in \autoref{sec:trends}: funding for public universities (per student) is decreasing while relative usage of  lecturers increased rapidly.

Decreases in state funding (via funding shocks) affects not only the total counts of professors, but also the composition in the years after thee funding shock.
A 10\% decrease in state funding increases count of lecturers per student by 4\% for the first two years after the initial shock.
While count of assistant professors decreases 10 to 18.5\% three years later;
count of tenured professors decreases 9 to 6\% three years later.
Together, the total count of professors per student decreases around 1\% for three years after the initial funding shock, and the effect is not distinguishable from zero after this point.
\autoref{fig:lecturer-count-lp}, \ref{fig:assistant-count-lp}, \ref{fig:full-count-lp} represents these estimates, using the local projection method, as a series of impulse responses for each outcome of professor count.
These findings show that public universities increase (decrease) their count of tenure-track and tenured professors per student in years when revenues are more (less) plentiful, possibly by increasing hiring intensity or efforts to maintain incumbent professors.
In the same vein, when revenues are bountiful, count of tenure-track and tenured professors per student increases --- and the opposite in years of funding cuts.

\subsection{Rates of Substitution}
\label{sec:results-substituion}

Write here about the log results, including a figure for the university funding elasticities of faculty demand.
Describe here the relevance of the elasticities, and describe the estimates for elasticity of substitution.

% Dicussion
%%%%%%%%%%%%%%%%%%%%%%%%%%%%%%%%%%%%%%%%%
%% Discussion section
\section{Discussion}

The results show that the recent stagnation in state funding for higher education has affected the composition of faculty within public universities, away from tenure-track and tenured professors towards non-tenured lecturers.
At the same time, there are little discernible effects on individual professors hired in the years 2011-2021 at Illinois public universities, except for the salaries of first year lecturers.
The rate that faculty leave their university, and their rate of promotion between positions, are unaffected by the university revenues, so this leaves one primary channel to explain changes in faculty composition: hiring.

\subsection{Faculty Hiring at all US Universities}

Write here about the faculty hiring network data, as provided by \cite{wapman2022quantifying}.

Give a visualisation of the correlation between state funding per student (OLS) and faculty hiring.
Also show the figure of faculty hiring in public vs private: did private unis hire more (in total and per student) over this time period?

Cite Jackson (2021), effects of spending shock.

IV model, where faculty hiring (per student) is the outcome, are presented in \autoref{tab:hiring-shock-reg}.\footnote{
    Give extra details on this specification.
}

\begin{table}[h!]
    \singlespacing
    \centering
    \caption{OLS and 2SLS Estimates for University Faculty Hiring, Total for 2011--2020.}
    \makebox[\textwidth][c]{
\begin{tabular}{@{\extracolsep{5pt}}lcccccc} 
\\[-1.8ex]\hline 
\hline \\[-1.8ex] 
 & \multicolumn{6}{c}{Dependent Variable: Professor Hiring Count} \\ 
\cline{2-7} 
\\[-1.8ex] & \multicolumn{2}{c}{Men} & \multicolumn{2}{c}{Women} & \multicolumn{2}{c}{Total} \\ 
 & OLS & 2SLS & OLS & 2SLS & OLS & 2SLS \\ 
\\[-1.8ex] & (1) & (2) & (3) & (4) & (5) & (6)\\ 
\hline \\[-1.8ex] 
 State Funding & 0.805 & 1.308 & 0.845 & 1.325 & 0.848 & 1.306 \\ 
  & (0.222) & (0.365) & (0.235) & (0.335) & (0.220) & (0.352) \\ 
 \hline \\[-1.8ex] 
Observations & 157 & 157 & 157 & 157 & 157 & 157 \\ 
R$^{2}$ & 0.396 & 0.366 & 0.415 & 0.383 & 0.408 & 0.381 \\ 
\hline 
\hline \\[-1.8ex] 
\end{tabular} 
}
    \begin{flushleft}
        \footnotesize
        \textbf{Note}: Standard errors are clustered at the state level.
    \end{flushleft}
    \label{tab:hiring-shock-reg}
\end{table}

Note that yearly variation is not observed here, so that only aggregate level, for 180 universities, can be considered.
Leave it to further research to delve deeper into the role that hiring plays as the primary causal mediator for the effect of state funding shocks on faculty composition.

\subsection{Faculty Hiring at Illinois Public Universities}

\cite{turner2014impact} documents the wide-spread practice of hiring freezes at universities in response to budget shocks around the 2008 recession.
Throughout the last decade, multiple such measures were taken by Illinois public universities in response to their deteriorating finances \citep{furlough2010}.
The University of Illinois\footnote{
    The University of Illinois includes three public university campuses: Urbana-Champaign, Chicago, Springfield.
}
did not receive the allocated state appropriations from the state of Illinois on time, so enacted cost-cutting measures to stay fiscally solvent.
The university system placed a hold on all hiring for filling state-funded positions and promotions.
Faculty were furloughed (placed on leave without pay) for a day each month, and university administrators placed on two days per month furlough, or eligible university employees could accept a voluntary, equivalent pay reduction.
The University of Western Illinois adopted very similar measures in response to the Illinois budget crisis in 2016-2017 \citep{wiu2016}.

\begin{figure}[h!]
    \centering
    \singlespacing
    \caption{Trends in New Hires at Illinois Public Universities 2011-2021.}
    \begin{subfigure}[b]{0.495\textwidth}
        \centering
        \caption{New Hire Count, Total.}
        \includegraphics[width=\textwidth]{figures/newhire-count-illinois.png}
        \label{fig:newhire-count-illinois}
    \end{subfigure}
    \begin{subfigure}[b]{0.495\textwidth}
        \centering
        \caption{Mean Salary in First Year for New Hires, \$ 2021 CPI-U.}
        \includegraphics[width=\textwidth]{figures/newhire-salary-illinois.png}
        \label{fig:newhire-salary-illinois}
    \end{subfigure}
    \label{fig:newhire-illinois}
\end{figure}

These examples show that Illinois public universities were affected by the negative shock to state funding for higher education, and instituted policy changes aimed at affecting faculty outcomes investigated in \autoref{sec:results}.
Yet the only measurable effect was on count of professor per student (implied via slower hiring), and first-year lecturer salaries.
To investigate these trends for the entire state, \autoref{fig:newhire-illinois} shows the count of professors hired in the years 2011-2021, and their mean salary, by position.
In particular, no position had sustained rises in hiring while enrolment within the university was rising (so that new hires per student necessarily fell), and hiring of new administrators jumped once the Illinois budget crisis ended in 2017.
At the same time, professor new hires' starting salaries remained relatively constant for lecturer and assistant professors, yet rose and fell over the decade for full professors and administrators.

% Conclusion
%%%%%%%%%%%%%%%%%%%%%%%%%%%%%%%%%%%%%%%%%
%% Conclusion section
\section{Summary and Concluding Remarks}
\label{sec:conclusion}

This analysis investigates how the recent stagnation in state support for higher education has affected faculty, and their composition, at US public universities.
This work contributes to the literature along two primary dimensions.
Firstly, by isolating changes in state funding on public universities via state funding shocks, this work provides an explanation for the increased relaiance on lecturers and away from tenure-track and tenured professors at US public universities.
This approach used multiple methods to estimate the short- and medium-run effects of state funding cuts on faculty outcomes.
Secondly, this work asks investigates individual faculty are affected by changes in state funding for their university using a dataset new in the economic literature (IBHED).
These data allow for detailed analysis of thousands of faculty salaries, and employment outcomes in the Illinois university system.

I found that public universities have systematically substituted away from tenure-track and tenured professors, towards lecturers, in the face of persistent declines in state funding for higher education.
The effects on incumbent professors in the state of Illinois are non-distinguishable from zero, while suggestive information shows falls in hiring, which together implies that the changes in faculty composition are driven by reduced hiring of new professors.
Public universities are using more contingent lecturers to teach their students, while private universities continue to employ more tenure-track and tenured professors than their public counterparts, and each year the gap widens.

While costs education have been rising in the US, public universities have also dealt with declining state funding.
I show that these headwinds led to systematic change at public universities, changes that affect their faculty, likely limiting public universities in their goals in research and education.
These results show large changes in faculty composition, and that stagnation in state support explains at least a third of the observed shift away from tenured professors and towards contingent lecturers.
At the same time, private universities were not exposed to financial headwinds of the same magnitude or persistence.
While public universities continue to educate the majority of higher education students in the US, we should worry about the effects of restricting their funding has on faculty composition, research at public universities, and the impact on higher education as a whole.


% Bibliography
\singlespacing
\bibliographystyle{agsm}
\bibliography{sections/08-bibliography.bib}

% Separated Figures \& Tables
\newpage
%%%%%%%%%%%%%%%%%%%%%%%%%%%%%%%%%%%%%%%%%
%% Section to host Figures
\section{Figures}
\label{sec:figures}

\begin{figure}[H]
    \centering
    \singlespacing
    \caption{Mean Total Funding among Public Universities, by Year.}
    \begin{subfigure}[b]{0.495\textwidth}
        \centering
        \caption{Total, \$ 2021 CPI-U.}
        \includegraphics[width=\textwidth]{figures/mean-funding-total.png}
        \label{fig:mean-funding-total}
    \end{subfigure}
    \begin{subfigure}[b]{0.495\textwidth}
        \centering
        \caption{Per Student, \$ 2021 CPI-U.}
        \includegraphics[width=\textwidth]{figures/mean-funding-fte.png}
        \label{fig:mean-funding-fte}
    \end{subfigure}
    \label{fig:funding}
    \justify
    \footnotesize
    \textbf{Note}:
    This figure shows the mean funding for US public universities as a total (in figure a.) and divided by student enrolment (figure b.).
    The numbers are adjusted to 2021 figures by CPI-U.
    Non-institutional revenues refers to the sum of federal, state, and local funding plus tuition revenues; these sum to the majority of university funding, but exclude numbers such as university income from capital projects.
    These figures are calculated in IPEDS data.
\end{figure}

\begin{figure}[H]
    \centering
    \singlespacing
    \caption{Total Student enrolment, by University Sector and Year.}
    \begin{subfigure}[b]{0.495\textwidth}
        \centering
        \caption{Total Enrolment, Nation-wide.}
        \includegraphics[width=\textwidth]{figures/enrollment-total.png}
        \label{fig:enrollment-total}
    \end{subfigure}
    \begin{subfigure}[b]{0.495\textwidth}
        \centering
        \caption{Mean Enrolment, per University.}
        \includegraphics[width=\textwidth]{figures/enrollment-mean.png}
        \label{fig:enrollment-mean}
    \end{subfigure}
    \label{fig:enrolment}
    \justify
    \footnotesize
    \textbf{Note}:
    This figure shows the total and mean enrolment for US universities, comparing public and private universities.
    Most of the higher education enrolment increase for the last 30 years was in public universities, who continue to enrol the vast majority of higher education students in the US.
    These figures are calculated in IPEDS data.
\end{figure}

\newpage
\begin{figure}[H]
    \centering
    \singlespacing
    \caption{Trends in Mean Student Enrolment per Professor, by University Sector and Faculty level.}
    \begin{subfigure}[b]{0.495\textwidth}
        \centering
        \caption{Lecturer.}
        \includegraphics[width=\textwidth]{figures/lecturer-fte-perprof.png}
        \label{fig:lecturer-fte-perprof}
    \end{subfigure}
    \begin{subfigure}[b]{0.495\textwidth}
        \centering
        \caption{Assistant.}
        \includegraphics[width=\textwidth]{figures/assistant-fte-perprof.png}
        \label{fig:assistant-fte-perprof}
    \end{subfigure}
    \begin{subfigure}[b]{0.495\textwidth}
        \centering
        \caption{Full.}
        \includegraphics[width=\textwidth]{figures/full-fte-perprof.png}
        \label{fig:full-fte-perprof}
    \end{subfigure}
    \begin{subfigure}[b]{0.495\textwidth}
        \centering
        \caption{All.}
        \includegraphics[width=\textwidth]{figures/all-fte-perprof.png}
        \label{fig:all-fte-perprof}
    \end{subfigure}
    \label{fig:fte-perprof}

    \justify
    \footnotesize
    \textbf{Note}:
    This figure shows the average number of students per faculty member, by different faculty position, at US public universities.
    E.g., panel A calculates the mean of (lecturer count) / (student enrolment) at US public universities, for each year of 1990--2017, to show the average trend in faculty composition.
    These figures are calculated in IPEDS data.
\end{figure}

\newpage
\begin{figure}[H]
    \centering
    \singlespacing
    \caption{Mean Funding Sources among Illinois Public Universities, by Year.}
    \begin{subfigure}[b]{0.495\textwidth}
        \centering
        \caption{Total, \$ 2021 CPI-U.}
        \includegraphics[width=\textwidth]{figures/illinois-funding-total.png}
        \label{fig:illinois-funding-total}
    \end{subfigure}
    \begin{subfigure}[b]{0.495\textwidth}
        \centering
        \caption{Per Student, \$ 2021 CPI-U.}
        \includegraphics[width=\textwidth]{figures/illinois-funding-fte.png}
        \label{fig:illinois-funding-fte}
    \end{subfigure}
    \label{fig:illinois-funding}

    \justify
    \footnotesize
    \textbf{Note}:       
    This figure shows the mean funding for Illinois public universities as a total (in figure a.) and divided by student enrolment (figure b.).
    The numbers are adjusted to 2021 figures by CPI-U.
    Non-institutional revenues refers to the sum of federal, state, and local funding plus tuition revenues; these sum to the majority of university funding, but exclude numbers such as university income from capital projects.
    These figures are calculated in IPEDS data.
\end{figure}

\newpage
\begin{figure}[H]
    \centering
    \singlespacing
    \caption{Local Projection Estimates for Effect of State Funding on Faculty Count per Student at Universities, by Professor Group.}
    \begin{subfigure}[b]{0.495\textwidth}
        \centering
        \caption{Lecturers.}
        \includegraphics[width=\textwidth]{figures/lecturer-count-lp.png}
        \label{fig:lecturer-count-lp}
    \end{subfigure}
    \begin{subfigure}[b]{0.495\textwidth}
        \centering
        \caption{Assistant Professors.}
        \includegraphics[width=\textwidth]{figures/assistant-count-lp.png}
        \label{fig:assistant-count-lp}
    \end{subfigure}
    \begin{subfigure}[b]{0.495\textwidth}
        \centering
        \caption{Full Professors.}
        \includegraphics[width=\textwidth]{figures/full-count-lp.png}
        \label{fig:full-count-lp}
    \end{subfigure}
    \begin{subfigure}[b]{0.495\textwidth}
        \centering
        \caption{All Professors.}
        \includegraphics[width=\textwidth]{figures/all-count-lp.png}
        \label{fig:all-count-lp}
    \end{subfigure}
    \label{fig:count-lp}
    \justify
    \footnotesize
    \textbf{Note}:
    These figures show the local projections estimates of regression specification \eqref{eqn:secondstage}, with the funding shock as an instrument for state funding
    The unit of analysis is the university, and uses IPEDS data.
    The coefficient estimate is effect of state funding ($X_{i,t}$) on faculty count per student ($Y_{i,t}$), while accounting for auto-correlation between different time periods --- i.e., between $X_{i,t}, X_{i,t-1}$ and $Y_{i,t}, Y_{i,t-1}$.
    These results use a $\log-\log$ specification, so the estimates are for the elasticity of professor count per student in a year $t+k$ with respect to state funding in year $t$, where years $k = 0, \hdots, 10$ are on the x-axis. 
    Standard errors are clustered at the state-year level.
\end{figure}

\newpage
\begin{figure}[H]
    \centering
    \singlespacing
    \caption{Local Projection Estimates for Effect of State Funding on Faculty Salaries per Student at Illinois Public Universities, by Professor Group.}
    \begin{subfigure}[b]{0.495\textwidth}
        \centering
        \caption{Lecturers.}
        \includegraphics[width=\textwidth]{figures/salaries-lecturer-illinois-lp-rolling.png}
        \label{fig:salaries-lecturer-illinois-lp-rolling}
    \end{subfigure}
    \begin{subfigure}[b]{0.495\textwidth}
        \centering
        \caption{Assistant Professors.}
        \includegraphics[width=\textwidth]{figures/salaries-assistant-illinois-lp-rolling.png}
        \label{fig:salaries-assistant-illinois-lp-rolling}
    \end{subfigure}
    \begin{subfigure}[b]{0.495\textwidth}
        \centering
        \caption{Full Professors.}
        \includegraphics[width=\textwidth]{figures/salaries-full-illinois-lp-rolling.png}
        \label{fig:salaries-full-illinois-lp-rolling}
    \end{subfigure}
    \begin{subfigure}[b]{0.495\textwidth}
        \centering
        \caption{Administrator Professors.}
        \includegraphics[width=\textwidth]{figures/salaries-administrator-illinois-lp-rolling.png}
        \label{fig:salaries-administrator-illinois-lp-rolling}
    \end{subfigure}
    \label{fig:salaries-illinois-lp-rolling}
    \justify
    \footnotesize
    \textbf{Note}:
    These figures show the local projections estimates of regression specification \eqref{eqn:secondstage}, with the funding shock as an instrument for state funding.
    The unit of analysis is an individual faculty member (at an Illinois public university); funding data come from IPEDS, and faculty salaries from IBHED.
    The coefficient estimate is effect of state funding ($X_{i(j),t}$) on faculty salaries ($Y_{j,t}$), while accounting for auto-correlation between different time periods --- i.e., between $X_{i(j),t}, X_{i(j),t-1}$ and $Y_{i(j),t}, Y_{i(j),t-1}$.
    These results use a $\log-\log$ specification, so the estimates are for the elasticity of faculty salaries student in a year $t+k$ with respect to state funding in year $t$, where years $k = 0, \hdots, 4$ are on the x-axis. 
    Standard errors are clustered at the university-year level.
\end{figure}


%%%%%%%%%%%%%%%%%%%%%%%%%%%%%%%%%%%%%%%%%
%% Section to host Tables
\newpage
\section{Tables}
\label{sec:tables}

\begin{table}[H]
    \singlespacing
    %\centering
    \caption{IPEDS Summary Statistics, Public Universities Panel 1990--2021}
    \makebox[\textwidth][c]{
\begin{tabular}{@{\extracolsep{5pt}}lccc} 
\\[-1.8ex]\hline 
\hline \\[-1.8ex] 
Statistic & \multicolumn{1}{c}{Mean} & \multicolumn{1}{c}{St. Dev.} & \multicolumn{1}{c}{N} \\ 
\hline \\[-1.8ex] 
Enrolment & 11,511 & 10,821 & 18,504 \\ 
State appropriations (millions 2021 USD) & 99 & 125 & 18,504 \\ 
Total revenues (millions 2021 USD) & 425 & 793 & 18,504 \\ 
Non-institutional revenues (millions 2021 USD) & 202 & 265 & 18,504 \\ 
Lecturers count & 60 & 74 & 17,329 \\ 
Assistant professors count & 113 & 102 & 17,826 \\ 
Full professors count & 261 & 284 & 17,929 \\ 
All professors count & 429 & 437 & 18,504 \\ 
\hline \\[-1.8ex] 
\end{tabular} 
}
    \label{tab:ipeds-summary}
    \justify
    \footnotesize
    \textbf{Note}:
    This table shows the summary statistics for every public university--year observation in IPEDS data.
    The numbers are adjusted to 2021 figures by CPI-U.
    Non-institutional revenues refers to the sum of federal, state, and local funding plus tuition revenues; these sum to the majority of university funding, but exclude numbers such as university income from capital projects.
\end{table}

\begin{table}[H]
    \singlespacing
    \centering
    \caption{IBHED Summary Statistics, Professor Panel 2010--2021.}
    \makebox[\textwidth][c]{
\begin{tabular}{@{\extracolsep{5pt}}lcccc} 
\\[-1.8ex]\hline 
\hline \\[-1.8ex] 
Statistic & \multicolumn{1}{c}{Mean} & \multicolumn{1}{c}{St. Dev.} & \multicolumn{1}{c}{Median} & \multicolumn{1}{c}{N} \\ 
\hline \\[-1.8ex] 
Lecturer, percent & 27 & 44 & 0 & 185,570 \\ 
Assistant professor, percent & 21 & 41 & 0 & 185,570 \\ 
Full professor, percent & 37 & 48 & 0 & 185,570 \\ 
Administrator professor, percent & 15 & 36 & 0 & 185,570 \\ 
Lecturer salary (2021 USD) & 31,650 & 25,825 & 27,474 & 49,637 \\ 
Assistant salary (2021 USD) & 77,075 & 38,078 & 73,387 & 39,051 \\ 
Full salary (2021 USD) & 109,535 & 48,949 & 100,044 & 68,243 \\ 
Administrator salary (2021 USD) & 119,462 & 61,377 & 107,824 & 28,639 \\ 
All salary (2021 USD) & 83,403 & 55,881 & 78,969 & 185,570 \\ 
Lecturer benefits (2021 USD) & 2,351 & 6,458 & 0 & 49,637 \\ 
Assistant benefits (2021 USD) & 2,964 & 7,092 & 0 & 39,051 \\ 
Full benefits (2021 USD) & 6,736 & 13,654 & 0 & 68,243 \\ 
Administrator benefits (2021 USD) & 3,607 & 15,976 & 0 & 28,639 \\ 
All benefits (2021 USD) & 4,286 & 11,547 & 0 & 185,570 \\ 
\hline \\[-1.8ex] 
\end{tabular} 
}
    \label{tab:illinois-summary}

    \justify
    \footnotesize
    \textbf{Note}:
    This table shows the summary statistics for every faculty--year observation in the IBHED data, which represents every faculty member in the Illinois public university system over years 2010--2021.
    The numbers are adjusted to 2021 figures by CPI-U.
    Lecturer is a binary for whether the faculty member is designated as a lecturer in the databse, and similarly for the assistant professors, full professors, and administrative faculty;
    salary refers the sum of base salary and benefits.
    All salary and benefits refers to summary statistics on the salary and benefits, respectively, of all faculty (regardless of position).
\end{table}

\newpage
\begin{table}[h!]
    \singlespacing
    \centering
    \caption{First Stage Estimates, for State Funding by funding shock.}
    \textbf{Panel A: units in \$ per student}
    
    \makebox[\textwidth][c]{
\begin{tabular}{@{\extracolsep{5pt}}lcccc} 
\\[-1.8ex]\hline 
\hline \\[-1.8ex] 
 & \multicolumn{4}{c}{Dependent Variable: State Funding} \\ 
\cline{2-5} 
\\[-1.8ex] & (1) & (2) & (3) & (4)\\ 
\hline \\[-1.8ex] 
 Funding Shift-Share & $-$1.176 & $-$0.160 & $-$1.100 & $-$1.071 \\ 
  & (0.226) & (0.265) & (0.242) & (0.264) \\ 
  Tuition Revenue &  &  & $-$0.295 & 1.012 \\ 
  &  &  & (0.136) & (0.329) \\ 
  Constant &  & 9,716.437 &  & $-$1,708.334 \\ 
  &  & (1,805.394) &  & (2,716.150) \\ 
 \hline \\[-1.8ex] 
Uni. + Year fixed effects? & Yes & No & Yes & No \\ 
F stat. & 20.712 & 16.512 & 26.999 & 0.365 \\ 
Observations & 17,012 & 17,012 & 17,012 & 17,012 \\ 
R$^{2}$ & 0.918 & 0.0004 & 0.919 & 0.074 \\ 
\hline 
\hline \\[-1.8ex] 
\end{tabular} 
}
    
    \textbf{Panel B: units in log \$ per student}
    
    \makebox[\textwidth][c]{
\begin{tabular}{@{\extracolsep{5pt}}lcccc} 
\\[-1.8ex]\hline 
\hline \\[-1.8ex] 
 & \multicolumn{4}{c}{Dependent Variable: State Funding} \\ 
\cline{2-5} 
\\[-1.8ex] & (1) & (2) & (3) & (4)\\ 
\hline \\[-1.8ex] 
 Funding Shift-share& $-$0.977 & $-$0.302 & $-$0.986 & $-$0.573 \\ 
  & (0.066) & (0.093) & (0.062) & (0.067) \\ 
  Tuition Revenue &  &  & 0.058 & 0.535 \\ 
  &  &  & (0.059) & (0.065) \\ 
  Constant &  & 6.419 &  & $-$0.484 \\ 
  &  & (0.769) &  & (0.844) \\ 
 \hline \\[-1.8ex] 
Uni. + Year fixed effects? & Yes & No & Yes & No \\ 
F stat. & 249.662 & 74.022 & 218.171 & 10.558 \\ 
Observations & 17,012 & 17,012 & 17,012 & 17,012 \\ 
R$^{2}$ & 0.790 & 0.047 & 0.790 & 0.180 \\ 
\hline 
\hline \\[-1.8ex] 
\end{tabular} 
}
    
    \label{tab:firststage-reg}
    \justify
    \footnotesize
    \textbf{Note}:
    These tables show the first stage OLS estimates of regression specification \eqref{eqn:firststage}, showing the effect of the funding shock on state funding to gauge performance as an instrument.
    Each observation is a public univeristy-year, in the IPEDS data.
    Panel A shows the effect of an funding shock of \$-1 per student in the state on the number of \$'s of state funding per student at the university --- i.e.,
    \$-1 funding shock per student in the state leads to \$1.176 less state funding per student at the university according to preferred specification column 1.
    Panel B shows the effect of a $-10$\% change funding shock per student in the state on $10$\% change in state funding per student at the university --- i.e.,
    $-10$\% funding shock per student in the state leads to $-9.77$\% less state funding per student at the university according to prefferred specification column 1.        
    Standard errors are clustered at the state-year level, and univeristy $+$ year fixed effects are included where noted.
\end{table}

\newpage
\begin{table}[H]
    \singlespacing
    \centering
    \caption{Shift-Share Instrument Balance Test, in IPEDS 1990--2017.}

    \textbf{Panel A: units in \$}

    \makebox[\textwidth][c]{
\begin{tabular}{@{\extracolsep{5pt}}lcccccccc} 
\\[-1.8ex]\hline 
\hline \\[-1.8ex] 
 & \multicolumn{8}{c}{Dependent Variables: University Characteristics} \\ 
\cline{2-9} 
\\[-1.8ex] & Enrolment & State Funding & Total revenues & Non-inst. revenues & Lecturers & Assistant professors & Full professors & All professors \\ 
\\[-1.8ex] & (1) & (2) & (3) & (4) & (5) & (6) & (7) & (8)\\ 
\hline \\[-1.8ex] 
 State Funding & 0.517 & $-$9,217.091 & $-$7,633.377 & 1,818.090 & 0.007 & $-$0.001 & $-$0.003 & 0.004 \\ 
  & (0.199) & (2,451.719) & (12,771.130) & (4,774.483) & (0.002) & (0.001) & (0.003) & (0.005) \\ 
 \hline \\[-1.8ex] 
Uni. + Year fixed effects? & Yes & Yes & Yes & Yes & Yes & Yes & Yes & Yes \\ 
Observations & 17,012 & 17,012 & 17,012 & 17,012 & 17,012 & 17,012 & 17,012 & 17,012 \\ 
R$^{2}$ & 0.960 & 0.950 & 0.881 & 0.922 & 0.745 & 0.886 & 0.973 & 0.954 \\ 
\hline 
\hline \\[-1.8ex] 
\end{tabular} 
}
    
    \textbf{Panel B: units in log \$ per student}
    
    \makebox[\textwidth][c]{
\begin{tabular}{@{\extracolsep{5pt}}lcccccccc} 
\\[-1.8ex]\hline 
\hline \\[-1.8ex] 
 & \multicolumn{8}{c}{Dependent Variables: University Characteristics} \\ 
\cline{2-9} 
\\[-1.8ex] & Enrolment & State Funding & Total revenues & Non-inst. revenues & Lecturers & Assistant professors & Full professors & All faculty \\ 
\\[-1.8ex] & (1) & (2) & (3) & (4) & (5) & (6) & (7) & (8)\\ 
\hline \\[-1.8ex] 
 State Funding & 0.107 & $-$0.870 & 0.031 & $-$0.081 & 0.534 & $-$0.024 & $-$0.027 & 0.043 \\ 
  & (0.037) & (0.079) & (0.082) & (0.065) & (0.122) & (0.066) & (0.034) & (0.038) \\ 
 \hline \\[-1.8ex] 
Uni. + Year fixed effects? & Yes & Yes & Yes & Yes & Yes & Yes & Yes & Yes \\ 
Observations & 17,012 & 17,012 & 17,012 & 17,012 & 17,012 & 17,012 & 17,012 & 17,012 \\ 
R$^{2}$ & 0.976 & 0.914 & 0.979 & 0.975 & 0.782 & 0.902 & 0.963 & 0.965 \\ 
\hline 
\hline \\[-1.8ex] 
\end{tabular} 
}

    \justify
    \footnotesize
    \textbf{Note}:
    These tables show the OLS estimates of regression specification \eqref{eqn:firststage}, showing the effect of the funding shock on other outcomes to gauge balance of the instrument with respect to university characteristics
    ENrolment is measured in thousands of students.
    Each observation is a public univeristy-year, in the IPEDS data.
    Panel A shows the effect of an funding shock of \$-1,000 per student in the state on the counts of the outcome --- i.e.,
    \$-1,000 funding shock per student in the state is associated with 517 more students (as enrolment is measured in thousands) in column 1.
    Panel B shows the effect of a $-10$\% change funding shock per student in the state on $10$\% change in the outcomes --- i.e.,
    $-10$\% funding shock per student in the state is associated with an increase in 1.07\% more students at the unviersity according to prefferred column 1.        
    Standard errors are clustered at the state-year level, and univeristy $+$ year fixed effects are included where noted.
    \label{tab:facultycount-shock-reg}
\end{table}

\newpage
\begin{table}[H]
    \singlespacing
    \centering
    \caption{Effects of Changes in State Funding on University Faculty Composition, IPEDS 1990--2017, OLS and 2SLS Estimates.}

    \textbf{Panel A: units in \$ per student}

    \makebox[\textwidth][c]{
\begin{tabular}{@{\extracolsep{5pt}}lcccccccc} 
\\[-1.8ex]\hline 
\hline \\[-1.8ex] 
 & \multicolumn{8}{c}{Dependent Variable: Faculty Count per 100 Students, by Position} \\ 
\cline{2-9} 
\\[-1.8ex] & \multicolumn{2}{c}{Lecturers} & \multicolumn{2}{c}{Asst. Professors} & \multicolumn{2}{c}{Full Professors} & \multicolumn{2}{c}{All Faculty} \\ 
 & OLS & 2SLS & OLS & 2SLS & OLS & 2SLS & OLS & 2SLS \\ 
\\[-1.8ex] & (1) & (2) & (3) & (4) & (5) & (6) & (7) & (8)\\ 
\hline \\[-1.8ex] 
 State Funding & $-$0.451 & $-$5.957 & $-$0.479 & 1.031 & $-$0.104 & 2.288 & $-$1.198 & $-$3.333 \\ 
  & (0.177) & (1.725) & (0.211) & (1.232) & (0.275) & (2.910) & (0.612) & (4.259) \\ 
 \hline \\[-1.8ex] 
Outcome Mean & 59.253 & 59.253 & 116.121 & 116.121 & 269.103 & 269.103 & 452.507 & 452.507 \\ 
Observations & 17,012 & 17,012 & 17,012 & 17,012 & 17,012 & 17,012 & 17,012 & 17,012 \\ 
R$^{2}$ & 0.742 & 0.595 & 0.887 & 0.881 & 0.973 & 0.971 & 0.954 & 0.954 \\ 
\hline 
\hline \\[-1.8ex] 
\end{tabular} 
}
    
    \textbf{Panel B: units in log \$ per student}
    
    \makebox[\textwidth][c]{
\begin{tabular}{@{\extracolsep{5pt}}lcccccccc} 
\\[-1.8ex]\hline 
\hline \\[-1.8ex] 
 & \multicolumn{8}{c}{Dependent Variable: Employment Count by Professor Group} \\ 
\cline{2-9} 
\\[-1.8ex] & \multicolumn{2}{c}{Lecturer} & \multicolumn{2}{c}{Assistant} & \multicolumn{2}{c}{Full} & \multicolumn{2}{c}{All} \\ 
 & OLS & 2SLS & OLS & 2SLS & OLS & 2SLS & OLS & 2SLS \\ 
\\[-1.8ex] & (1) & (2) & (3) & (4) & (5) & (6) & (7) & (8)\\ 
\hline \\[-1.8ex] 
 Non-inst. Revenues & $-$0.059 & $-$1.426 & 0.348 & 0.659 & 0.453 & 0.530 & 0.360 & 0.310 \\ 
  & (0.162) & (0.590) & (0.126) & (0.207) & (0.154) & (0.134) & (0.144) & (0.100) \\ 
  Tuition Revenue & 0.432 & 0.951 & 0.091 & $-$0.028 & $-$0.060 & $-$0.090 & 0.040 & 0.059 \\ 
  & (0.103) & (0.234) & (0.098) & (0.101) & (0.080) & (0.065) & (0.061) & (0.053) \\ 
 \hline \\[-1.8ex] 
Observations & 17,329 & 17,329 & 17,826 & 17,826 & 17,929 & 17,929 & 18,504 & 18,504 \\ 
R$^{2}$ & 0.673 & 0.631 & 0.720 & 0.712 & 0.794 & 0.793 & 0.827 & 0.827 \\ 
\hline 
\hline \\[-1.8ex] 
\end{tabular} 
}

    \label{tab:facultycount-shock-reg}
    \justify
    \footnotesize
    \textbf{Note}:
    These tables show the second stage OLS and 2SLS estimates of regression specification \eqref{eqn:secondstage}, showing the effect of state funding changes on faculty outcomes, using the funding shock to instrument for state funding in the columns labelled 2SLS.
    Each observation is a public university-year, in the IPEDS data.
    Panel A shows the effect of a fall in state funding \$-1,000 per student in the state on the number of professors --- i.e.,
    an extra \$1,000 per student leads to 6 fewer lecturers according to column 2.
    Panel B shows the effect of a $10$\% change in state funding per student at the university on the 10\% change in the number of professors per students --- i.e.,
    an extra 10\% of state funding per student leads to 4.37\% fewer lecturers per student according to column 2.
    Outcome-mean is the mean of the outcome, for Panel A the number of professors per student, for Panel B the number of faculty per student.
    Panel B uses $\log$ faculty count per student as the outcome, though the outcome mean is count of faculty per student (not in $\log$ terms).
    Standard errors are clustered at the state-year level, and university $+$ year fixed effects are included through--out.
\end{table}


\newpage
\begin{table}[H]
    \singlespacing
    \centering
    \caption{Effects of Changes in State Funding on University Faculty Composition, in Illinois 2010--2021, OLS and 2SLS Estimates.}

    \textbf{Panel A: units in \$ per student}

    \makebox[\textwidth][c]{
\begin{tabular}{@{\extracolsep{5pt}}lcccccccc} 
\\[-1.8ex]\hline 
\hline \\[-1.8ex] 
 & \multicolumn{8}{c}{Dependent Variable: Employment Count by Professor Group} \\ 
\cline{2-9} 
\\[-1.8ex] & \multicolumn{2}{c}{Lecturers} & \multicolumn{2}{c}{Asst. Professors} & \multicolumn{2}{c}{Full Professors} & \multicolumn{2}{c}{All Faculty} \\ 
 & OLS & 2SLS & OLS & 2SLS & OLS & 2SLS & OLS & 2SLS \\ 
\\[-1.8ex] & (1) & (2) & (3) & (4) & (5) & (6) & (7) & (8)\\ 
\hline \\[-1.8ex] 
 State Funding & $-$8.399 & $-$32.989 & $-$1.981 & $-$1.491 & $-$2.567 & $-$2.012 & $-$12.508 & 27.514 \\ 
  & (4.661) & (31.616) & (2.194) & (22.859) & (2.799) & (10.909) & (9.620) & (84.320) \\ 
 \hline \\[-1.8ex] 
Outcome Mean & 351.306 & 351.306 & 273.757 & 273.757 & 477.597 & 477.597 & 1303.014 & 1303.014 \\ 
Observations & 144 & 144 & 144 & 144 & 144 & 144 & 144 & 144 \\ 
R$^{2}$ & 0.893 & 0.843 & 0.973 & 0.973 & 0.992 & 0.992 & 0.983 & 0.978 \\ 
\hline 
\hline \\[-1.8ex] 
\end{tabular} 
}
    
    \textbf{Panel B: units in log \$ per student}
    
    \makebox[\textwidth][c]{
\begin{tabular}{@{\extracolsep{5pt}}lcccccccc} 
\\[-1.8ex]\hline 
\hline \\[-1.8ex] 
 & \multicolumn{8}{c}{Dependent Variable: Employment Count by Professor Group} \\ 
\cline{2-9} 
\\[-1.8ex] & \multicolumn{2}{c}{Lecturer} & \multicolumn{2}{c}{Assistant} & \multicolumn{2}{c}{Full} & \multicolumn{2}{c}{All} \\ 
 & OLS & 2SLS & OLS & 2SLS & OLS & 2SLS & OLS & 2SLS \\ 
\\[-1.8ex] & (1) & (2) & (3) & (4) & (5) & (6) & (7) & (8)\\ 
\hline \\[-1.8ex] 
 State Funding & $-$0.015 & $-$0.017 & 0.049 & 0.058 & 0.007 & $-$0.029 & 0.004 & $-$0.008 \\ 
  & (0.033) & (0.026) & (0.045) & (0.038) & (0.021) & (0.032) & (0.028) & (0.022) \\ 
 \hline \\[-1.8ex] 
Outcome Mean & 2.523 & 2.523 & 1.491 & 1.491 & 2.729 & 2.729 & 8.141 & 8.141 \\ 
Observations & 144 & 144 & 144 & 144 & 144 & 144 & 144 & 144 \\ 
R$^{2}$ & 0.700 & 0.700 & 0.789 & 0.789 & 0.839 & 0.836 & 0.633 & 0.632 \\ 
\hline 
\hline \\[-1.8ex] 
\end{tabular} 
}

    \label{tab:facultycount-illinois-reg}
    \justify
    \footnotesize
    \textbf{Note}:
    These tables show the second stage OLS and 2SLS estimates of regression specification \eqref{eqn:secondstage}, showing the effect of state funding changes on number of faculty per student in Illinois universities, using the funding shock to instrument for state funding in the columns labelled 2SLS.
    Each observation is a public university-year in the state of Illinois, where funding data come from IPEDS and faculty count come from IBHED data.
    Panel A shows the effect of a fall in state funding \$-1,000 per student in the state on the number of professors.
    Panel B shows the effect of a $10$\% change in state funding per student at the university on the 10\% change in the number of professors per students.
    Outcome-mean is the mean of the outcome, for Panel A the number of professors per student, for Panel B the number of faculty per student.
    Panel B uses $\log$ faculty count per student as the outcome, though the outcome mean is count of faculty per student (not in $\log$ terms).
    Standard errors are clustered at the university-year level, and university $+$ year fixed effects are included through--out.
\end{table}

\newpage
\begin{table}[H]
    \singlespacing
    \centering
    \caption{
        Effects of Changes in State Funding on Faculty Salaries and Exit Rate, in Illinois 2010--2021, 2SLS Estimates.}
    \makebox[\textwidth][c]{
\begin{tabular}{@{\extracolsep{5pt}}lccccc} 
\\[-1.8ex]\hline 
\hline \\[-1.8ex] 
 & \multicolumn{5}{c}{Dependent Variable: Salaries by Professor Group} \\ 
\cline{2-6} 
 & Lecturer & Assistant & Full & Admin & All \\ 
\\[-1.8ex] & (1) & (2) & (3) & (4) & (5)\\ 
\hline \\[-1.8ex] 
 Non-inst. Revenues & 0.009 & $-$0.072 & $-$0.044 & $-$0.013 & $-$0.011 \\ 
  & (0.094) & (0.040) & (0.038) & (0.049) & (0.086) \\ 
 \hline \\[-1.8ex] 
Observations & 25,820 & 22,156 & 9,001 & 11,472 & 68,449 \\ 
R$^{2}$ & 0.217 & 0.051 & 0.074 & 0.143 & 0.161 \\ 
\hline 
\hline \\[-1.8ex] 
\end{tabular} 
}
    
    \makebox[\textwidth][c]{
\begin{tabular}{@{\extracolsep{5pt}}lccccc} 
\\[-1.8ex]\hline 
\hline \\[-1.8ex] 
 & \multicolumn{5}{c}{Dependent Variable: Exit rate by Professor Group} \\ 
\cline{2-6} 
 & Lecturer & Assistant & Full & Admin & All \\ 
\\[-1.8ex] & (1) & (2) & (3) & (4) & (5)\\ 
\hline \\[-1.8ex] 
 State Funding & $-$0.007 & 0.002 & $-$0.004 & $-$0.003 & $-$0.006 \\ 
  & (0.024) & (0.006) & (0.008) & (0.020) & (0.015) \\ 
 \hline \\[-1.8ex] 
Observations & 23,376 & 19,757 & 7,190 & 10,191 & 60,514 \\ 
R$^{2}$ & 0.013 & 0.006 & 0.014 & 0.068 & 0.016 \\ 
\hline 
\hline \\[-1.8ex] 
\end{tabular} 
}

    \makebox[\textwidth][c]{
\begin{tabular}{@{\extracolsep{5pt}}lcccc} 
\\[-1.8ex]\hline 
\hline \\[-1.8ex] 
 & \multicolumn{4}{c}{Dependent Variable: Promotion Rate} \\ 
\cline{2-5} 
 & Lecturer & Assistant & Associate & All \\ 
\\[-1.8ex] & (1) & (2) & (3) & (4)\\ 
\hline \\[-1.8ex] 
 State Funding & 0.015 & 0.036 & 0.029 & 0.016 \\ 
  & (0.007) & (0.019) & (0.065) & (0.008) \\ 
 \hline \\[-1.8ex] 
Observations & 16,420 & 16,972 & 4,340 & 42,132 \\ 
R$^{2}$ & 0.008 & 0.022 & 0.031 & 0.007 \\ 
\hline 
\hline \\[-1.8ex] 
\end{tabular} 
}

    \justify
    \footnotesize
    \textbf{Note}: 
    These tables show the second stage 2SLS estimates of regression specification \eqref{eqn:secondstage}, showing the effect of state funding changes on faculty outcomes at Illinois universities, using the funding shock to instrument for state funding.
    The shift-share instrument is based in the year the professor was hired, following definition in \autoref{sec:iv-model-indiv}.
    Each observation is a faculty member--year at an Illinois public university, where funding data come from IPEDS and faculty outcomes from IBHED data.
    The panels shows the effect of a $1$\% change in state funding per student at the university on faculties' salaries (base salary $+$ benefits), promotion rate (e.g., assistant professor to associate professor) , and rate of leaving their university (i.e., by quitting, retiring, or moving to another university).
    Standard errors are clustered at the university-year of hire level, and university $+$ year of hire fixed effects are included through--out.
    \label{tab:faculty-shock-illinois-rolling}
\end{table}


% Appendix
%%%%%%%%%%%%%%%%%%%%%%%%%%%%%%%%%%%%%%%%%
%% Appendix section
% Set-up the section.
\newpage
\appendix
\setcounter{table}{0}
\renewcommand{\thetable}{A\arabic{table}}
\setcounter{figure}{0}
\renewcommand{\thefigure}{A\arabic{figure}}

% Start appendix
\section{Appendix}
\label{appendix}
This project used data which are fully public, and computational tools which are fully open-source.
As such, all code and data (anonymised versions where necessary) involved in this project are available at this project's Github repository, available at \url{https://github.com/shoganhennessy/state-faculty-composition}.
They may be used for replication, or as the basis for further work, as needed.
Any comments or suggestions may be sent to me at \href{mailto:seh325@cornell.edu}{\nolinkurl{seh325@cornell.edu}}, or raised as an issue on the Github project.

A number of statistical packages, for the $R$ language \citep{R2022}, made the empirical analysis for this paper possible.
\begin{itemize}
    \item \textit{Tidyverse} \citep{tidyverse} collected tools for data analysis in the $R$ language.
    \item \textit{LFE} \citep{lfe} implemented linear fixed effect models, with instruments, crucial for the empirical estimation in \autoref{sec:empirics}.
    \item \textit{Stargazer} \citep{stargazer} provided methods to efficiently convert empirical results into presentable output in \LaTeX.
    \item \textit{Lpirfs} \citep{lpirfs2019} implemented estimation of the \cite{jorda2005} local projections methods, with instrumental variables, crucial to the local projections estimates presented in this project.
\end{itemize}

\subsection{IPEDS First Stage}
\label{sec:appendix-ipeds-firststage}

\begin{table}[H]
    \singlespacing
    %\centering
    \caption{Shift-Share Instrument Balance, Mean Characteristics Across Instrument Distribution.}
    \makebox[\textwidth][c]{% latex table generated in R 4.3.1 by xtable 1.8-4 package
% Tue Apr 30 17:21:52 2024
\begin{tabular}{lccccc}
  \hline
Instrument Quantile: & 1st & 2nd & 3rd & 4th & 5th \\ 
  \hline
IV Components, \$ per student: &  &  &  &  &  \\ 
  Funding shift--share & -1,474 & -2,589 & -3,566 & -5,002 & -8,208 \\ 
  Shift in state--wide funding & -6,138 & -7,112 & -8,593 & -10,575 & -14,018 \\ 
  Share reliance on state funding, \% in 1990--1993 & 0.26 & 0.38 & 0.42 & 0.48 & 0.59 \\ 
  \hline University Funding and Spending, \$ millions: &  &  &  &  &  \\ 
  State funding & 110 & 99 & 100 & 107 & 107 \\ 
  Tuition revenue & 216 & 124 & 93 & 77 & 58 \\ 
  Total non-inst. revenues & 355 & 241 & 203 & 191 & 170 \\ 
  Instruction spending & 219 & 139 & 108 & 101 & 93 \\ 
  Research Spending & 150 & 75 & 54 & 36 & 26 \\ 
  \hline University Funding and Spending, \$ per student &  &  &  &  &  \\ 
  State funding & 12,900 & 9,956 & 9,335 & 9,305 & 12,767 \\ 
  Tuition revenue & 13,130 & 8,530 & 6,875 & 6,047 & 5,507 \\ 
  Total non-inst. revenues & 30,502 & 20,394 & 17,194 & 15,877 & 18,823 \\ 
  Instruction spending & 21,680 & 12,526 & 9,798 & 8,482 & 9,953 \\ 
  Research spending & 16,750 & 5,093 & 3,328 & 2,226 & 2,432 \\ 
  \hline Selectivity: &  &  &  &  &  \\ 
  Reported enrolment & 14,088 & 12,434 & 11,329 & 11,546 & 10,253 \\ 
  Full-time equivalent enrolment & 12,453 & 10,597 & 9,638 & 9,877 & 8,555 \\ 
  Acceptance rate, \% & 0.71 & 0.73 & 0.71 & 0.64 & 0.60 \\ 
  6 Year graduation rate, \% & 0.56 & 0.47 & 0.44 & 0.45 & 0.45 \\ 
   \hline
\end{tabular}
}
    \label{tab:summary-quantiles}
    \justify
    \footnotesize
    \textbf{Note}:
    This table shows the summary statistics for every public university--year observation in IPEDS data, for each of the 5 quantiles of the funding shift-share instrument.
    Funding shift-share is the instrument defined in \autoref{sec:empirics} the product of (1) state-wide funding shift and (2) share reliance on state funding.
    State-wide funding shift is total funding for higher education in that university's state (divided by the count of state students); share reliance on state funding is the the total amount of state funding received by the university in 1990--1993 divided by total revenues for those years. 
    The column labelled ``1st'' refers to the mean for all university-year observations in the first quintile (bottom 20\%) of the funding shift-share distribution, and so on.
    The numbers are adjusted to 2021 figures by CPI-U.
    Non-institutional revenues refers to the sum of federal, state, and local funding plus tuition revenues; these sum to the majority of university funding, but exclude numbers such as university income from capital projects.
    Acceptance rate and 6 year graduation rate are for university undergraduates, and are calculated from IPEDS data available for academic years 1997 through 2018.
\end{table}

\begin{figure}[H]
    \centering
    \singlespacing
    \caption{Correlation Between State Funding Shift-Share and Public University State Funding in Surrounding Years.}
    \includegraphics[width=0.75\textwidth]{figures/lag-firststage.png}
    \label{fig:lag-firststage}
    \justify
    \footnotesize
    \textbf{Note}:
    This figure shows the correlation between state funding in year $t+k$ with the funding shift-share in year $t$ for a university, where $k = -7, \hdots, 5$ are the years on the $x$-axis.
    This shows that state funding and the funding shift-share are correlated across years, so that dynamic effects must be estimated by local projections --- and not simple OLS or 2SLS.
    The estimates are of \eqref{eqn:firststage}, calculated with IPEDS data, separately for each year relative to initial shock, using the $\log-\log$ specification, including fixed effects for university $+$ year, and clustering standard errors by university $+$ year.
\end{figure}

\begin{figure}[H]
    \centering
    \singlespacing
    \caption{Local Projection Estimates for Funding Shift-Share on State Funding, in IPEDS Data.}
    \includegraphics[width=0.6\textwidth]{figures/firststage-lp.png}
    \label{fig:firststage-lp}
    \justify
    \footnotesize
    \textbf{Note}:
    These figures show the local projections estimates of regression specification \eqref{eqn:firststage}, with the funding shift-share as an instrument for state funding, using IPEDS data.
    The coefficient estimate is effect of funding shift-share ($Z_{i,t}$) on state funding ($X_{i,t}$), while accounting for auto-correlation between different time periods --- i.e., between $Z_{i,t}, Z_{i,t-1}$ and $X_{i,t}, X_{i,t-1}$.
    These results use a $\log-\log$ specification, so the estimates are for the elasticity of state funding per student in a year $t+k$ with respect to funding shift-share in year $t$, where years $k = 0, \hdots, 10$ are on the $x$-axis. 
    Standard errors are clustered at the state-year level.
\end{figure}

\newpage
\subsection{Illinois IBHED First Stage}
\label{sec:iv-model-indiv}

This paper uses data on individual professors in the Illinois university system, to investigate the effects of changes in university revenues on the individual professors at the universities.
The outcomes here now refer to individual professors (e.g., their salary and promotion rate), so requires adjustment to the empirical approach, leveraging variation in university funding for the years after a professor joins the university.

\autoref{eqn:rolling-instrument} defines a rolling-share variant of the instrument, $\tilde Z_{i(j),t}$, where the university's state funding share exposure is based in the year a professor joins the university --- and not the base period 1990--1993.
$j$ indexes each professor in year $t$, $\tau(j)$ for the year the professor first joins their institution.
Identifying $\tau(j)$ is possible for $j$ by restricting to all professors hired 2011-2021 --- i.e., in the years after the start of the full panel.
It is not possible to discern the hiring year for professors who  were hired in the years preceding 2011, and so the entire sample is only possible to analyse using the base-share in years 1990-1993 formulation.

\begin{align}
    \label{eqn:rolling-instrument}
    \tilde Z_{i(j),t} &\coloneqq - \left[
    \left( \frac{\text{Total State Funding}_{s(j),t}}{\text{Student Population}_{s(j),t}} \right)
    \left( \frac{\text{State Funding}_{\tau(j)}}{\text{Total Revenues}_{i,\tau(j)}} \right) \right]
\end{align}

This approach leverages an insight, made available by level of the data: that an individual professor is affected by changes in university revenues after they have joined the university.
\autoref{sec:iv-model-uni} considers the number of professors employed by the university; whether a professor becomes employed at the university is likely affected by that university's finances.
The formulation here takes as given that the professor is already employed at the university, and then projects the effect of changes in state funding on these \textit{incumbent} professors following the state funding shift-share.
\autoref{tab:firststage-illinois} presents the first-stage results in Illinois data.

\begin{table}[h!]
    \singlespacing
    \centering
    \caption{First Stage Estimates, for State Funding by Funding Shift-Share in IBHED Data.}
    \textbf{Panel A: units in \$ per student}
    
    \makebox[\textwidth][c]{
\begin{tabular}{@{\extracolsep{5pt}}lcccc} 
\\[-1.8ex]\hline 
\hline \\[-1.8ex] 
 & \multicolumn{4}{c}{Dependent Variable: State Funding} \\ 
\cline{2-5} 
\\[-1.8ex] & (1) & (2) & (3) & (4)\\ 
\hline \\[-1.8ex] 
 Funding Shift-Share & $-$1.176 & $-$0.160 & $-$1.100 & $-$1.071 \\ 
  & (0.226) & (0.265) & (0.242) & (0.264) \\ 
  Tuition Revenue &  &  & $-$0.295 & 1.012 \\ 
  &  &  & (0.136) & (0.329) \\ 
  Constant &  & 9,716.437 &  & $-$1,708.334 \\ 
  &  & (1,805.394) &  & (2,716.150) \\ 
 \hline \\[-1.8ex] 
Uni. + Year fixed effects? & Yes & No & Yes & No \\ 
F stat. & 20.712 & 16.512 & 26.999 & 0.365 \\ 
Observations & 17,012 & 17,012 & 17,012 & 17,012 \\ 
R$^{2}$ & 0.918 & 0.0004 & 0.919 & 0.074 \\ 
\hline 
\hline \\[-1.8ex] 
\end{tabular} 
}
    
    \textbf{Panel A: units in log \$ per student}
    
    \makebox[\textwidth][c]{
\begin{tabular}{@{\extracolsep{5pt}}lcccc} 
\\[-1.8ex]\hline 
\hline \\[-1.8ex] 
 & \multicolumn{4}{c}{Dependent Variable: Non-institutional Revenues} \\ 
\cline{2-5} 
\\[-1.8ex] & (1) & (2) & (3) & (4)\\ 
\hline \\[-1.8ex] 
 Appropriations Shock & 0.245 & 0.210 & 0.201 & $-$0.018 \\ 
  & (0.025) & (0.027) & (0.031) & (0.108) \\ 
  Tuition Revenue & 0.813 & 0.810 &  &  \\ 
  & (0.127) & (0.083) &  &  \\ 
  Constant &  & 0.568 &  & 10.083 \\ 
  &  & (0.836) &  & (0.885) \\ 
 \hline \\[-1.8ex] 
Fixed effects? & Yes & No & Yes & No \\ 
F stat. & 84.131 & 70.734 & 40.985 & 0.028 \\ 
Observations & 185,570 & 185,570 & 185,570 & 185,570 \\ 
R$^{2}$ & 0.936 & 0.789 & 0.899 & 0.001 \\ 
\hline 
\hline \\[-1.8ex] 
\end{tabular} 
}
    
    \label{tab:firststage-illinois}
    \justify
    \footnotesize
    \textbf{Note}:
    These tables show the first stage OLS estimates of regression specification \eqref{eqn:secondstage1_indiv}, showing the effect of the funding shift-share on state funding to gauge performance as an instrument.
    Each observation is a professor-year, in the IBHED data, and funding data are merged from IPEDS.
    %Panel A shows the effect of an funding shift-share of \$-1 per student in the state on the number of \$'s of state funding per student at the university --- i.e.,
    %\$-1 funding shift-share per student in the state leads to \$1.176 less state funding per student at the university according to preferred specification column 1.
    Panel A shows the effect of a $-10$\% change funding shift-share per student in the state on $10$\% change in state funding per student at the university --- i.e.,
    $-10$\% funding shift-share per student in the state leads to $-9.77$\% less state funding per student at the university according to prefferred specification column 1.        
    Standard errors are clustered at the institution-year level, and institution $+$ year fixed effects are included where noted.
\end{table}

Exogeneity and relevance of the rolling-share instrument, $\tilde Z_{i(j),t}$, follows the same reasoning as that for the base-share instrument, $Z_{i,t}$, discussed in \autoref{sec:approp-shocks}.
The base-share instrument is appropriate for some outcomes with the individual Illinois professors, where appropriate.
We satisfy the assumptions for exogeneity by noting that none of the Illinois public campuses take the majority of state funding, and that the identification strategy relies on exogeneity in changes in state funding to individual professor-outcomes, following the year they joined the university.
Additionally, within-institution changes resulting from share reliance on state funding may be correlated with unobserved changes in the outcomes, so that \cite{NBERw27885} note the importance of controlling for the base share and state student population.
The formulation here implicitly controls for these factors via the fixed effects; results are relatively similar while including these controls with and without including fixed effects, and so are omitted.

The instrumental variables model is then defined as follows, where $i(j)$ refers to the institution that professor $j$ is employed at, and $Y_{j,t}$ for salary, rate of promotion, and propensity to leave the Illinois public university system.
The system includes fixed effects for the institution and first year of employment.
The instrument varies by institution, based in the year of first employment, so that these are the corresponding fixed effects and level of clustered standard errors.
\begin{eqnarray}
    \label{eqn:secondstage1_indiv}
    X_{i(j),t} &=& \theta_{i(j)} + \phi_{\tau(j)} + \delta \tilde Z_{i(j),t} + \epsilon_{i(j),t} \\
    \label{eqn:secondstage2_indiv}
    Y_{j,t} &=& \mu_{i(j)} + \nu_{\tau(j)} + \beta \widehat X_{i(j),t} + \varepsilon_{j,t}
\end{eqnarray}
We then interpret parameter $\beta$ as the effect of changes in state funding at an Illinois public university, via state funding shift-shares, on an individual professor's outcome $Y_{j,t}$.

\newpage
\begin{figure}[H]
    \centering
    \singlespacing
    \caption{Local Projection Estimates for Effect of State Funding on Faculty Promotion Rate at Illinois Public Universities, by Professor Group.}
    \begin{subfigure}[b]{0.495\textwidth}
        \centering
        \caption{Lecturers.}
        \includegraphics[width=\textwidth]{figures/promoted-lecturer-illinois-lp-rolling.png}
        \label{fig:promoted-lecturer-illinois-lp-rolling}
    \end{subfigure}
    \begin{subfigure}[b]{0.495\textwidth}
        \centering
        \caption{Assistant Professors.}
        \includegraphics[width=\textwidth]{figures/promoted-assistant-illinois-lp-rolling.png}
        \label{fig:promoted-assistant-illinois-lp-rolling}
    \end{subfigure}
    \begin{subfigure}[b]{0.495\textwidth}
        \centering
        \caption{Full Professors.}
        \includegraphics[width=\textwidth]{figures/promoted-full-illinois-lp-rolling.png}
        \label{fig:promoted-full-illinois-lp-rolling}
    \end{subfigure}
    \label{fig:promoted-illinois-lp-rolling}
    \justify
    \footnotesize
    \textbf{Note}:
    These figures show the local projections estimates of regression specification \eqref{eqn:secondstage}, with the funding shift-share as an instrument for state funding.
    The unit of analysis is an individual faculty member (at an Illinois public university); funding data come from IPEDS, and faculty promotion rate from IBHED.
    The coefficient estimate is effect of state funding ($X_{i(j),t}$) on faculty promotion rate ($Y_{j,t}$), using the funding shift-share instrument ($Z_{i(j),t}$), while accounting for auto-correlation between different time periods --- i.e., between $X_{i(j),t}, X_{i(j),t-1}$ and $Y_{i(j),t}, Y_{i(j),t-1}$.
    These results use a $\log-\log$ specification, so the estimates are for the rate of promotion in a year $t+k$ affected by a 1\% change in state funding in year $t$, where years $k = 0, \hdots, 4$ are on the $x$-axis. 
    Standard errors are clustered at the university-year level, and \autoref{sec:iv-model-indiv} fully describes the differences in empirical specification when unit of analysis is an individual faculty member.
\end{figure}

\newpage
\begin{table}[H]
    \singlespacing
    \centering
    \caption{Effects of Changes in State Funding on University Faculty Composition, in Illinois 2010--2021, OLS and 2SLS Estimates.}

    \textbf{Panel A: units in \$ per student}

    \makebox[\textwidth][c]{
\begin{tabular}{@{\extracolsep{5pt}}lcccccccc} 
\\[-1.8ex]\hline 
\hline \\[-1.8ex] 
 & \multicolumn{8}{c}{Dependent Variable: Employment Count by Professor Group} \\ 
\cline{2-9} 
\\[-1.8ex] & \multicolumn{2}{c}{Lecturers} & \multicolumn{2}{c}{Asst. Professors} & \multicolumn{2}{c}{Full Professors} & \multicolumn{2}{c}{All Faculty} \\ 
 & OLS & 2SLS & OLS & 2SLS & OLS & 2SLS & OLS & 2SLS \\ 
\\[-1.8ex] & (1) & (2) & (3) & (4) & (5) & (6) & (7) & (8)\\ 
\hline \\[-1.8ex] 
 State Funding & $-$8.399 & $-$32.989 & $-$1.981 & $-$1.491 & $-$2.567 & $-$2.012 & $-$12.508 & 27.514 \\ 
  & (4.661) & (31.616) & (2.194) & (22.859) & (2.799) & (10.909) & (9.620) & (84.320) \\ 
 \hline \\[-1.8ex] 
Outcome Mean & 351.306 & 351.306 & 273.757 & 273.757 & 477.597 & 477.597 & 1303.014 & 1303.014 \\ 
Observations & 144 & 144 & 144 & 144 & 144 & 144 & 144 & 144 \\ 
R$^{2}$ & 0.893 & 0.843 & 0.973 & 0.973 & 0.992 & 0.992 & 0.983 & 0.978 \\ 
\hline 
\hline \\[-1.8ex] 
\end{tabular} 
}
    
    \textbf{Panel B: units in log \$ per student}
    
    \makebox[\textwidth][c]{
\begin{tabular}{@{\extracolsep{5pt}}lcccccccc} 
\\[-1.8ex]\hline 
\hline \\[-1.8ex] 
 & \multicolumn{8}{c}{Dependent Variable: Employment Count by Professor Group} \\ 
\cline{2-9} 
\\[-1.8ex] & \multicolumn{2}{c}{Lecturer} & \multicolumn{2}{c}{Assistant} & \multicolumn{2}{c}{Full} & \multicolumn{2}{c}{All} \\ 
 & OLS & 2SLS & OLS & 2SLS & OLS & 2SLS & OLS & 2SLS \\ 
\\[-1.8ex] & (1) & (2) & (3) & (4) & (5) & (6) & (7) & (8)\\ 
\hline \\[-1.8ex] 
 State Funding & $-$0.015 & $-$0.017 & 0.049 & 0.058 & 0.007 & $-$0.029 & 0.004 & $-$0.008 \\ 
  & (0.033) & (0.026) & (0.045) & (0.038) & (0.021) & (0.032) & (0.028) & (0.022) \\ 
 \hline \\[-1.8ex] 
Outcome Mean & 2.523 & 2.523 & 1.491 & 1.491 & 2.729 & 2.729 & 8.141 & 8.141 \\ 
Observations & 144 & 144 & 144 & 144 & 144 & 144 & 144 & 144 \\ 
R$^{2}$ & 0.700 & 0.700 & 0.789 & 0.789 & 0.839 & 0.836 & 0.633 & 0.632 \\ 
\hline 
\hline \\[-1.8ex] 
\end{tabular} 
}

    \label{tab:facultycount-illinois-reg}
    \justify
    \footnotesize
    \textbf{Note}:
    These tables show the second stage OLS and 2SLS estimates of regression specification \eqref{eqn:secondstage}, showing the effect of state funding changes on number of faculty per student in Illinois universities, using the funding shift-share to instrument for state funding in the columns labelled 2SLS.
    Each observation is a public university-year in the state of Illinois, where funding data come from IPEDS and faculty count come from IBHED data.
    Panel A shows the effect of a fall in state funding \$$-1,000$ per student in the state on the number of professors.
    Panel B shows the effect of a $10$\% change in state funding per student at the university on the 10\% change in the number of professors per students.
    Outcome-mean is the mean of the outcome, for Panel A the number of professors per student, for Panel B the number of faculty per student.
    Panel B uses $\log$ faculty count per student as the outcome, though the outcome mean is count of faculty per student (not in $\log$ terms).
    Standard errors are clustered at the university-year level, and university $+$ year fixed effects are included through--out.
\end{table}

\newpage
\begin{figure}[H]
    \centering
    \singlespacing
    \caption{Local Projection Estimates for Effect of State Funding on Faculty Exit Rate at Illinois Public Universities, by Professor Group.}
    \begin{subfigure}[b]{0.495\textwidth}
        \centering
        \caption{Lecturers.}
        \includegraphics[width=\textwidth]{figures/exit-lecturer-illinois-lp-rolling.png}
        \label{fig:exit-lecturer-illinois-lp-rolling}
    \end{subfigure}
    \begin{subfigure}[b]{0.495\textwidth}
        \centering
        \caption{Assistant Professors.}
        \includegraphics[width=\textwidth]{figures/exit-assistant-illinois-lp-rolling.png}
        \label{fig:exit-assistant-illinois-lp-rolling}
    \end{subfigure}
    \begin{subfigure}[b]{0.495\textwidth}
        \centering
        \caption{Full Professors.}
        \includegraphics[width=\textwidth]{figures/exit-full-illinois-lp-rolling.png}
        \label{fig:exit-full-illinois-lp-rolling}
    \end{subfigure}
    \begin{subfigure}[b]{0.495\textwidth}
        \centering
        \caption{Administrator Professors.}
        \includegraphics[width=\textwidth]{figures/exit-administrator-illinois-lp-rolling.png}
        \label{fig:exit-administrator-illinois-lp-rolling}
    \end{subfigure}
    \label{fig:exit-illinois-lp-rolling}
    \justify
    \footnotesize
    \textbf{Note}:
    These figures show the local projections estimates of regression specification \eqref{eqn:secondstage}, with the funding shift-share as an instrument for state funding.
    The unit of analysis is an individual faculty member (at an Illinois public university); funding data come from IPEDS, and faculty promotion rate from IBHED.
    The coefficient estimate is effect of state funding ($X_{i(j),t}$) on faculty promotion rate ($Y_{j,t}$), using the funding shift-share instrument ($Z_{i(j),t}$), while accounting for auto-correlation between different time periods --- i.e., between $X_{i(j),t}, X_{i(j),t-1}$ and $Y_{i(j),t}, Y_{i(j),t-1}$.
    These results use a rate$-\log$ specification, so the estimates are for the rate of promotion in a year $t+k$ affected by a 1\% change in state funding in year $t$, where years $k = 0, \hdots, 4$ are on the $x$-axis. 
    Standard errors are clustered at the university-year level, and \autoref{sec:iv-model-indiv} fully describes the differences in empirical specification when unit of analysis is an individual faculty member.
\end{figure}

\newpage
\subsection{Professor Hiring}
\label{sec:appendix-hiring}

These results were produced by integrating the total count of professor hires for 2010--2021 for the top-ranked 180 US universities with a sum of the funding variables, and then estimating the models specified in \autoref{sec:iv-model-uni}.
\begin{table}[H]
    \singlespacing
    \centering
    \caption{OLS and 2SLS Estimates for University Faculty Hires, in Illinois 2011--2021.}

    \textbf{Panel A: units in \$ per student}

    \makebox[\textwidth][c]{
\begin{tabular}{@{\extracolsep{5pt}}lcccccccc} 
\\[-1.8ex]\hline 
\hline \\[-1.8ex] 
 & \multicolumn{8}{c}{Dependent Variable: Yearly New Hires by Professor Group} \\ 
\cline{2-9} 
\\[-1.8ex] & \multicolumn{2}{c}{Lecturers} & \multicolumn{2}{c}{Asst. Professors} & \multicolumn{2}{c}{Full Professors} & \multicolumn{2}{c}{All Faculty} \\ 
 & OLS & 2SLS & OLS & 2SLS & OLS & 2SLS & OLS & 2SLS \\ 
\\[-1.8ex] & (1) & (2) & (3) & (4) & (5) & (6) & (7) & (8)\\ 
\hline \\[-1.8ex] 
 State Funding & $-$2.551 & $-$2.280 & 0.060 & $-$5.126 & $-$0.095 & $-$2.562 & $-$2.677 & 23.012 \\ 
  & (1.420) & (9.168) & (0.675) & (7.210) & (0.249) & (7.051) & (2.185) & (52.652) \\ 
 \hline \\[-1.8ex] 
Outcome Mean & 73.275 & 73.275 & 42.771 & 42.771 & 12.301 & 12.301 & 151.932 & 151.932 \\ 
Observations & 131 & 131 & 131 & 131 & 113 & 113 & 132 & 132 \\ 
R$^{2}$ & 0.839 & 0.839 & 0.934 & 0.902 & 0.788 & 0.752 & 0.918 & 0.793 \\ 
\hline 
\hline \\[-1.8ex] 
\end{tabular} 
}
    
    \textbf{Panel B: units in log \$ per student}
    
    \makebox[\textwidth][c]{
\begin{tabular}{@{\extracolsep{5pt}}lcccccccc} 
\\[-1.8ex]\hline 
\hline \\[-1.8ex] 
 & \multicolumn{8}{c}{Dependent Variable: Employment Count by Professor Group} \\ 
\cline{2-9} 
\\[-1.8ex] & \multicolumn{2}{c}{Lecturer} & \multicolumn{2}{c}{Assistant} & \multicolumn{2}{c}{Full} & \multicolumn{2}{c}{All} \\ 
 & OLS & 2SLS & OLS & 2SLS & OLS & 2SLS & OLS & 2SLS \\ 
\\[-1.8ex] & (1) & (2) & (3) & (4) & (5) & (6) & (7) & (8)\\ 
\hline \\[-1.8ex] 
 State Funding & 0.120 & 0.158 & 0.172 & 0.174 & 0.191 & 0.235 & 0.046 & 0.082 \\ 
  & (0.071) & (0.065) & (0.154) & (0.143) & (0.113) & (0.137) & (0.080) & (0.047) \\ 
 \hline \\[-1.8ex] 
Outcome Mean & 0.494 & 0.494 & 0.234 & 0.234 & 0.051 & 0.051 & 0.993 & 0.993 \\ 
Observations & 131 & 131 & 131 & 131 & 113 & 113 & 132 & 132 \\ 
R$^{2}$ & 0.749 & 0.749 & 0.482 & 0.482 & 0.628 & 0.627 & 0.580 & 0.579 \\ 
\hline 
\hline \\[-1.8ex] 
\end{tabular} 
}
    \label{tab:facultyhires-illinois-reg}
    \justify
    \footnotesize
    \textbf{Note}:
    These tables show the second stage OLS and 2SLS estimates of regression specification \eqref{eqn:secondstage}, showing the effect of state funding changes on number of faculty hires at Illinois universities, using the funding shift-share to instrument for state funding in the columns labelled 2SLS.
    Each observation is a public university-year in the state of Illinois, where funding data come from IPEDS and faculty count come from IBHED data.
    Panel A shows the effect of a fall in state funding \$$-1,000$ per student in the state on the number of new faculty hires by position.
    Panel B shows the effect of a $10$\% change in state funding per student at the university on the 10\% change in the number of faculty hires per students.
    Outcome-mean is the mean of the outcome, for Panel A the number of faculty hires, for Panel B the number of faculty hires per student.
    Panel B uses new faculty hires per student as the outcome (in $\log$ terms), though the outcome mean is count of new faculty hires per student (not in $\log$ terms).
    Standard errors are clustered at the university-year level, and university $+$ year fixed effects are included through--out.
\end{table}

There were no observable differences in the hiring rate of male vs female faculty.

\begin{figure}[H]
    \centering
    \singlespacing
    \caption{Correlation Between State Funding and Total COunt Professors Hired, 2011--2021.}
    \includegraphics[width=\textwidth]{figures/hiring-correlation.png}
    \label{fig:hiring-correlation}
    \justify
    \footnotesize
    \textbf{Note}:
    This figure shows the correlation between total state funding per student and total professors hired, 2010--2021.
    Funding data are taken from IPEDS, and data on professor hiring  from \cite{wapman2022quantifying,wapman2022zenodo}.
\end{figure}

\begin{table}[h!]
    \singlespacing
    \centering
    \caption{OLS and 2SLS Estimates for Professor Hiring, Total for 2011--2020.}
    \makebox[\textwidth][c]{
\begin{tabular}{@{\extracolsep{5pt}}lcccccc} 
\\[-1.8ex]\hline 
\hline \\[-1.8ex] 
 & \multicolumn{6}{c}{Dependent Variable: Professor Hiring Count} \\ 
\cline{2-7} 
\\[-1.8ex] & \multicolumn{2}{c}{Men} & \multicolumn{2}{c}{Women} & \multicolumn{2}{c}{Total} \\ 
 & OLS & 2SLS & OLS & 2SLS & OLS & 2SLS \\ 
\\[-1.8ex] & (1) & (2) & (3) & (4) & (5) & (6)\\ 
\hline \\[-1.8ex] 
 State Funding & 0.805 & 1.308 & 0.845 & 1.325 & 0.848 & 1.306 \\ 
  & (0.222) & (0.365) & (0.235) & (0.335) & (0.220) & (0.352) \\ 
 \hline \\[-1.8ex] 
Observations & 157 & 157 & 157 & 157 & 157 & 157 \\ 
R$^{2}$ & 0.396 & 0.366 & 0.415 & 0.383 & 0.408 & 0.381 \\ 
\hline 
\hline \\[-1.8ex] 
\end{tabular} 
}
    \label{tab:hiring-shock-reg}
    \justify
    \footnotesize
    \textbf{Note}: 
    This table show the second stage 2SLS estimates of regression specification \eqref{eqn:secondstage}, showing the effect of state funding changes on the number of faculty hires (per student) total for 2011--2021 at US public universities, using the funding shift-share to instrument for state funding.
    Yearly variation in total hires is not observed here, so only the total hires across 2011--2021 for 157 universities, can be considered.
    Each observation is a public university across the years 2011--2021, where data on total funding across 2011--2021 come from IPEDS and faculty count total from \citep{wapman2022quantifying,wapman2022zenodo}.
    The panels show the effect of a $1$\% change in state funding per student at the university (total for 2011--2021) on the number of new faculty hires by gender (and all).
    Standard errors are clustered at the state level, and state fixed effects are included through--out.
\end{table}

\newpage
\subsection{Robustness Checks}

\begin{table}[H]
    \singlespacing
    \centering
    \caption{Effects of State Funding on Faculty Counts, IPEDS 1990--2021, IV Estimates by Institution Selectivity.}

    \textbf{Panel A: units in \$ per student}

    \makebox[\textwidth][c]{
        \begin{tabular}{@{\extracolsep{5pt}}lcccccl} 
            \\[-1.8ex]\hline 
            \hline \\[-1.8ex]
            & First-stage & Lecturers & Asst. Professors & Full Professors & All Faculty & Observations \\ 
            \cline{2-7} 
            \\[-1.8ex]
            % latex table generated in R 4.3.1 by xtable 1.8-4 package
% Tue Apr 30 17:23:04 2024
  &  &  &  &  &  &  \\ 
  Most Selective: & -1.804 & 0.176 & 5.421 & 3.011 & 14.298 & 367 \\ 
   & (0.178) & (1.836) & (2.389) & (4.015) & (10.112) &  \\ 
   & [13036.59] & [0.004] & [0.011] & [0.033] & [0.05] &  \\ 
  Selective: & -5.555 & -0.089 & 0.025 & 0.044 & 0.022 & 815 \\ 
   & (0.324) & (0.019) & (0.006) & (0.004) & (0.003) &  \\ 
   & [12084.012] & [0.005] & [0.011] & [0.03] & [0.046] &  \\ 
  Unranked: & -7.264 & -0.058 & 0.018 & 0.017 & 0.008 & 15830 \\ 
   & (0.272) & (0.005) & (0.002) & (0.002) & (0.001) &  \\ 
   & [10206.27] & [0.006] & [0.012] & [0.022] & [0.041] &  \\ 
  
            \\[-1.8ex] \hline 
            \hline 
        \end{tabular}}
    
        \vspace{0.2cm}
    \textbf{Panel B: units in log \$ per student}
    
    \makebox[\textwidth][c]{
        \begin{tabular}{@{\extracolsep{5pt}}lcccccl} 
            \\[-1.8ex]\hline 
            \hline \\[-1.8ex]
            & First-stage & Lecturers & Asst. Professors & Full Professors & All Faculty & Observations \\ 
            \cline{2-7} 
            \\[-1.8ex]
            % latex table generated in R 4.3.1 by xtable 1.8-4 package
% Tue Apr 30 17:23:04 2024
  &  &  &  &  &  &  \\ 
  Most Selective: & -0.944 & -0.201 & 0.162 & 0.001 & -0.01 & 367 \\ 
   & (0.043) & (0.134) & (0.069) & (0.041) & (0.037) &  \\ 
   & [13036.59] & [0.004] & [0.011] & [0.033] & [0.05] &  \\ 
  Selective: & -1.067 & -0.465 & 0.129 & 0.229 & 0.116 & 815 \\ 
   & (0.022) & (0.092) & (0.031) & (0.018) & (0.016) &  \\ 
   & [12084.012] & [0.005] & [0.011] & [0.03] & [0.046] &  \\ 
  Unranked: & -0.965 & -0.436 & 0.137 & 0.131 & 0.062 & 15830 \\ 
   & (0.019) & (0.032) & (0.015) & (0.012) & (0.009) &  \\ 
   & [10206.27] & [0.006] & [0.012] & [0.022] & [0.041] &  \\ 
  
            \\[-1.8ex] \hline 
            \hline 
        \end{tabular}}
    \label{tab:facultycount-heterogeneity}
    \justify
    \footnotesize
    \textbf{Note}:
    These tables show the IV estimates of regression specification \eqref{eqn:secondstage}, 
    in the same manner as \autoref{tab:facultycount-shock-reg}, but restricting to institutions ranked as most selective, selective, and unranked by \cite{barrons2009} --- and including state
    The first column presents the coefficient between state funding and shift-share IV, which is a string first-stage among every level of selectivity for public universities.
    The other columns show the coefficient between state funding the count of faculty per students.
    For example, Row 1 of panel B shows that a $10\%$ cut in the state shift-share leads to a fall of $9.44\%$ in state funding for the public universities ranked as ``most selective.''
    Standard errors for the coefficient estimates are in brackets, and the outcome mean are in square brackets beneath.
\end{table}

\begin{table}[H]
    \singlespacing
    \centering
    \caption{First-stage Robustness Checks for Effects of State Funding Shift-Share on State Funding, OLS Estimates.}
    \makebox[\textwidth][c]{
\begin{tabular}{@{\extracolsep{5pt}}lcc} 
\\[-1.8ex]\hline 
\hline \\[-1.8ex] 
 & \multicolumn{2}{c}{Dependent Variable: State Funding} \\ 
\cline{2-3} 
 & Raw Count Units & Log Units \\ 
\\[-1.8ex] & (1) & (2)\\ 
\hline \\[-1.8ex] 
 State Funding & $-$1.299 & $-$1.050 \\ 
  & (0.172) & (0.041) \\ 
  State Funding, Base Share \% & $-$95.489 & $-$0.895 \\ 
  & (40.941) & (0.141) \\ 
  Acceptance Rate, \% & $-$13.540 & $-$0.012 \\ 
  & (15.208) & (0.046) \\ 
  6--Year Completion Rate, \% & 69.341 & 0.148 \\ 
  & (23.718) & (0.074) \\ 
  Tuition Revenue, \$ millions & 1.223 & 0.350 \\ 
  & (1.433) & (0.040) \\ 
  Enrolment, FTE thousands & $-$4.748 & $-$0.669 \\ 
  & (58.129) & (0.055) \\ 
  State Enrolment, thousands & 2.215 & 0.258 \\ 
  & (1.236) & (0.053) \\ 
  Percent of State Enrolment & 0.803 & 0.0003 \\ 
  & (3.356) & (0.0001) \\ 
 \hline \\[-1.8ex] 
Units  & Raw counts & Log, \% terms \\ 
F stat. & 57.356 & 660.878 \\ 
State + Year fixed effects? & Yes & Yes \\ 
Observations & 13,687 & 13,687 \\ 
R$^{2}$ & 0.382 & 0.453 \\ 
\hline 
\hline \\[-1.8ex] 
\end{tabular} 
}
    \label{tab:firststage-robustness-checks}
    \justify
    \footnotesize
    \textbf{Note}:
    These tables show the second stage OLS and 2SLS estimates of regression specification \eqref{eqn:firststage}, showing the effect of state funding shift-share on each university's state funding.
    This table differs from the main analysis by replacing University $+$ Year fixed effects with State $+$ Year fixed effects, measuring enrolment in full-time equivalent (FTE), and including controls for (1)
    State Funding as a percent of total funding in base period, (2) acceptance rate in mid-2000s, (3) 6--year completion rate in mid-2000s, (4) tuition revenue, (5) enrolment measured by full-time equivalent FTE, (6) total public university enrolment in the entire state, (7) percent of public university enrolment for the state enrolled at this university.
    The first column uses the raw count specification, and the second column uses the $\log$ specification for percentage terms.
\end{table}

\begin{table}[H]
    \singlespacing
    \centering
    \caption{Robustness Checks for Effects of State Funding Cuts on Faculty Composition, OLS and 2SLS Estimates in $\log$ Units.}
    \makebox[\textwidth][c]{
        %\small
        
\begin{tabular}{@{\extracolsep{5pt}}lcccccccc} 
\\[-1.8ex]\hline 
\hline \\[-1.8ex] 
 & \multicolumn{8}{c}{Dependent Variable: Log Faculty Count per Students, by Position} \\ 
\cline{2-9} 
\\[-1.8ex] & \multicolumn{2}{c}{Lecturers} & \multicolumn{2}{c}{Asst. Professors} & \multicolumn{2}{c}{Full Professors} & \multicolumn{2}{c}{All Faculty} \\ 
 & OLS & 2SLS & OLS & 2SLS & OLS & 2SLS & OLS & 2SLS \\ 
\\[-1.8ex] & (1) & (2) & (3) & (4) & (5) & (6) & (7) & (8)\\ 
\hline \\[-1.8ex] 
 State Funding & $-$0.116 & $-$0.352 & 0.072 & 0.069 & 0.147 & 0.104 & 0.095 & 0.038 \\ 
  & (0.042) & (0.088) & (0.026) & (0.048) & (0.050) & (0.022) & (0.038) & (0.023) \\ 
  State Funding, Base Share \% & $-$0.053 & $-$0.042 & $-$0.022 & $-$0.022 & 0.075 & 0.077 & 0.025 & 0.028 \\ 
  & (0.160) & (0.172) & (0.049) & (0.049) & (0.047) & (0.047) & (0.037) & (0.038) \\ 
  Acceptance Rate, \% & 0.015 & 0.00004 & 0.004 & 0.004 & $-$0.012 & $-$0.015 & $-$0.013 & $-$0.016 \\ 
  & (0.077) & (0.082) & (0.043) & (0.044) & (0.038) & (0.039) & (0.029) & (0.031) \\ 
  6--Year Completion Rate, \% & $-$0.029 & 0.023 & 0.130 & 0.130 & 0.241 & 0.250 & 0.156 & 0.169 \\ 
  & (0.112) & (0.125) & (0.034) & (0.033) & (0.036) & (0.034) & (0.021) & (0.024) \\ 
  Tuition Revenue, \$ millions & 0.347 & 0.421 & 0.046 & 0.046 & 0.235 & 0.248 & 0.215 & 0.233 \\ 
  & (0.153) & (0.163) & (0.053) & (0.055) & (0.033) & (0.033) & (0.038) & (0.033) \\ 
  Enrolment, FTE thousands & $-$0.580 & $-$0.698 & $-$0.223 & $-$0.224 & $-$0.278 & $-$0.300 & $-$0.336 & $-$0.364 \\ 
  & (0.179) & (0.198) & (0.062) & (0.065) & (0.039) & (0.036) & (0.048) & (0.037) \\ 
  State Enrolment, thousands & 0.232 & 0.331 & $-$0.112 & $-$0.111 & $-$0.196 & $-$0.178 & $-$0.102 & $-$0.079 \\ 
  & (0.189) & (0.177) & (0.068) & (0.069) & (0.061) & (0.050) & (0.046) & (0.032) \\ 
  Percent of State Enrolment & 0.150 & 0.417 & 0.021 & 0.024 & $-$0.021 & 0.027 & 0.095 & 0.159 \\ 
  & (0.591) & (0.595) & (0.177) & (0.183) & (0.168) & (0.130) & (0.158) & (0.135) \\ 
 \hline \\[-1.8ex] 
Outcome Mean & 0.681 & 0.681 & 1.394 & 1.394 & 2.666 & 2.666 & 4.816 & 4.816 \\ 
Observations & 13,687 & 13,687 & 13,687 & 13,687 & 13,687 & 13,687 & 13,687 & 13,687 \\ 
R$^{2}$ & 0.341 & 0.327 & 0.440 & 0.440 & 0.438 & 0.433 & 0.543 & 0.529 \\ 
\hline 
\hline \\[-1.8ex] 
\end{tabular} 
}
    \label{tab:facultycount-robustness-checks}
    \justify
    \footnotesize
    \textbf{Note}:
    These tables show the second stage OLS and 2SLS estimates of regression specification \eqref{eqn:secondstage}, showing the effect of state funding changes on number of faculty per student in Illinois universities, using the funding shift-share to instrument for state funding in the columns labelled 2SLS.
    This table differs from the main analysis by replacing University $+$ Year fixed effects with State $+$ Year fixed effects, measuring enrolment in full-time equivalent (FTE), and including controls for (1)
    State Funding as a percent of total funding in base period, (2) acceptance rate in mid-2000s, (3) 6--year completion rate in mid-2000s, (4) tuition revenue, (5) enrolment measured by full-time equivalent FTE, (6) total public university enrolment in the entire state, (7) percent of public university enrolment for the state enrolled at this university.
\end{table}


\begin{table}[H]
    \singlespacing
    \centering
    \caption{Robustness Checks for Effects of State Funding Cuts on Faculty Composition, OLS and 2SLS Estimates in Raw Count Units.}
    \makebox[\textwidth][c]{
        \small
        
\begin{tabular}{@{\extracolsep{5pt}}lcccccccc} 
\\[-1.8ex]\hline 
\hline \\[-1.8ex] 
 & \multicolumn{8}{c}{Dependent Variable: Faculty Count per 1,000 Students, by Position} \\ 
\cline{2-9} 
\\[-1.8ex] & \multicolumn{2}{c}{Lecturers} & \multicolumn{2}{c}{Asst. Professors} & \multicolumn{2}{c}{Full Professors} & \multicolumn{2}{c}{All Faculty} \\ 
 & OLS & 2SLS & OLS & 2SLS & OLS & 2SLS & OLS & 2SLS \\ 
\\[-1.8ex] & (1) & (2) & (3) & (4) & (5) & (6) & (7) & (8)\\ 
\hline \\[-1.8ex] 
 State Funding & $-$1.001 & $-$1.641 & 1.032 & 4.573 & 7.128 & 6.836 & 7.213 & 10.746 \\ 
  & (0.461) & (1.345) & (0.397) & (2.094) & (1.212) & (3.097) & (1.396) & (5.276) \\ 
  State Funding, Base Share \% & 55.056 & 57.126 & 15.654 & 4.200 & $-$38.543 & $-$37.598 & 18.057 & 6.627 \\ 
  & (24.373) & (25.604) & (35.625) & (37.658) & (53.606) & (51.900) & (95.185) & (95.017) \\ 
  Acceptance Rate, \% & $-$7.862 & $-$8.694 & 6.849 & 11.450 & $-$41.508 & $-$41.888 & $-$53.375 & $-$48.783 \\ 
  & (10.581) & (9.949) & (9.243) & (14.230) & (21.648) & (20.581) & (26.159) & (29.083) \\ 
  6--Year Completion Rate, \% & 8.357 & 12.754 & 21.899 & $-$2.429 & 169.846 & 171.853 & 202.607 & 178.329 \\ 
  & (18.767) & (21.069) & (13.444) & (23.497) & (42.483) & (45.635) & (39.685) & (47.031) \\ 
  Tuition Revenue, \$ millions & 0.185 & 0.186 & 0.123 & 0.118 & 0.215 & 0.216 & 0.622 & 0.617 \\ 
  & (0.035) & (0.035) & (0.027) & (0.028) & (0.064) & (0.064) & (0.137) & (0.137) \\ 
  Enrolment, FTE thousands & 2.580 & 2.575 & 7.029 & 7.060 & 22.975 & 22.972 & 32.420 & 32.450 \\ 
  & (0.858) & (0.848) & (0.595) & (0.554) & (1.272) & (1.276) & (1.925) & (1.900) \\ 
  State Enrolment, thousands & 0.088 & 0.086 & $-$0.061 & $-$0.054 & $-$0.218 & $-$0.218 & $-$0.188 & $-$0.182 \\ 
  & (0.021) & (0.021) & (0.020) & (0.020) & (0.069) & (0.067) & (0.098) & (0.098) \\ 
  Percent of State Enrolment & $-$7.625 & $-$6.651 & 126.755 & 121.370 & 168.080 & 168.525 & 255.309 & 249.936 \\ 
  & (48.023) & (48.705) & (42.249) & (37.381) & (72.631) & (72.682) & (105.737) & (102.570) \\ 
 \hline \\[-1.8ex] 
Outcome Mean & 59.253 & 59.253 & 116.121 & 116.121 & 269.103 & 269.103 & 452.507 & 452.507 \\ 
Observations & 13,687 & 13,687 & 13,687 & 13,687 & 13,687 & 13,687 & 13,687 & 13,687 \\ 
R$^{2}$ & 0.647 & 0.646 & 0.861 & 0.845 & 0.942 & 0.942 & 0.954 & 0.954 \\ 
\hline 
\hline \\[-1.8ex] 
\end{tabular} 
}
    \label{tab:facultycount-rawcount-robustness-checks}
    \justify
    \footnotesize
    \textbf{Note}:
    These tables show the second stage OLS and 2SLS estimates of regression specification \eqref{eqn:secondstage}, showing the effect of state funding changes on count of faculty in Illinois universities, using the funding shift-share to instrument for state funding in the columns labelled 2SLS.
    This table differs from the main analysis by replacing University $+$ Year fixed effects with State $+$ Year fixed effects, measuring enrolment in full-time equivalent (FTE), and including controls for (1)
    State Funding as a percent of total funding in base period, (2) acceptance rate in mid-2000s, (3) 6--year completion rate in mid-2000s, (4) tuition revenue, (5) enrolment measured by full-time equivalent FTE, (6) total public university enrolment in the entire state, (7) percent of public university enrolment for the state enrolled at this university.
\end{table}

\begin{table}[H]
    \singlespacing
    \centering
    \caption{Effects of State--Wide Funding Changes on Private University Faculty Counts, IPEDS 1990--2021, OLS and 2SLS Estimates.}

    \textbf{Panel A: units in \$ per student}

    \makebox[\textwidth][c]{
\begin{tabular}{@{\extracolsep{5pt}}lcccccccc} 
\\[-1.8ex]\hline 
\hline \\[-1.8ex] 
 & \multicolumn{8}{c}{Dependent Variable: Faculty Count per 1,000 Students, by Position} \\ 
\cline{2-9} 
\\[-1.8ex] & \multicolumn{2}{c}{Lecturers} & \multicolumn{2}{c}{Asst. Professors} & \multicolumn{2}{c}{Full Professors} & \multicolumn{2}{c}{All Faculty} \\ 
 & OLS & 2SLS & OLS & 2SLS & OLS & 2SLS & OLS & 2SLS \\ 
\\[-1.8ex] & (1) & (2) & (3) & (4) & (5) & (6) & (7) & (8)\\ 
\hline \\[-1.8ex] 
 State Funding & 0.002 & $-$0.099 & 0.003 & 0.037 & 0.005 & $-$0.060 & 0.011 & $-$0.099 \\ 
  & (0.002) & (0.193) & (0.001) & (0.190) & (0.003) & (0.176) & (0.006) & (0.406) \\ 
 \hline \\[-1.8ex] 
Outcome Mean & 12.392 & 12.392 & 35.058 & 35.058 & 62.593 & 62.593 & 111.495 & 111.495 \\ 
Observations & 25,309 & 25,309 & 25,309 & 25,309 & 25,309 & 25,309 & 25,309 & 25,309 \\ 
R$^{2}$ & 0.617 & 0.415 & 0.788 & 0.778 & 0.944 & 0.937 & 0.904 & 0.897 \\ 
\hline 
\hline \\[-1.8ex] 
\end{tabular} 
}
    
    \textbf{Panel B: units in log \$ per student}
    
    \makebox[\textwidth][c]{
\begin{tabular}{@{\extracolsep{5pt}}lcccccccc} 
\\[-1.8ex]\hline 
\hline \\[-1.8ex] 
 & \multicolumn{8}{c}{Dependent Variable: Log Faculty Count per Students, by Position} \\ 
\cline{2-9} 
\\[-1.8ex] & \multicolumn{2}{c}{Lecturers} & \multicolumn{2}{c}{Asst. Professors} & \multicolumn{2}{c}{Full Professors} & \multicolumn{2}{c}{All Faculty} \\ 
 & OLS & 2SLS & OLS & 2SLS & OLS & 2SLS & OLS & 2SLS \\ 
\\[-1.8ex] & (1) & (2) & (3) & (4) & (5) & (6) & (7) & (8)\\ 
\hline \\[-1.8ex] 
 State Funding & 0.412 & $-$0.728 & 0.366 & $-$1.716 & 0.394 & $-$0.331 & 0.381 & $-$1.563 \\ 
  & (0.079) & (6.354) & (0.058) & (4.346) & (0.063) & (1.819) & (0.060) & (3.969) \\ 
 \hline \\[-1.8ex] 
Outcome Mean & 0.744 & 0.744 & 1.935 & 1.935 & 3.075 & 3.075 & 5.824 & 5.824 \\ 
Observations & 25,309 & 25,309 & 25,309 & 25,309 & 25,309 & 25,309 & 25,309 & 25,309 \\ 
R$^{2}$ & 0.639 & 0.549 & 0.741 & 0.002 & 0.819 & 0.734 & 0.845 & $-$0.125 \\ 
\hline 
\hline \\[-1.8ex] 
\end{tabular} 
}

    \label{tab:facultycount-shock-private-robustness}
    \justify
    \footnotesize
    \textbf{Note}:
    These tables show the second stage OLS and 2SLS estimates of regression specification \eqref{eqn:secondstage}, among private universities --- as described in \autoref{sec:results-robustness}.
    Each observation is a public university-year, in the IPEDS data.
    Panel A shows the effect of a fall in state funding \$$-1,000$ per student on the number of professors --- i.e.,
    an extra \$$1,000$ per student leads to 6 fewer lecturers according to column 2.
    Panel B shows the effect of a $10$\% change in state funding per student at the university on the 10\% change in the number of professors per students --- i.e.,
    an extra 10\% of state funding per student leads to 4.37\% fewer lecturers per student according to column 2.
    Outcome-mean is the mean of the outcome, for Panel A the number of professors per student, for Panel B the number of faculty per student.
    Panel B uses $\log$ faculty count per student as the outcome, though the outcome mean is count of faculty per student (not in $\log$ terms).
    Standard errors are clustered at the state-year level, and university $+$ year fixed effects are included through--out.
\end{table}

%\newpage
%\subsection{Rates of Substitution}
%\label{sec:appendix-substitution}
%The funding elasticities can be used to recover the marginal rate of substitution between two outcomes.
%For example, write $Y^1$ for the number of lecturers per student at a university, and $Y^2$ for the number of full professors.
%I use the above approaches to estimate the funding elasticities, where $\% \Delta$ denotes percent change.
%\[ \beta_1 = \frac{\% \Delta Y^1}{\% \Delta X}
%\text{, and }
%\beta_2 = \frac{\% \Delta Y^2}{\% \Delta X} \]
%As such, it is possible to recover the elasticity for substitution between lecturers and full professors by the universities via the respective funding elasticities.
%\[ \frac{\% \Delta Y^1}{\% \Delta Y^2}
%= \frac{\% \Delta Y^1 / \% \Delta X}{\% \Delta Y^2 / \% \Delta X}
%= \frac{\beta_1}{\beta_2} \]
%I present results for the rates of substitution between different levels of faculty by this approach, dividing the relevant coefficient estimates and presenting standard errors calculated by a non-parametric bootstrap.
%In practice, this corresponds to division of the estimates of the elasticity for employment of professors (by rank) with respect to state funding, presented in Panel B \autoref{tab:facultycount-shock-reg}, and bootstrapping the results to generate standard errors and confidence intervals.
%% https://stackoverflow.com/questions/63777368/computing-the-standard-error-when-dividing-coefficients-of-different-regressions
%
%The implied marginal rate of substitution between lecturers and assistant professors is estimated as -3.26 (standard error 0.50), based on 10,000 bootstrap samples.
%This means that public universities increased their number lecturers per student by 3.26\% when they decreased their count of assistant professors, on average and subject to the changes in state funding they experienced 1990--2017.
%Between lecturers and full professors the rate of substitution is -3.19 (0.34), which implies that universities substitute between lecturers and full professors in the same way.
%Between assistant and full professors the rate of substitution is 0.99 (0.11), which intuitively implies that universities treated assistant and full professors (i.e., those before and after tenure in the tenure system) as complements.
%
%"Calculated point est for substitution between lecturers + ast profs" 
%                                                                      
%                                                  "-3.26372574229461" 
%                                                                      
%                                                           "with SEs" 
%                                                                      
%                                                  "0.501721750806993" 
%                                                                      
%                                                        "and 95 % CI" 
%                                                                 2.5% 
%                                                  "-4.38380704259523" 
%                                                                97.5% 
%                                                  "-2.40372605889863"

                                                                      
% "Calculated point est for substitution between lecturers + full profs" 
%                                                                       
%                                                    "-3.1876596143219" 
%                                                                       
%                                                            "with SEs" 
%                                                                       
%                                                   "0.340087353132307" 
%                                                                       
%                                                         "and 95 % CI" 
%                                                                  2.5% 
%                                                   "-3.88637781601822" 
%                                                                 97.5% 
%                                                   "-2.56722681973736" 

% "Calculated point est for substitution between ast + full profs" 
%                                                                  
%                                              "0.990674987591129" 
%                                                                  
%                                                       "with SEs" 
%                                                                  
%                                              "0.116874891784044" 
%                                                                  
%                                                    "and 95 % CI" 
%                                                             2.5% 
%                                              "0.771569155168642" 
%                                                            97.5% 
%                                               "1.23151772759759" 

\end{document}
